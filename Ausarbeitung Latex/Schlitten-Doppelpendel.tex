
\documentclass[
	ngerman,
	ruledheaders=section,%Ebene bis zu der die Überschriften mit Linien abgetrennt werden, vgl. DEMO-TUDaPub
	class=report,% Basisdokumentenklasse. Wählt die Korrespondierende KOMA-Script Klasse
	thesis={type=Projektseminar},% Dokumententyp Thesis
	accentcolor=2c,% Auswahl der Akzentfarbe
	custommargins=false,% Ränder werden mithilfe von typearea automatisch berechnet
	marginpar=false,% Kopfzeile und Fußzeile erstrecken sich nicht über die Randnotizspalte
	BCOR=10mm,%Bindekorrektur, falls notwendig
	parskip=half-,%Absatzkennzeichnung durch Abstand vgl. KOMA-Sript
	fontsize=11pt,%Basisschriftgröße laut Corporate Design ist mit 9pt häufig zu klein
  twoside,
  %ignore-missing-data=true,
  IMRAD=false,
  numbers=noendperiod,
  fleqn,
]{tudapub}

% Der folgende Block ist nur bei pdfTeX auf Versionen vor April 2018 notwendig
\usepackage{iftex}
\ifPDFTeX
\usepackage[utf8]{inputenc}%kompatibilität mit TeX Versionen vor April 2018
\fi


%%%%%%%%%%%%%%%%%%%
%Sprachanpassung & Verbesserte Trennregeln
%%%%%%%%%%%%%%%%%%%
\usepackage[english, main=ngerman]{babel}
\usepackage[autostyle]{csquotes}% Anführungszeichen vereinfacht
\usepackage{microtype}


%%%%%%%%%%%%%%%%%%%
%Literaturverzeichnis
%%%%%%%%%%%%%%%%%%%
\usepackage[backend=biber ,bibencoding=utf8]{biblatex}   % Literaturverzeichnis      % backend=biber
\bibliography{bib/literature} % Name der Bib-Datei ! OHNE ENDUNG .BIB !



\usepackage{iflang}
\usepackage{xspace}


\usepackage{bibgerm}		% Für deutsche Literaturverwaltung


\usepackage{graphicx}		% zum Einbinden von Postscript

\usepackage{subfig}			% Für Unterabbildungen
%	\captionsetup[subtable]{position=top}
\usepackage{rotating}		% Zum Drehen von Objekten
\usepackage{placeins}		% Für \FloatBarrier


\usepackage{booktabs}
\usepackage{array}			% Für Zellentyp "m{}" in tabular-Umgebungen (Vertikal zentriert)
\usepackage{ltxtable} 	% Vereinigt TabularX und Longtable
\usepackage{multirow}		% Für mehrzeilige Felder in Tabellen
\usepackage{cellspace}	% Für gescheiten Abstand von Formeln zu Tabellen-
                        % rändern
              

\usepackage{amsmath}		% Mehr mathematischen Formelsatz
\usepackage{amsfonts}
\usepackage{amssymb}
\usepackage{icomma}			% Damit nach Dezimalkommas kein Abstand eingefügt wird
							          % (in math-Umgebungen)
                        
% =========================================================
% Workaround für Problem mit amsmath und doppelten Akzenten
\makeatletter
\protected\def\mathaccentV#1#2#3#4#5%
  {%
    \ifmmode
      \mathaccentV@do{#2}{#3}{#4}{#5}%
    \else
      \@xp\nonmatherr@\csname #1\endcsname
    \fi
  }
\def\mathaccentV@do#1#2#3#4%
  {%
    \global\let\macc@nucleus\@empty
    \mathaccent"\accentclass@#1#2#3{#4}\macc@nucleus
  }
\makeatother
% =========================================================

\usepackage{upgreek}		% Für nicht-kursive kleine griechischen Buchstaben


\usepackage[ngerman]{varioref}







\usepackage{color}


\input{common/include_and_setup_lstlistings.tex}
\usepackage{fp} 
\usepackage{tikz}				% Zum Erzeugen von Bildern mit TikZ
\usepackage{pgf}
\usepackage{pgfplots}			% Zum Erzeugen von Diagrammen mit pgfplots
\usepackage{pgfplotstable}
\usetikzlibrary{arrows}
\usetikzlibrary{arrows.meta}
\usetikzlibrary{backgrounds}
\usetikzlibrary{calc}
\usetikzlibrary{circuits}
\usetikzlibrary{circuits.ee.IEC}
\usetikzlibrary{decorations.markings}
\usetikzlibrary{decorations.shapes}
\usetikzlibrary{decorations.text}
\usetikzlibrary{fadings}
\usetikzlibrary{fit}
\usetikzlibrary{intersections}
\usetikzlibrary{matrix}
\usetikzlibrary{patterns}
\usetikzlibrary{positioning}
\usetikzlibrary{shapes}
\usetikzlibrary{shadows}
\usetikzlibrary{spy}
\IfLanguageName{german}{
	\tikzset{
		/pgf/number format/use comma,
		/pgf/number format/1000 sep=\,
	}
}{}
\IfLanguageName{ngerman}{
	\tikzset{
		/pgf/number format/use comma,
		/pgf/number format/1000 sep=\,
	}
}{}

\usepgfplotslibrary{groupplots}
%\usepgfplotslibrary{fillbetween}

\pgfplotsset{compat=newest}
\IfLanguageName{german}{
	\pgfplotsset{
		x tick label style={/pgf/number format/use comma, /pgf/number format/1000 sep=\,},
		y tick label style={/pgf/number format/use comma, /pgf/number format/1000 sep=\,},
		z tick label style={/pgf/number format/use comma, /pgf/number format/1000 sep=\,}
	}
}{}
\IfLanguageName{ngerman}{
	\pgfplotsset{
		x tick label style={/pgf/number format/use comma, /pgf/number format/1000 sep=\,},
		y tick label style={/pgf/number format/use comma, /pgf/number format/1000 sep=\,},
		z tick label style={/pgf/number format/use comma, /pgf/number format/1000 sep=\,}
	}
}{}

\edef\pgfdatafolder{./Bilder/Daten} % Verzeichnis in dem die csv Dateien für pgfplots liegen

\newcommand{\xmin}{1e-2}
\newcommand{\xmax}{1e2}
\newcommand{\mywidth}{0.8\textwidth}
\newcommand{\myheight}{60mm}

\newcommand{\omegaD}{1}

\newcommand{\bodestyle}{
	\pgfplotsset{
		major grid style={line width=0.3pt, color=gray},
		minor grid style={line width=0.3pt, color=gray},
		major tick style={line width=0.4pt, color=black},
		major tick length={4pt},
		minor tick length={3pt},
		tick label style={font=\small}
	}
}

\newcommand{\nyquiststyle}{
	\pgfplotsset{
		major grid style={line width=0.3pt, color=gray},
		minor grid style={line width=0.3pt, color=gray},
		major tick style={line width=0.4pt, color=black},
		major tick length={4pt},
		minor tick length={3pt},
		tick label style={font=\small}
	}
}

\newcommand{\plotstyle}{
	\pgfplotsset{
		major grid style={line width=0.3pt, color=gray},
		minor grid style={line width=0.3pt, color=gray},
		major tick style={line width=0.4pt, color=black},
		major tick length={4pt},
		minor tick length={3pt},
		tick label style={font=\small}
	}
}

\newcommand{\plotyystyle}{
	\pgfplotsset{
		every non boxed y axis/.style={ytick align=inside,y axis line style={-}},
		every boxed y axis/.style={}
	}
}

\newenvironment{bodeAmpDB}[1][]{
\bodestyle
\begin{semilogxaxis}[
	ylabel=$|G(\mathrm{j}\omega)|_{\mathrm{dB}}$,
	%ylabel=Amplitude in dB,
	ylabel style={yshift=2pt},
	enlarge x limits=false,
	xminorgrids=true,
	xmajorgrids=true,
	ymajorgrids=true,
	yminorgrids=true,
	xticklabels=\empty,
	width=\mywidth,
	height=\myheight,#1]
}{\end{semilogxaxis}}

\newenvironment{bodeAmpLOG}[1][]{
\begin{loglogaxis}[
	ylabel=$|G(\mathrm{j}\omega)|$,
	%ylabel=Amplitude,
	ylabel style={yshift=2pt},
	enlarge x limits=false,
	xminorgrids=true,
	xmajorgrids=true,
	ymajorgrids=true,
	yminorgrids=true,
	xticklabels=\empty,
	y tick label style={font=\small},
	width=\mywidth,
	height=\myheight,#1]
}{\end{loglogaxis}}

\newenvironment{bodePhase}[1][]{
\begin{scope}[yshift=-\myheight+12mm]	% zweiten Plot (Phase) unter den Amplitudengang setzen
	\bodestyle
	\begin{semilogxaxis}[
		xlabel=Frequenz $\omega$ in \ensuremath{\mathrm{{\frac{rad}{s}}}},
		%ylabel=Phase in $^\circ$,
		ylabel=$\angle{G(\mathrm{j}\omega)}$,
		ylabel style={yshift=2pt},
		enlarge x limits=false,
		xminorgrids=true,
		xmajorgrids=true,
		ymajorgrids=true,
		yminorgrids=true,
		ytick={-360, -315,..., 360},
		yticklabel={$\pgfmathprintnumber{\tick}^\circ$},% ° als Einheitenzeichen an alle yticks
		width=\mywidth,
		height=\myheight,#1]
}{\end{semilogxaxis}\end{scope}}

\newenvironment{plotyyLeft}[1][]{
\plotyystyle
\begin{axis}[
	scale only axis,
	enlarge x limits=0,
	axis y line=left,
	width=\mywidth,
	height=\myheight,#1]
}{\end{axis}}

\newenvironment{plotyyRight}[1][]{
\plotyystyle
\begin{axis}[
	scale only axis,
	enlarge x limits=0,
	axis y line=right,
	axis x line=none,
	width=\mywidth,
	height=\myheight,#1]
}{\end{axis}}

\newcommand{\TikZpole}[1][blue]{
\begin{tikzpicture}
	\draw[#1, very thick, line cap=round] (-1,1)  -- (1,-1);
	\draw[#1, very thick, line cap=round] (-1,-1) -- (1,1);
\end{tikzpicture}
}

\newcommand{\TikZzero}[1][blue]{
\begin{tikzpicture}
	\draw[#1, very thick] (0,0) circle (5pt);
\end{tikzpicture}
}

% Blockschaltbilder
\newcommand{\TikZscale}{1}
\newcommand{\mm}{*\TikZscale mm}
\input{common/TikZ_BSBnormal.tex}



% Diese Datei dient zum Definieren nützlicher Befehle.
% Sie soll lediglich als Beispiel dienen, wie Befehle definiert werden, und welche Befehle nützlich sein können.

% Inhalt
% ======
%	Makros für Referenzen (Abbildungen, Zitate, ...)
%	Makros für Abbildungen
%	Makros für Einheiten, Exponenten
%	Makros für Formeln
%	Makros für Entwurf
%   Definitionen für Umgebungen

% Allgemeine Abkürzungen
% ======================
	\newcommand{\bzw}{bzw.\@\xspace}
	\newcommand{\Bzw}{Bzw.\@\xspace}
	\newcommand{\bzgl}{bzgl.\@\xspace}
	\newcommand{\ca}{ca.\@\xspace}
	\newcommand{\dah}{d.\thinspace{}h.\@\xspace}
	\newcommand{\Dah}{D.\thinspace{}h.\@\xspace}
	\newcommand{\ds}{d.\thinspace{}s.\@\xspace}
	\newcommand{\evtl}{evtl.\@\xspace}
	\newcommand{\ua}{u.\thinspace{}a.\@\xspace}
	\newcommand{\Ua}{U.\thinspace{}a.\@\xspace}
	\newcommand{\uU}{u.\thinspace{}U.\@\xspace}
	\newcommand{\UU}{U.\thinspace{}U.\@\xspace}
	\newcommand{\usw}{usw.\@\xspace}
	\newcommand{\etc}{etc.\@\xspace}
	\newcommand{\va}{v.\thinspace{}a.\@\xspace}
	\newcommand{\Vgl}{Vgl.\@\xspace}
	\newcommand{\vgl}{vgl.\@\xspace}
	\newcommand{\zB}{z.\thinspace{}B.\@\xspace}
	\newcommand{\ZB}{Zum Beispiel\xspace}
	\newcommand{\sa}{s.\thinspace{}a.\@\xspace}
	\newcommand{\ia}{i.\thinspace{}a.\@\xspace}
	\newcommand{\bspw}{bspw.\@\xspace}
	\newcommand{\Bspw}{Bspw.\@\xspace}
	\newcommand{\ggf}{ggf.\@\xspace}
	\newcommand{\Ggf}{Ggf.\@\xspace}
	\newcommand{\zT}{z.\thinspace{}T.\@\xspace}
	\newcommand{\ZT}{Z.\thinspace{}T.\@\xspace}
	\newcommand{\iA}{i.\thinspace{}A.\@\xspace}
	\newcommand{\IA}{I.\thinspace{}A.\@\xspace}

	\newcommand{\ie}{i.\thinspace{}e.\@\xspace}
	\newcommand{\Ie}{I.\thinspace{}e.\@\xspace}
	\newcommand{\eg}{e.\thinspace{}g.\@\xspace}






% Makros für Referenzen (Abbildungen, Zitate, ...)
% ================================================

	% Referenzierung auf Abbildungen, Tabellen, etc. (Hyperref-fähig)
	\newcommand{\figref}[1]{\hyperref[#1]{\figurename\ \ref*{#1}}}
	\newcommand{\tabref}[1]{\hyperref[#1]{\tablename\ \ref*{#1}}}
	\newcommand{\equref}[1]{\hyperref[#1]{Gl.~(\ref*{#1})}}
	\newcommand{\defref}[1]{\hyperref[#1]{Definition~\ref*{#1}}}
	\newcommand{\thrref}[1]{\hyperref[#1]{Satz~\ref*{#1}}}
	\newcommand{\figvref}[1]{\hyperref[#1]{\figurename\ }\vref{#1}}
	\newcommand{\tabvref}[1]{\hyperref[#1]{\tablename\ }\vref{#1}}
	\newcommand{\eqvref}[1]{\hyperref[#1]{Gl.~(\ref*{#1}) auf Seite~\pageref*{#1}}}
	\newcommand{\pagerefh}[1]{\hyperref[#1]{Seite~\pageref*{#1}}}
	
	\newcommand{\charef}[1]{\hyperref[#1]{\chaptername~\ref*{#1}}}
  \newcommand{\secref}[1]{\hyperref[#1]{Abschnitt~\ref*{#1}}}
	\newcommand{\appref}[1]{\hyperref[#1]{Anhang~\ref*{#1}}}

	\newcommand{\lstref}[1]{\hyperref[#1]{Listing~\ref*{#1}}}
	\newcommand{\algoref}[1]{\hyperref[#1]{Algorithmus~\ref*{#1}}}
	\newcommand{\ftnref}[1]{\hyperref[#1]{Fußnote~\ref*{#1}}}
  
  
% Makros für Abbildungen
% ======================

% Textbausteine
% =============
	% Produktnamen
	\newcommand*{\Matlab}{\textsc{Matlab}}
	\newcommand*{\Matlabreg}{\textsc{Matlab}\textsuperscript{\tiny \textregistered}}
	\newcommand*{\MatSim}{\textsc{Matlab/Simulink}}
	\newcommand*{\Simulink}{\textsc{Simulink}}
	\newcommand*{\Simulinkreg}{\textsc{Simulink}\textsuperscript{\tiny \textregistered}}
	
	% Das Makro |\name|\marg{person} formatiert einen Personennamen bspw. eines Erfinders oder Entdeckers gemäß |\name{Euler}| \arrow\ \name{Euler}.
	\newcommand*{\name}[1]{\textsc{#1}}



% Makros für Einheiten, Exponenten
% ================================

	\newcommand*{\unit}[1]{\ensuremath{\mathrm{#1}}}
	
	% Wert mit Einheit (mit kleinem Leerzeichen dazwischen), aus Text- UND Math-Modus
	\newcommand*{\valunit}[2]{\ensuremath{#1\,\mrm{#2}}}


	% "°C", im Text- oder Mathe-Modus
	\newcommand*{\degC}{
		\ifmmode
			^\circ \mrm{C}%
		\else
			\textdegree C%
		\fi}

	\newcommand*{\degree}{
		\ifmmode
			^\circ%
		\else
			\textdegree%
		\fi}
	
	% Für Exponentenschreibweise ( Anwendung: 123\E{3} )
	\newcommand*{\E}[1]{\ensuremath{\cdot 10^{#1}}}
	
	\newcommand*{\eexp}[1]{\ensuremath{\mathrm{e}^{#1}}}
	\newcommand*{\iu}{\ensuremath{\mathrm{j}}}

	\newcommand*{\todots}{\ensuremath{,\,\hdots,\,}}


% Makros für Formeln
% ==================

	% Definition für Vektor und Matizen
    \newcommand*{\mat}[1]{{\ensuremath{\boldsymbol{\mathrm{#1}}}}}
    \newcommand*{\ma}[1]{{\ensuremath{\boldsymbol{\mathrm{#1}}}}}
    \newcommand*{\mas}[1]{\ensuremath{\boldsymbol{#1}}}
    \newcommand*{\ve}[1]{\ensuremath{\boldsymbol{#1}}}
    \newcommand*{\ves}[1]{\ensuremath{\boldsymbol{\mathrm{#1}}}}

	\newcommand*{\AP}{\ensuremath{\mathrm{AP}}}
	\newcommand*{\doti}{\ensuremath{(i)^\cdot}}
	
	\newcommand*{\inprod}[2]{\ensuremath{\langle #1,\,#2 \rangle}}
	
	\newcommand*{\ul}[1]{\underline{#1}}

	% gerades "d" (z.B. für Integral)
	\newcommand*{\ud}{\ensuremath{\mathrm{d}}}
	
	% normaler Text in Formeln
	\newcommand*{\tn}[1]{\textnormal{#1}}
	
	% nicht-kursive Schrift in Formeln
	\newcommand*{\mrm}[1]{\ensuremath{\mathrm{#1}}}
	
	% gerades "T" für Transponiert
	\newcommand*{\transp}{\ensuremath{\mathrm{T}}}
	
	% gerades "rg"
	\newcommand*{\rang}{\ensuremath{\operatorname{rg}}}

	% Für geklammerte Ausdrücke mit Index (Subscript)
	% (einmal mit kursiven Index, einmal mit geradem Index)
	\newcommand*{\grpsb}[2]{\ensuremath{\left(#1\right)_{#2}}}
	\newcommand*{\grprsb}[2]{\ensuremath{\left(#1\right)_{\mathrm{#2}}}}

	% Ableitungen und Integrale
		% "normale" Ableitung (mit geraden "d"s)
		\newcommand*{\normd}[2]{\ensuremath{\frac{\mathrm{d}#1}{\mathrm{d}#2}}}
		\newcommand*{\normdat}[3]{\ensuremath{\left.\frac{\mathrm{d} #1}{\mathrm{d} #2}\right|_{#3}}}
	
		% Materielle Ableitung
		\newcommand*{\matd}[2]{\ensuremath{\frac{\mathrm{D} #1}{\mathrm{D} #2}}}
		\newcommand*{\matdat}[3]{\ensuremath{\left.\frac{\mathrm{D} #1}{\mathrm{D} #2}\right|_{#3}}}
	
		% Partielle Ableitung
		\newcommand*{\partiald}[2]{\ensuremath{\frac{\partial #1}{\partial #2}}}
		\newcommand*{\partialdat}[3]{\ensuremath{\left.\frac{\partial #1}{\partial #2}\right|_{#3}}}
	
	
	% Transformationen
	\newcommand*{\FT}[1]{\ensuremath{\mathcal{F}\left(#1\right)}}
	\newcommand*{\FTabs}[1]{\ensuremath{\left|\mathcal{F}\left(#1\right)\right|}}
	\newcommand*{\IFT}[1]{\ensuremath{\mathcal{F}^{-1}\left(#1\right)}}
	\newcommand*{\DFT}[1]{\ensuremath{\mathrm{DFT}\left(#1\right)}}
	\newcommand*{\DFTabs}[1]{\ensuremath{\left|\mathrm{DFT}\left(#1\right)\right|}}
	\newcommand*{\Laplace}[1]{\ensuremath{\mathcal{L}\left(#1\right)}}
	\newcommand*{\InvLaplace}[1]{\ensuremath{\mathcal{L}^{-1}\left(#1\right)}}
	\newcommand*{\invtrans}{\ensuremath{\quad\bullet\!\!-\!\!\!-\!\!\circ\quad}}
	\newcommand*{\trans}{\ensuremath{\quad\circ\!\!-\!\!\!-\!\!\bullet\quad}}


	% Manche textcomp-Zeichen funktionieren mit dem TU-Design nicht, diese können dann mit diesem
	% Befehl gesetzt werden.
	\newcommand*{\textcompstdfont}[1]{{\fontfamily{cmr} \fontseries{m} \fontshape{n} \selectfont #1}}
	


% =================================================================================
% Definitionen für Tabellen
% =================================================================================
% Spaltenstil zum Ausrichten von Zahlen am Dezimaltrennzeichen
\newcolumntype{d}[1]{D{.}{,}{#1}}

%\newcolumntype{L}[1]{>{\raggedright\arraybackslash}m{#1}}
%\newcolumntype{C}[1]{>{\centering\arraybackslash}m{#1}}
%\newcolumntype{R}[1]{>{\raggedleft\arraybackslash}m{#1}}



\usepackage[straightvoltages]{circuitikz}


\newcommand*{\Miktex}{\textsc{MiKTeX}\xspace}
\newcommand*{\texlive}{\textsc{texlive}\xspace}
\newcommand{\Texstudio}{\textsc{TeXStudio}\xspace}
\newcommand{\Texniccenter}{\textsc{TeXnicCenter}\xspace}
\newcommand{\Sumatra}{\textsc{Sumatra}\xspace}
\newcommand*{\zitat}[1]{\glqq{}#1\grqq{}}


\captionsetup{format=plain,labelsep=colon,justification=centering}
\setkomafont{caption}{\sffamily}
\setkomafont{captionlabel}{\bfseries\sffamily}
%\KOMAoption{captions}{justification=centering}

\addtokomafont{section}{\fontsize{12}{18}\selectfont}


\renewcommand{\ttdefault}{lcmtt}
  

\begin{document}

\Metadata{
	title=Steuerung und Regelung eines Doppelpendels,
	author=Tobias Gebhard und Frederik Tesar,
}

\title{Steuerung und Regelung eines Doppelpendels}
\author[T. Gebhard, F. Tesar]{Tobias Gebhard \and Frederik Tesar}%optionales Argument ist die Signatur, 
%\birthplace{Geburtsort}%Geburtsort, bei Dissertationen zwingend notwendig
\reviewer{Prof. Dr.-Ing. Ulrich Konigorski \and Dr.-Ing. Eric Lenz}%Gutachter

%Diese Felder erden untereinander auf der Titelseite platziert. 
%\department ist eine notwendige Angabe, siehe auch dem Abschnitt `Abweichung von den Vorgaben für die Titelseite'
\department{etit} % Das Kürzel wird automatisch ersetzt und als Studienfach gewählt, siehe Liste der Kürzel im Dokument.
%\institute{Institut}

\addTitleBox{\includegraphics[width=\linewidth]{common/rtm_mit_schrift}}

\submissiondate{\today}

\maketitle

\affidavit

\tableofcontents

\cleardoublepage



\newcommand{\dpd}{Doppel-Pendel}
\newcommand{\spd}{Schlitten-Pendel}
\newcommand{\spds}{Schlitten-Pendel-System}
\newcommand{\crb}{Coulomb-Reibung}
\newcommand{\drb}{viskose Reibung}
\newcommand{\koor}{Koordinaten}
\newcommand{\bwgl}{Bewegungsgleichungen}
\newcommand{\zrm}{Zustandsraummodell}
\newcommand{\nlzrm}{nichtlineare \zrm}
\newcommand{\ap}{Arbeitspunkt}
\newcommand{\aprg}{Arbeitspunkt-Regelung}
\newcommand{\syp}{Systemparameter}
\newcommand{\traj}{Trajektorie}
\newcommand{\lin}{Linearisierung}
\newcommand{\bss}{Beschleunigungssystem}
\newcommand{\krs}{Kraftsystem}
\newcommand{\ew}{Eigenwert}
\newcommand{\ewe}{Eigenwerte}
\newcommand{\zsr}{Zustandsregler}
\newcommand{\beob}{Beobachter}

\newcommand{\einhalb}{\frac{1}{2}}
\newcommand{\pii}{\ensuremath{\pi}}

\newcommand{\xoo}{\ensuremath{x_{0,0}}}
\newcommand{\xor}{\ensuremath{x_{0R}}}
\newcommand{\xo}{\ensuremath{x_0}}  % Zahlen im Befehl nicht möglich
\newcommand{\xe}{x_1}
\newcommand{\xz}{x_2}
\newcommand{\yo}{y_0}
\newcommand{\ye}{y_1}
\newcommand{\yz}{y_2}
\newcommand{\xop}{\ensuremath{\dot{x}_0}}
\newcommand{\xopp}{\ensuremath{\ddot{x}_0}}
\newcommand{\xep}{\dot{x}_1}
\newcommand{\xzp}{\dot{x}_2}
\newcommand{\yop}{\dot{y}_0}
\newcommand{\yep}{\dot{y}_1}
\newcommand{\yzp}{\dot{y}_2}
\newcommand{\phe}{\ensuremath{\varphi_1}}
\newcommand{\phz}{\ensuremath{\varphi_2}}
\newcommand{\phep}{\ensuremath{\dot{\varphi}_1}}
\newcommand{\phepp}{\ddot{\varphi}_1}
\newcommand{\phzp}{\ensuremath{\dot{\varphi}_2}}
\newcommand{\phzpp}{\ddot{\varphi}_2}
\newcommand{\Fc}{F_c}
\newcommand{\Fco}{\ensuremath{F_{c0}}}
\newcommand{\Mce}{M_{c1}}
\newcommand{\Mcz}{M_{c2}}
\newcommand{\Mceo}{\ensuremath{M_{c10}}}
\newcommand{\Mczo}{\ensuremath{M_{c20}}}
\newcommand{\xopth}{\ensuremath{\dot{x}_{0,c76}}}
\newcommand{\pheth}{\ensuremath{\dot{\varphi}_{1,c76}}}
\newcommand{\phzth}{\ensuremath{\dot{\varphi}_{2,c76}}}
\newcommand{\Fd}{F_d}
\newcommand{\Mde}{M_{d1}}
\newcommand{\Mdz}{M_{d2}}


\newcommand{\vex}{\ve{x}}
\newcommand{\vexr}{\ve{x}_R}
\newcommand{\vef}{\ve{f}}
\newcommand{\vexp}{\ve{\dot{x}}}
\newcommand{\ddt}{\frac{\ud}{\ud t}}

\newcommand{\api}[1]{\ensuremath{\AP_{#1}}}
\newcommand{\ape}{\api{1}}
\newcommand{\apz}{\api{2}}
\newcommand{\apd}{\api{3}}
\newcommand{\apv}{\api{4}}

\newcommand{\sign}[1]{\ensuremath{\mrm{sign}\left(#1\right)}}

\definecolor{grey}{RGB}{150,150,150}




\newcommand{\qq}[1]{\glqq#1\grqq}
\newcommand{\nv}{Newton-Verfahren}
\newcommand{\jm}{Jakobi-Matrix}
\newcommand{\alg}{Algorithmus}
\newcommand{\ml}{\Matlab}
\newcommand{\rf}{Redundanzfaktor}
\newcommand{\std}{Standardabweichung}
\newcommand{\gls}{Gleichungssystem}
\newcommand{\urll}{\textsc{url: }}

\newcommand{\exref}[1]{Beispiel \ref{#1}}

\newcommand{\valdeg}[1]{\valunit{#1}{\degree}}
\newcommand{\diag}[1]{\ensuremath{\mrm{diag}\left(#1\right)}}
\newcommand{\diaga}{\ensuremath{\mrm{diag} \,}}
\newcommand{\spur}{\ensuremath{\operatorname{sp}}}
\newcommand{\rg}[1]{\ensuremath{\rang #1}}
\newcommand{\rgk}[1]{\ensuremath{\rang \left(#1\right)}}
\newcommand{\spu}[1]{\ensuremath{\spur #1}}
\newcommand{\spuk}[1]{\ensuremath{\spur \left(#1\right)}}

\newcommand{\dx}[1]{\delta_{#1}}
\newcommand{\dd}{\delta}
\newcommand{\di}{\delta_i}
\newcommand{\dij}{\delta_i-\delta_j}
\newcommand{\dik}{\delta_i-\delta_k}
\newcommand{\dil}{\delta_i-\delta_l}
\newcommand{\doj}{\delta_1-\delta_j}
\newcommand{\dnj}{\delta_\n-\delta_j}
\newcommand{\cosdi}{\cos(\di)}

\newcommand{\yij}{\clx{Y}_{ij}}
\newcommand{\rij}{r_{ij}}
\newcommand{\pbi}{p_i}
\newcommand{\pij}{\ensuremath{p_{ij}}}

\chapter{Einleitung}\label{cha:intro}

Das Schlittendoppelpendel ist ein nichtlineares System, an dem interessante steuerungs- und regelungstechnische Verfahren untersucht werden können. 
Am Fachgebiet Regelungstechnik und Mechatronik (rtm) der TU Darmstadt wird dazu ein Versuchsstand betrieben, der Untersuchungen und Demonstrationen am Einfach- und Doppelpendel ermöglicht. 

Eine Übersicht über die in der Vergangenheit durchgeführten Arbeiten zu diesem Versuchsstand kann Chang \cite{chang} entnommen werden. Im Zuge einer Neukonstruktion des Pendels im Jahr 2019 durch Chang \cite{chang} haben sich die Systemparameter geändert. Erste Erfahrungen mit dem neuen System zeigen im Vergleich zu den Vorherigen jedoch ein ungünstigeres Verhalten in Bezug auf die regelungstechnischen Eigenschaften. Dies legt eine Untersuchung des Einflusses der Systemparameter auf die Regelbarkeit des Systems nahe.

Da es bei der Trajektorienfolgeregelung aufgrund der Beschränkung einiger Systemzustände immer wieder zu Schwierigkeiten bei der Berechnung und Umsetzung von Trajektorien kam, wurde durch Fauvé \cite{fauve} der Ansatz der Modellprädiktiven Regelung (engl.: \textit{Model Predictive Control})(MPC), die im Gegensatz zu den klassischen Regelungsverfahren eine Berücksichtigung der Systembeschränkungen ermöglicht, für das bestehende Pendelsystem untersucht. Obwohl sich der Ansatz aufgrund ungenügender Echtzeitfähigkeit nicht für die Regelung eignete, erwies er sich in Bezug auf die Berechnung von Trajektorien für Arbeitspunktwechsel als vielversprechend. Erste Versuche die erfolgreich berechneten Trajektorien in der Simulation mit Hilfe einer Trajektorienfolgeregelung zu stabilisieren, gelangen jedoch nicht. Daher bleibt zu zeigen, dass die mittels NMPC gefundenen Trajektorien am System stabilisiert werden können. Zudem wird vermutet, dass neben der Regelbarkeit auch das Finden von Trajektorien durch die Systemparameter beeinflusst wird. 

Ziel dieser Arbeit ist daher die Untersuchung des Einflusses der Systemparameter auf Steuerung und Regelung des Schlittendoppelpendels. Das System wird dazu im Rahmen dieser Arbeit rein simulativ betrachtet. Aus diesem Grund soll in einem ersten Schritt ein ausführliches Simulationsmodell des Versuchsstands aufgebaut werden. Da in der Vergangenheit mehrfach verschiedene Systemparameter angenommen wurden, soll nun ein begründeter Parameterstand unter Berücksichtigung der vorherigen Arbeiten recherchiert werden. Anschließend ist eine Regelung auf Basis der Vorgängerarbeiten auszulegen, mit deren Hilfe auch die Untersuchung des Einflusses der Systemparameter auf die Regelbarkeit des Systems in den vier typischen Arbeitspunkten des Doppelpendels erfolgen soll. Der Einfluss wird zudem auch in Bezug auf die Trajektorienberechnung untersucht. Hierzu ist die bestehende NMPC für die Parameteruntersuchungen sowie den zukünftigen Einsatz als Instrument für die Trajektoriensuche weiterzuentwickeln. Um an die Arbeit von Fauvé \cite{fauve} anzuknüpfen, soll zunächst eine Trajektorienfolgeregelung entworfen werden, mit der zu zeigen ist, dass die mittels NMPC gefundenen Trajektorien in der Simulation stabilisiert werden können. 

\chapter{Grundlagen}\label{cha:grdl}


\input{Kapitel/Grundlagen_MPC.tex}


\input{Kapitel/Grundlagen_Sim.tex}
\chapter{Modellierung}\label{cha:modell}

In diesem Kapitel wird die Modellierung des Gesamtsystems erläutert und auf dessen Implementierung in \ml\ und \Simulink\ eingegangen.

\section{Modell des \spd-Systems}

Die Modellierung des \spd-Systems orientiert sich zunächst an den Modellen der vergangenen Arbeiten. Diese bezogen sich meist auf die Herleitung von \cite{modpen}. Dabei gibt es die Variante \emph{Kraftsystem}, das als Eingang die Kraft annimmt, welche am Schlitten wirkt, sowie das vereinfachte \emph{Beschleunigungssystem}, das direkt die Beschleunigung des Schlittens als Eingang erhält.

\subsection{Koordinaten}

%\begin{figure}[bp]
	%\centering
		%\includegraphics[width=0.7\textwidth]{Bilder/intro.pdf}
	%\caption{Idee des zentralen Reglerentwurfs}
	%\label{fig:intro}
%\end{figure}

Die Minimal-Koordinaten sind 
\begin{align*}
	q_0 = x_0  \\
	q_1 = \varphi_1  \\
	q_2 = \varphi_2
\end{align*}

\subsection{Herleitung der Bewegungsgleichungen}

Um auf die Bewegungsgleichungen des Systems zu gelangen, wird in \cite{modpen} der \emph{Lagrange}-Formalismus verwendet.

\subsection{Ruhelagen}



\subsection{\crb}

Da die \crb sowohl des Schlittens als auch der Pendelstäbe einen wesentlichen Einfluss zu haben scheint, darf diese nicht vernachlässigt werden. In den bisherigen Modellierungen wurde höchstens die \crb des Schlittens berücksichtigt. Da jedoch durch die Neukonstruktion des \dpd s die Messsignalübertragung (zur Vermeidung einer Kabelaufwickelung) über einen Schleifring realisiert wurde, besteht die Vermutung, dass dieser für eine erhöhte \crb verantwortlich ist. Dies würde das System bereits um einen sehr kleinen Arbeitsbereich nicht-linear machen, was die Regelung erschwert.

Im vorigen Projektseminar \cite{ribeiro} wurde die Reibung der Pendelstäbe mittels Identifikation ermittelt, aufgeteilt auf den viskosen und den Coulombanteil.

Die Formel der Gleitreibung lautet eigentlich 
	\[
	F_c = F_{c0}  \cdot  \sign{\dot{x}} ,
\]
allerdings führt diese Implementierung aufgrund der signum-Funktion zu Komplikationen in der Simulation. Daher wird der Verlauf bei sehr niedrigen Geschwindigkeiten mit der $\tanh$ Funktion angenähert.


\section{Motor-Modell}\label{sec:mot}

Das Motormodell besteht aus den drei Baugruppen Spannungs-Strom-Wandler, Gleichstrommotor und Getriebe. Die Modellierung der Baugruppen basiert auf den in Franke \cite{franke} erstmalig aufgestellten Gleichungen, die auch in den nachfolgenden Arbeiten zum Versuchsstand zur Anwendung kamen. Das Modell des Gleichstrommotors wird im Rahmen dieser Arbeit nun zusätzlich um die Berücksichtigung der Gegeninduktion des Motors erweitert.

\subsection{Spannungs-Strom-Wandler}

Bei dem am Versuchsstand eingesetzten Spannungs-Strom-Wandler handelt es sich um einen Servoverstärker, der ursprünglich zur Drehzahlregelung von Gleichstrommotoren vorgesehen war. Entsprechend folgt der Verstärker dem Prinzip einer übergeordneten Drehzahlregelung mit unterlagerter Stromregelung. Für die Anwendung am Versuchsstand ist der übergeordnete Drehzahlregelkreis jedoch aufgetrennt und in einen Spannungs-Strom-Wandler umfunktioniert worden. Dieser dient zur Vorgabe eines konstanten Ankerstroms mittels Pulsweitenmodulation (PWM) der Zwischenkreisspannung des Wandlers proportional zur eingehenden Steuerspannung. Das stationäre Verhalten kann daher durch einen Proportionalitätsfaktor $K_{UI}$ beschrieben werden.
\begin{align}
	I_a = K_{UI} \cdot U_{\mrm{Steuer}}
\end{align}

Gemäß Franke \cite{franke} lässt sich die Dynamik des Wandlers durch ein \mrm{PT_1}-Glied modellieren, sodass sich für den Wandler im Laplace-Bereich die Gleichung
\begin{align} \label{eq:UI}
	I_a(s) = \frac{K_{UI}}{1+T_{UI} \cdot s} \cdot U_{\mrm{Steuer}}(s)
\end{align}
ergibt.


\subsection{Gleichstrommotor}\label{subsec:dcMotor}

Bei dem verwendeten Motor handelt es sich um eine fremderregte Gleichstrommaschine, wobei die magnetische Erregung durch einen Permanentmagneten erzeugt wird \cite{franke}. Das elektromagnetische Drehmoment des Gleichstrommotors ist näherungsweise proportional zum Ankerstrom \cite{binder}. Hierbei wird vorausgesetzt, dass keine magnetische Sättigung vorliegt.
\begin{align}
	M_e = K_I \cdot I_a
\end{align}

Neben dem elektromagnetischen Drehmoment sind außerdem parasitäre Reibmomente zu berücksichtigen. Die Modellierung der Reibung von Motor und Getriebe wird zusammen mit der Schlittenreibung in \secref{sec:spdModell} behandelt. Ein rückwirkendes Moment durch die Federkraft des Riemens wird auf Grund der Annahme unendlicher Riemensteifigkeit vernachlässigt. 

Weiterhin wird das Drehmoment durch die Gegeninduktion geschwächt. Dieser Effekt ist in den Motormodellen von Franke \cite{franke} und den Nachfolgearbeiten bisher nicht berücksichtigt worden. Da die Erfahrung am realen Versuchsstand jedoch gezeigt hat, dass eine Modellierung der Gegeninduktion sinnvoll erscheint, wird diese im Rahmen dieser Arbeit in das Motormodell integriert.

Zum besseren Verständnis des Effekts wird zunächst das physikalische Prinzip der Gegeninduktion betrachtet. Fließt ein Strom durch den ruhenden Anker, der sich im Magnetfeld der Permanentmagneten befindet, so wirkt senkrecht zu den Richtungen des Stroms und des Magnetfelds die Lorenzkraft auf die in der Ankerwicklung befindlichen Ladungsträger (Drei-Finger-Regel bzw. Rechte-Hand-Regel in technischer Stromrichtung). Durch das entstehende Drehmoment beginnt der Anker zu rotieren. Auf Grund der Rotation bewegen sich die Ladungsträger nun zusätzlich zur eigentlichen Stromrichtung auch in Rotationsrichtung des Ankers. Auf diese Bewegungskomponente kann nun erneut das Prinzip der Lorentzkraft angewendet werden. Die resultierende zusätzliche Kraftkomponente, die in Abhängigkeit der Rotationsgeschwindigkeit auf die Ladungsträger wirkt, zeigt nun gegen die Stromrichtung (Lenz'sche Regel). Der resultierenden Ladungsbeschleunigung kann ein Spannungsabfall über der Ankerwicklung zugeordnet werden, die sogenannte induzierte Gegenspannung oder Gegen-EMK (Gegen-Elektromagnetische-Kraft). Der in Folge sinkende Ankerstrom reduziert das Drehmoment des Motors. \cite{binder}

Am Versuchsstand wird dieser Effekt bei geringen Winkelgeschwindigkeiten durch den Stromregler des Spannungs-Strom-Wandlers kompensiert. Ab einer bestimmten Winkelgeschwindigkeit bei konstantem Sollstrom wird die induzierte Gegenspannung größer als die maximale Zwischenkreisspannung des Spannungs-Strom-Wandlers, sodass nicht mehr ausgeregelt werden kann. Nun nimmt der Ankerstrom und damit das Drehmoment mit steigender Winkelgeschwindigkeit ab bis die Leerlaufdrehzahl erreicht ist. 

\begin{figure}[htbp]
	\centering
		\includegraphics[width=0.50\textwidth]{Bilder/Motor/Stromkennlinie.pdf}
	\caption{Stromkennlinie}
	\label{fig:Stromkennlinie}
\end{figure}

Auf Grund der integrierten Strombegrenzung zum Schutz von Motor und Verstärker wird der vom Wandler bereitstellbare Strom zusätzlich begrenzt. Der dadurch parametrierte Maximalstrom des Wandlers wird durch den Stromregler zunächst konstant gehalten bis die Strombegrenzung der Gegen-EMK ab einer bestimmten Winkelgeschwindigkeit $\omega_\mrm{EMK}$ die des Wandler unterschreitet (siehe Abbildung \ref{fig:Stromkennlinie}). An dieser Stelle tritt ein "`Knick"' im Stromverlauf auf. Für den Stromregler wird dabei vereinfachend angenommen, über ausreichend hohe Dynamik und Stellenergie zu verfügen, um den Strom bis zur EMK-Grenze zu jedem Zeitpunkt konstant halten zu können.

Es werden zunächst nur positive Winkelgeschwindigkeiten betrachtet. 
Der Abschnitt des Verlaufs in Abbildung \ref{fig:Stromkennlinie} mit konstantem maximalen Strom wird durch
\begin{equation}
	I_a  = I_{w\mrm{, max}} = \mrm{const.}, 0 \leq \omega \leq \omega_{\mrm{EMK}}
	\label{eq:Iconst}
\end{equation}
beschrieben, wobei $I_{w\mrm{, max}}$ der Begrenzungsstrom des Wandlers ist.

\begin{figure}[h]
	\centering
		\begin{circuitikz}[european]
	\draw (0,0) 
		to[short, -, i=$I_a$] (1,0)
		to[R=$R$] (3.5,0)
		to[L=$L$] (6,0)
		to[V=$U_i$] (6,-2)
		-- (0,-2);
	\draw
		node[ocirc] (A) at (0,0) {}
		node[ocirc] (B) at (0,-2) {};
		%(A) to[open, v=$U_a$] (B);	
		\begin{scope}[shorten >= 10pt,shorten <= 10pt]
			\draw[->] (A) -- node[left] {$U_a$} (B);
		\end{scope}  
\end{circuitikz}
	\caption{Ersatzschaltbild eines Gleichstrommotors}
	\label{fig:dcESB}
\end{figure}

Der Verlauf des Bereichs mit zeitlich abnehmender maximaler Stromstärke wird aus dem Ersatzschaltbild des Gleichstrommotors in Abbildung \ref{fig:dcESB} hergeleitet. Aus dem zweiten Kirchhoff'schen Gesetz (Maschenregel) ergibt sich
\begin{equation}
	U_a =  RI_a + L \normd{I_a(t)}{t} + U_i \ . 
	\label{eq:dcMasche}
\end{equation}

Für die Kennlinie des maximalen Stroms liegt der Tastgrad (engl.: \textit{duty cylcle}) der pulsweitenmodulierten Ankerspannung bei $100\%$, sodass 
\[
	U_{a \mrm{, max}}  = const.
\]
gilt. Die elektrische Zeitkonstante des Motors, die mit den Angaben des Datenblatts aus Franke \cite{franke} zu 
\[
	\tau_{\mrm{el}} = \frac{L}{R} \approx 0,0023 \mrm{s} \ll 1 \mrm{s} \ ,
\]
berechnet werden kann, ist ausreichend klein, dass näherungsweise von stationärem Betrieb ausgegangen werden kann und für die selbstinduzierte Spannung 
\[
	L \normd{I_a(t)}{t} \approx 0
\]
gilt. Die induzierte Gegenspannung wird gemäß \cite{binder} durch Proportionalität zur Winkelgeschwindigkeit
\[
	U_i  = K_I \cdot \omega \ 
\]
modelliert.
Damit kann Gleichung \eqref{eq:dcMasche} zur Geraden
\begin{equation}
	I_{a \mrm{, max}}(\omega) = \frac{U_a}{R} - \frac{K_I}{R} \omega \ , \qquad \omega > \omega_{\mrm{EMK}} 
	\label{eq:Ivar}
\end{equation}
umgeformt werden.
Die Beschreibung des Bereichs negativer Winkelgeschwindigkeiten kann aus den Gleichungen \eqref{eq:Iconst} und \eqref{eq:Ivar} durch eine Punktspiegelung des in \figref{fig:Stromkennlinie} dargestellten Verlaufs in den dritten Quadranten gewonnen werden.

Zusammenfassend ergibt sich für den Verlauf des maximal verfügbaren Stroms  
\[
I_{a \mrm{, max}}(\omega) = \left\{\begin{array}{lll}
														-\frac{U_a}{R} - \frac{K_I}{R} \omega, & \omega < -\omega_{\mrm{EMK}}   \\ \\
														I_{w\mrm{, max}} \cdot \sign{\omega} 	& -\omega_{\mrm{EMK}} \leq \omega \leq \omega_{\mrm{EMK}} \\ \\				
														\frac{U_a}{R} - \frac{K_I}{R} \omega, & \omega > \omega_{\mrm{EMK}} \end{array}\right. . \
\] 



\subsection{Getriebe}

%\begin{figure}[h]
	%\centering
		%\includegraphics[width=0.5\textwidth]{Bilder/Motor/Getriebe.PNG}
	%\caption{Getriebe \cite{franke}}
	%\label{fig:Getriebe}
%\end{figure}

Die Rotorbewegung des Motors wird über das Getriebe und das Antriebszahnrad auf den Zahnriemen weitergegeben. Auf Grund der hohen Steifigkeit des mit eingebetteten Stahlseilen unterstützten Riemens wird dieser wie in Apprich \cite{apprich} als unendlich starr angenommen, sodass die Kinematik zwischen der Winkelgeschwindigkeit des Motors und der Schlittengeschwindigkeit ohne Federkopplung durch 
\[
	\xop = K_G \cdot r_{32} \cdot \omega
\]
modelliert werden kann.
 Hierbei wird die Getriebeübersetzung $K_G$ über das Zahnverhältnis 
\[
	K_G =  \frac{Z_{16}}{Z_{60}}
\]
berechnet, während der Antriebsradius $r_{32}$ als Abstand zwischen der neutralen Faser des Riemens und der Drehachse definiert ist.
Die Kraft 




\section{Modellparameter}\label{sec:mparams}

Bevor die in dieser Arbeit verwendeten Parametersätze für Motor- und \spd-Modell vorgestellt werden, soll im folgenden Abschnitt zunächst ein Überblick über die Entwicklung der in den vergangenen Arbeiten verwendeten Modellparameter gegeben werden.  


\subsection{Begründeter Stand der Modellparameter}\label{subsec:paramshist}

Die für die Modellierung erforderlichen Systemparameter des Versuchsstands wurden erstmalig 1997 von Franke \cite{franke} durch Messungen identifiziert. Die Antriebseinheit aus Spannungs-Strom-Wandler, Motor, Getriebe und Riemen ist seitdem nicht verändert worden. Daher repräsentieren die von Franke \cite{franke} identifizierten Modellparameter in Bezug auf die Antriebseinheit weiterhin den aktuellen Stand (siehe \secref{subsec:motorparams})
   
Während zuvor noch ein Einfachpendel verwendet worden war, konstruierte Apprich \cite{apprich} 2009 erstmalig ein Doppelpendel für den Versuchsstand. Von den Änderungen betroffen war neben den Pendeln auch die Schlittenmasse, da auch der obere Teil des Schlittens neu konstruiert wurde. Die Modellparameter für die neuen Pendelstäbe wurden, anders als bei Franke \cite{franke}, nicht gemessen, sondern aus dem CAD-Modell abgeleitet. Dies betrifft die Massen, Trägheitsmomente, Längen und Schwerpunkte der beiden Stäbe. Die Masse des Schlittens wurde von Franke \cite{franke} übernommen. Durch Messungen wurden lediglich die viskose und trockene Reibung des Schlittens gegenüber den Schienen erneut identifiziert. 

Im selben Jahr wurde von Kämmerer \cite{kämmerer} die viskose Lagerreibung $d_1$ zwischen Stab 1 und dem Schlitten als fehlender Modellparameter durch Messungen ergänzt. Die viskose Lagerreibung $d_2$ zwischen Stab 1 und Stab 2 wurde rechnerisch bestimmt, da das Lager gegen Ende der Arbeit getauscht werden musste. Die viskose Dämpfung des Schlittens, die von Apprich \cite{apprich} zuvor gemessen worden war, wurde durch einen deutlich höheren Schätzwert ersetzt. Außerdem wurde erstmalig die Masse von Schlitten und Antrieb zu einer schlittenseitig wirkenden Gesamtmasse zusammengefasst und ebenfalls als Schätzwert ausgewiesen.

Die Reibwerte $d_1$ und $d_2$ wurden 2011 durch Kisner \cite{kisner} erneut bestimmt. Durch die \textit{Prediction-Error Minimization Method} aus der \textit{System Identification Toolbox} von \Matlab\ wurden die Parameter $d_1$ und $d_2$ so variiert, dass die quadratische Fehlersumme minimal wird. Als Fehler ist hierbei die Differenz zwischen den gemessenen und den vom Modell vorhergesagten zeitlichen Winkelverläufen $\varphi_1(t)$ und $\varphi_2(t)$ bei ruhendem Schlitten und frei gewählter Anfangsauslenkung zu verstehen. Für die Optimierung wurden zudem nicht näher beschriebene Anfangsschätzwerte für $d_1$ und $d_2$ gewählt. Es wird davon ausgegangen, dass es sich um Erfahrungswerte handelt, da sie einerseits nicht mit den zuletzt von Kämmerer \cite{kämmerer} bestimmten Werten übereinstimmen, jedoch andererseits zu guten Ergebnissen am Versuchsstand führten. Statt der optimierten Werte wurden in den Nachfolgearbeiten die Anfangsschätzwerte weiterverwendet. Der Reibwert $d_0$ der viskosen Schlittenreibung, der nicht Gegenstand der Optimierung war, wurde mit einem deutlich höheren Wert als bei Kämmerer \cite{kämmerer} angegeben. Da keine explizite Begründung vorliegt, wird von einem Erfahrungswert ausgegangen. Die zuvor von Kämmerer \cite{kämmerer} geschätzte effektive Gesamtmasse von Schlitten und Antrieb wurde nach unten korrigiert, wobei die Hintergründe für diesen Schritt ebenfalls nicht bekannt sind. Der neue Wert ist jedoch plausibel und wird daher ebenfalls als erfahrungsbasierter Schätzwert verstanden. Der ist in den weiteren Arbeiten nicht mehr verändert worden, sodass er als aktueller Stand zu betrachten ist. Bei Kisner \cite{kisner} wurde erstmalig auch eine Begrenzung der Stellkraft von $F_{\mrm{max}}=\valunit{400}{N}$ bezüglich des Schlittens angegeben, die als gegeben dokumentiert ist. 

2011 wurde außerdem von Noupa \cite{noupa} sowohl die viskose als auch die trockene Schlittenreibung gemessen, wobei besonders für die viskose Reibung eine hohe Richtungsabhängigkeit beobachtet wurde. Die gemessenen Werte wurden in den weiteren Arbeiten jedoch nicht weiter beachtet.

Auf Grund eines Austauschs des Lagers von Stab 2 wurde 2014 von Brehl \cite{brehl} eine erneute Identifikation der viskosen Dämpfungskonstanten $d_2$ messungsbasiert durchgeführt. Dabei wurden auch Länge und Masse von Stab 2 gemessen, womit ebenfalls das Massenträgheitsmoment neu berechnet wurde. In den Nachfolgearbeiten wurden jedoch nur die Dämpfungskonstante weiterverwendet, während für Länge, Masse und Trägheitsmoment weiterhin die CAD-Werte von Apprich \cite{apprich} verwendet wurden.  

Chang \cite{chang} konstruierte im Sommersemester 2019 ein neues Doppelpendel, wobei Schlitten und Antrieb nicht verändert worden sind. Die neuen Modellparameter wurden wieder aus dem CAD-Modell abgeleitet. Die Angabe des Schwerpunkts $s_2$ von Stab 2 scheint in der Ausarbeitung jedoch zu fehlen. Die Dämpfungskonstanten $d_1$ und $d_2$ wurden von den Vorgängern übernommen, da bei der Konstruktion die gleichen Rillenkugellager gewählt wurden wie bei Apprich \cite{apprich}. Für $d_1$ wurde der Schätzwert von Kisner \cite{kisner} und für $d_2$ der Messwert von Brehl \cite{brehl} übernommen, wobei in der Ausarbeitung die Werte von Apprich \cite{apprich} genannt werden. Die viskose und die trockene Reibung des Schlittens wurden durch Messung selbst bestimmt. Wie bei Noupa \cite{noupa} wurde bei der viskosen Reibung eine auffällige Richtungsabhängigkeit festgestellt. Die Berücksichtigung der Richtungsabhängigkeit mit einer Vorsteuerung führte jedoch zu einer Verschlechterung des Systemverhaltens. Daher wurde schließlich die linksseitige Dämpfungskonstante für beide Seiten übernommen. Da das neu konstruierte Pendel noch nicht für die weiteren Bestandteile der Arbeit, wie die Auslegung der Regelung und deren Erprobung am Versuchsstand, zur Verfügung stand, wurde weiterhin auf die CAD-Werte von Apprich \cite{apprich} zurückgegriffen. Der Betrag der maximalen Stellkraft $F_{\mrm{max}}$ wurde zudem von dem von Kisner \cite{kisner} zuletzt genannten Wert von \valunit{400}{N} auf \valunit{421}{N} erhöht. Der neue Wert lässt sich rechnerisch nachvollziehen, wie \secref{subsec:motorparams} entnommen werden kann.

Im Rahmen einer Verifikation der von Chang \cite{chang} angegebenen Modellparameter für das neue Doppelpendel, wurden im Wintersemester 2019/2020 durch Ribeiro \cite{ribeiro} die Parameter aus dem CAD-Modell erneut abgeleitet. Da hierbei nicht genauer auf den Anlass der Neubestimmung eingegangen wurde, sich die neu bestimmten Parameter jedoch deutlich von den Vorherigen unterscheiden, wird angenommen, dass sich die von Chang \cite{chang} angegebenen Werte im Rahmen der Verifikation als fehlerbehaftet herausstellt hatten. 
Darüber hinaus wurde durch Messung die Reibung der Pendelgelenke identifiziert. Hierbei wurde neben der viskosen Reibung erstmalig auch die trockene Reibung in den Gelenken ermittelt. Die Parameter von Ribeiro \cite{ribeiro} zu den Pendelstäben sind somit aktueller Stand. Übernommen wurde weiterhin die von Kisner \cite{kisner} geschätzte Gesamtmasse mit Schlitten und Antrieb. Für die maximale Stellkraft wurde im Gegensatz zur Vorgängerarbeit Chang \cite{chang} statt \valunit{421}{N} wieder \valunit{400}{N} angenommen. Es wird vermutet, dass dadurch eine Stellkraftreserve für die Regelung vorgehalten werden sollte.


\subsection{Parameter des Motor-Modells}\label{subsec:motorparams}

Gemäß \secref{subsec:paramshist} werden die Modellparameter für das Motormodell Franke \cite{franke} entnommen. 

Die Steuerspannung $U_{\mrm{Steuer}}$ am Eingang des Spannungs-Strom-Wandlers darf maximal $\pm \valunit{10}{V}$ betragen. Ein Betrag größer als \valunit{10}{V} sollte vermieden werden, da sonst der Impulsstrom über den zulässigen Wert ansteigen und der Stromregler zerstört werden kann.

Die Verstärkung $K_{UI}$ des Spannungs-Strom-Wandlers kann durch ein Potentiometer bis zu einem Wert von etwa \valunit{2}{A/V} eingestellt werden, sodass mit der maximalen Steuerspannung der maximal zulässige Impulsstrom von \valunit{20}{A} erreicht wird. Die zuletzt dokumentierte Einstellung liegt bei
\[
	K_{UI} = \valunit{1,87}{\frac{A}{V}} \ .
\]
Der Wert beinhaltet bereits eine Idealisierung, da Messungen von Franke \cite{franke} gezeigt haben, dass die reale Verstärkung am Versuchsstand eine leichte Richtungsabhängigkeit bezüglich des Vorzeichens aufweist.

Die Zeitkonstante $T_{UI}$ wird als ausreichend klein angesehen, sodass die Dynamik des Wandlers vernachlässigt werden kann.

Alle weiteren verwendeten Parameter des Motormodells sind in \tabref{tab:motorparams} aufgeführt.

\begin{table}[htbp]
	\centering
	\caption{Parameter -- Motorsystem}
		\begin{tabular}{lcc|l}
			\toprule
			Bezeichnung	&	Symbol	&	Einheit	&	Franke97	\\
			\midrule
			Maximale Steuerspannung & $U_{\mrm{Steuer, max}}$ & \unit{V} & 10 \\ 
			Wandlerverstärkung & $K_{UI}$ & \unit{\frac{V}{A}} & 1,87 \\
			Wandlerzeitkonstante & $T_{UI}$ & \unit{s} & 0,00075 \\
			Maximale Ankerspannung & $U_{a \mrm{, max}}$ & \unit{V} & 65 \\
			Ankerwiderstand & $R$ & \unit{\Omega} & 0,9 \\
			Drehmomentkonstante & $K_I$ & \unit{\frac{N m}{A}} & 0,153 \\
			Getriebeübersetzung & $K_G$ & - & $\frac{16}{60}$ \\
			Antriebsradius & $r_{32}$ & \unit{m} & 0,0255 \\			
			\bottomrule
		\end{tabular}
	\label{tab:motorparams}
\end{table}


Mit der statischen Verstärkung zwischen Eingang $U_{\mrm{Steuer}}$ und Ausgang $F$
\begin{align}
	K_{\mrm{MotorGain}} = K_{UI} \cdot K_I \cdot \frac{1}{K_G} \cdot \frac{1}{r_{32}} = \valunit{42,075}{\frac{N}{V}}
	\label{eq:motgain}
\end{align}
lässt sich die maximale Stellkraft
\begin{align}
	F_{\mrm{max}} = U_{\mrm{Steuer, max}} \cdot K_{\mrm{MotorGain}} = \valunit{420,75}{N}
\end{align}
berechnen, wobei $K_{UI} = \valunit{1,87}{A/V}$ ist.
Für $K_{UI \mrm{, max}} = \valunit{2}{A/V}$ läge die maximale Stellkraft bei $F_{\mrm{max}} = \valunit{450}{N}$.

Mit $K_{UI} = \valunit{1,87}{A/V}$ kann zudem
\[
	I_{w\mrm{, max}} = U_{\mrm{Steuer, max}} \cdot K_{UI} = \valunit{18,7}{A} 
\]
berechnet werden, sodass sich mit Hilfe von \equref{eq:wemk} für die Drehzahl, ab der die Gegeninduktion als Strombegrenzung wirksam wird,
\[
	\omega_{\mrm{EMK}} = \valunit{314,84}{\frac{rad}{s}}
\]

ergibt. Dies entspricht nach \equref{eq:kinematik} einer Schlittengeschwindigkeit von $\xop = \valunit{2,14}{\frac{m}{s}}$ 



\subsection{Parameter des Schlittendoppelpendels}\label{subsec:spdparams}

Ausgehend von \secref{subsec:paramshist} werden in dieser Arbeit drei Parametersätze angelegt \siehe{\tabref{tab:spdparams}}.
Dies ermöglicht einen Vergleich der unterschiedlichen Systeme hinsichtlich des Systemverhaltens und der Stabilisierbarkeit.

Aufgrund der Neukonstruktion des \dpd s und dabei aufgetretener Schwierigkeiten in der Regelung ist ein Vergleich zum vorigen System von Interesse.
Die Parameter der alten Konstruktion werden mit \qq{Apprich09} bezeichnet, obwohl hier auch neu bestimmte Werte von Kisner \cite{kisner} und Brehl \cite{brehl} (Reibung und Schlittenträgheit) enthalten sind.
Der zweite Parametersatz \qq{Chang19} stellt die Daten der Ausarbeitung von \cite{chang} dar, in welcher die Neukonstruktion stattfand. 
Da diese erst in der nächsten Arbeit (Ribeiro \cite{ribeiro}) in Betrieb genommen wurde und dort sowohl die CAD-Parameter erneut bestimmt als auch die Reibungsparameter identifiziert wurden, werden die Werte \qq{Ribeiro20} als korrekte Parameter des neukonstruierten \dpd s betrachtet.

\begin{table}[htbp]
	\centering
	\caption{Parameter -- \spds}
		\begin{tabular}{lcc|lll}
			\toprule
			Bezeichnung	&	Symbol	&	Einheit	&	Apprich09	&	Chang19	&	Ribeiro20	\\
			\midrule
			Masse des Schlittens	&	$m_0$	&	\unit{kg}	&	16,5	&	16,5	&	16,5	\\
			Masse des ersten Pendels	&	$m_1$	&	\unit{kg}	& 0,615	&	0,329	&	0,8534	\\
			Masse des zweiten Pendels	&	$m_2$	&	\unit{kg}	&	0,347	&	0,3075	&	0,3957	\\
			Trägheitsmoment des ersten Pendels	&	$J_1$	&	\unit{kg \,m^2}	&	0,00647 &	0,01457	&	0,01128	\\
			Trägheitsmoment des zweiten Pendels	&	$J_2$	&	\unit{kg \,m^2}	&	0,00407	&	0,00334	&	0,003343	\\
			Länge des ersten Pendels	&	$l_1$	&	\unit{m}	&	0,2905	&	0,325	&	0,282	\\
			Länge des zweiten Pendels	&	$l_2$	&	\unit{m}	&	0,3388	&	0,305	&	0,280	\\
			Schwerpunkt des ersten Pendels	&	$s_1$	&	\unit{m}	&	0,0775	&	0,1425	&	0,09373	\\
			Schwerpunkt des zweiten Pendels	&	$s_2$	&	\unit{m}	&	0,146	&	0,114254	&	0,114254	\\
			Viskose Dämpfung des Schlittens	&	$d_0$	&	\unit{\frac{N s}{m}}	&	17	&	17,6	&	17	\\
			Viskose Dämpfung des ersten Pendels	&	$d_1$	&	\unit{\frac{N m s}{rad}}	&	0,0091	&	0,005	&	0,00768	\\
			Viskose Dämpfung des zweiten Pendels	&	$d_2$	&	\unit{\frac{N m s}{rad}}	&	0,0006905	&	0,005	&	0,000285	\\
			\crb\ des Schlittens	&	\Fco	&	\unit{N}	&	16,232	&	17,5	&	13,43	\\
			\crb\ des ersten Pendels	& \Mceo	&	\unit{Nm}	&	0	&	0,0538	&	0,0538	\\
			\crb\ des zweiten Pendels	& \Mczo	&	\unit{Nm}	&	0	&	0,0000912	&	0,0000912	\\
			Erdbeschleunigung	&	$g$	&	\unit{\frac{m}{s^2}}	&	9,81	&	9,81	&	9,81	\\
			\bottomrule
		\end{tabular}
	\label{tab:spdparams}
\end{table}

Es wird somit im Folgenden zwischen den \qq{alten Parametern} (\texttt{Apprich09}) und den \qq{neuen Parametern} (\texttt{Ribeiro20}) unterschieden.
Der größte Unterschied zwischen beiden Systemen ist die \crb\ in den Pendelgelenken, die zusätzlich zur viskosen Reibung erstmals bei Ribeiro \cite{ribeiro} bestimmt wurde.
Da bei der Neukonstruktion die Messsignalübertragung zur Vermeidung einer Kabelaufwickelung über einen Schleifring realisiert wurde, besteht die Vermutung, dass dieser für die erhöhte \crb\ verantwortlich ist. 
Dies macht das System bereits um einen sehr kleinen Arbeitsbereich nichtlinear, was die Regelung erschwert.

Für die Skalierungsparameter \xopth, \pheth\ und \phzth\ der \crb\ \siehe{\secref{sec:crb}} wird jeweils 0,01 angenommen.



\section{Aufbau des Simulationsmodells}

In dieser Arbeit wird das \dpds\ rein simulativ betrachtet.
Das Simulationsmodell wird im Vergleich zu Vorgängerarbeiten wesentlich systematischer und detaillierter aufgebaut.
Dadurch sind umfangreiche und automatisierte Tests möglich, mit denen das System umfassend untersucht werden kann.

Besonders wird in dieser Arbeit Wert auf einen strukturierten, modularen Aufbau gelegt.
Dies soll flexible Änderungen am System, wie \zB Systemparameter oder Reglerparameter ermöglichen.
Von der Berechnung der \bwgl\ über die Parametrisierung des Systems bis zur Reglerberechnung kann das Modell vollständig und konsistent initialisiert werden.
Außerdem soll dadurch die Nutzbarkeit und Wiederverwendbarkeit in zukünftigen Arbeiten gewährleistet sein.

Das Simulationsmodell voriger Arbeiten wie \cite{chang} ist nur wenig strukturiert aufgebaut, enthält Redundanzen und viele fest-kodierte Parameter.
Es wird daher lediglich zur Orientierung verwendet.
Hart-kodierte Parameter werden in dieser Arbeit vermieden, sodass jeder Parameter bei der \init\ geändert werden kann (ohne das \sm-Modell ändern zu müssen).
Auch Gleichungen können variabel ausgelegt werden, was durch die symbolische Schreibweise von \ml\ ermöglicht wird.
Code-Dopplungen werden vermieden. 
Die gesamte \init\ ist stark funktionalisiert, da dies für die Variationstests notwendig ist.
Dadurch wird auch sichergestellt, dass Änderungen nur an einer Stelle vorgenommen werden müssen und automatisch direkt an allen entsprechenden Stellen angepasst werden.

Die \sm-Modelle und \ml-Funktionen zur Modellierung des Systems und \init\ des Simulationsmodells befinden sich im Ordner \texttt{Modell}. 


\subsection{Aufbau in \Simulink}

\subsubsection{Subsysteme}
In \sm\ lassen sich sogenannte \emph{Subsysteme} erstellen, um ein Simulationsmodell übersichtlicher zu gestalten.
Diese Subsysteme (im Folgenden auch Module genannt) lassen sich zudem als Datei externalisieren (\emph{Referenced Subsystem}), wodurch sie einerseits separat bearbeitet werden können und andererseits an mehreren Stellen modular wiederverwendet werden können.
Somit wird lediglich in der obersten Schicht (die Testebene) ein \sm-\emph{Model} verwendet.

In dieser Arbeit werden immer Module erstellt, sofern es sinnvoll erscheint.
Dadurch wird das Gesamtsystem übersichtlich gehalten und kann sehr flexibel modifiziert werden.
Die einzelnen Module (wie \zB Motor, Gesamtmodell, \zsr) stellen zudem wiederverwendbare Teile dar und können so an verschiedenen Stellen im Projekt eingebaut werden, aber möglicherweise auch am \sm-Modell des Versuchsstandes zum Einsatz kommen.
So könnte relativ einfach derselbe Regler an dem echten und dem simulierten System eingesetzt und verglichen werden.

\subsubsection{Maske}
Bei Subsystemen kann eine sogenannte \emph{Maske} eingerichtet werden, wodurch Parameter übergeben werden können.
In deren Abhängigkeit können in einem Subsystem auch initialisierende Berechnungen durchgeführt werden.

Grundsätzlich können in \sm\ alle Variablen aus dem globalen \ml-Workspace verwendet werden, was allerdings fehleranfällig ist.
Geschickter ist es, dem Simulationsmodell eine einzige Variable vom Typ \texttt{struct} zu übergeben, in der alle Parameter vorhanden sind.
So werden jedem Modul über die jeweilige Maske nur genau die Parameter übergeben, von denen es abhängt.
Dadurch bleibt der Gesamtaufbau modularer und strukturierter.

\subsubsection{Scopes und ToWorkspace}
Um die Signalverläufe direkt zu untersuchen, werden an allen wichtigen Stellen in den Simulationsmodellen \emph{Scopes} installiert.
So lässt sich das Systemverhalten oder Probleme in der Regelung schnell und einfach analysieren.
Oft werden Scopes mit mehreren Eingängen verwendet, um Signale besser vergleichen zu können.

Für alle Variablen, die später in \ml\ für weitere Auswertungen zur Verfügung stehen sollen, werden \emph{ToWorkspace}-Blöcke verwendet.
Diese geben die Daten an die Ausgabevariable \texttt{out}.

Für die Screenshots werden diese Blöcke meist aus Übersichtsgründen entfernt.


\subsection{\Simulink\ Module und Modelle}


\subsubsection{Motor.slx}
\begin{figure}[h]
	\centering
		\includegraphics[scale=0.4]{Bilder/Simulink/motor.PNG}
	\caption{Motormodell in \sm}
	\label{fig:simmot}
\end{figure}
Das Motormodell wird \secref{sec:mot} entsprechend als Modul implementiert \siehe{\figref{fig:simmot}}.
Eingangsgröße ist die Steuerspannung sowie die Schlittengeschwindigkeit, da sich daraus die Induktionsspannung ergibt.
Ausgangsgröße ist die Kraft, die am Schlitten wirkt.
Der V/A Wandler wird als Subsystem implementiert und gibt den Motorstrom aus \siehe{\figref{fig:simva}}.
Die Berechnung der Strombegrenzung \siehe{\secref{subsec:dcMotor}} wird als \emph{matlabFunction} realisiert.
Dabei werden die Signale \texttt{satState\_w} (falls die Sättigungsdrehzahl erreicht ist) und \texttt{satState\_I} (falls der Sollstrom nicht erreicht werden kann) gesetzt und können später analysiert werden.
Die Tiefpass-Charakteristik der Induktivität wird mit dem \mrm{PT_1}-Glied \eqref{eq:UI} dargestellt.


\begin{figure}
	\centering
		\includegraphics[scale=0.4]{Bilder/Simulink/va_wandler.PNG}
	\caption{V/A Wandler in \sm}
	\label{fig:simva}
\end{figure}

\subsubsection{SchlittenPendel.slx}
Das \zrm\ des \spds s wird mit einer \emph{S-Function} dargestellt.
Die zugehörige \ml-Datei \texttt{SchlittenPendelFunc.m} wird automatisch bei der \init\ erstellt \siehe{\secref{subsec:init}}.
Lediglich die Anfangswerte werden über die Simulationsparameter übergeben.

\subsubsection{Gesamtmodell.slx}\label{sec:gesamtslx}
Dieses Modul stellt das gesamte Modell dar und fasst die beiden vorigen Module zusammen \siehe{\figref{fig:simges}}.
Die Eingangsspannung wird an den Motor gegeben und dessen Ausgang $F$ ist die Schnittstelle zum \sdpd.
Die Schlittengeschwindigkeit wird zum Motor zurückgeführt, da durch diese die Motordrehzahl bestimmt wird.

\begin{figure}[h]
	\centering
		\includegraphics[scale=0.4]{Bilder/Simulink/gesamtmodell.PNG}
	\caption{Gesamtmodell in \sm}
	\label{fig:simges}
\end{figure}

\subsubsection{Gesamtmodell\_Test.slx}
Dies ist ein \sm-Modell und bindet das Gesamtmodell-Modul ein.
Auf dieses können testweise verschiedene Eingangsverläufe gegeben werden, um das Systemverhalten auf Plausibilität zu überprüfen, beispielsweise:
\begin{itemize}
	\item Keine Spannung, aber Anregung durch Anfangsauslenkung
	\item Konstante Spannung
	\item Sinusförmige Spannung
\end{itemize}
Da die durchgeführten Tests am Modell plausible Ergebnisse zeigen, wird im Folgenden von der Korrektheit des Simulationsmodells ausgegangen.


\subsection{Parameter}

Größere zusammenhängende Parameterdaten werden als Datei \bzw Funktion ausgelagert, um eine einfache Austauschbarkeit zu erreichen.
Die drei Parametersätze des \spd s \siehe{\tabref{tab:spdparams}} sowie die Motorparameter befinden sich im Ordner \path{Modellparameter}.


\subsection{Initialisierung}\label{subsec:init}

Zum Initialisieren des Modells und der Simulation führt man das Skript \texttt{Init.m} aus.
Dadurch werden folgende 4 Schritte durchgeführt:
\begin{enumerate}
	\item \textbf{InitSymEq}: Berechnet symbolisch die \bwgl\ nach \secref{subsec:bwgl} für das \krs\ (\path{SchlittenPendelSymF}) und das \bss\ (\path{SchlittenPendelSymA}) und speichert sie als globale Variablen. Auch die Ruhelagen werden initialisiert (definiert in\break \path{SchlittenPendelRuhelagen}).
	\item \textbf{InitParams}: Initialisiert die Parameter des Motors und des Schlittens (standardmäßig \texttt{Ribeiro20}) als globale Variablen. Hier können die Parameter ausgetauscht oder verändert werden (\zB \crb\ abschalten).
	\item \textbf{InitSystem(SchlittenPendelParams)}: Diese Funktion erstellt die \zrm e des\break \spds s nach \secref{subsec:zrm}. Dafür wird die Funktion \mbox{\path{SchlittenPendelNLZSR.m}} genutzt, welche das \zrm\ in Abhängigkeit der \bwgl\ und der Parameter erstellt. Aus dem \krs\ wird mittels \path{sys2sfct} eine \emph{S-Function} generiert und in \path{SchlittenPendelFunc.m} gespeichert.
	\item \textbf{InitSim}: Bereitet die globale Variable \path{simparams} vor, in welcher alle Parameter für das \sm-Modell übergeben werden. Dazu gehören beim Modell die Motorparameter und die Startwerte des \spds s.
\end{enumerate}
Nach der \init\ kann \path{Gesamtmodell_Test.slx} ausgeführt werden.


\subsection{Auswertung}\label{subsec:ausw}

Nach einer Simulation ist es nützlich, die wesentlichen Ergebnisse direkt ablesen zu können (\zB ob der Schlitten außerhalb der Begrenzung war oder ob der Motor in Sättigung war).
Außerdem wird eine Plot-Funktion implementiert, die die wichtigsten Verläufe darstellt.
Um das Verhalten des \dpd s besser interpretieren zu können, wird zudem eine Animationsfunktion erstellt, die direkt das Pendelverhalten visualisiert.

Die entsprechenden \ml-Funktionen sind im Ordner \path{Auswertung} vorhanden.
Weitere Informationen und Kommentare finden sich auch meist in den \ml-Funktionen selbst.

\subsubsection{plot\_outputs}
Erstellt ein Diagramm, dass die drei Verläufe der Ausgangsgrößen (inklusive deren Schätzwerte) und die Stellgröße (Soll- und Ist-Wert) darstellt.
Der Funktion können einige Parameter übergeben werden, sodass das Diagramm auch abgespeichert werden kann.
Für weitere Informationen sei auf die Dokumentation in der Datei \path{plot_outputs.m} verwiesen.

\subsubsection{animate\_outputs}
Stellt die Bewegungen des Schlittens und des \dpd s zeitlich dar.
Parameter für die Bildwiederholrate (FPS) und ein Zeitfaktor können angegeben werden.
Außerdem kann das Video gespeichert werden.
Für mehr Informationen siehe \path{animate_outputs.m}.

\subsubsection{plotanimate}
Führt die beiden obigen Funktionen aus, sodass Ergebnisse mit einem Funktionsaufruf gemeinsam gespeichert werden können.
Beispiel:
	\[
	\texttt{plotanimate(out, 'u=0,AP4,phi2=0.01', 'Modell Tests', 60, 1/4 )}
\]


%\subsection{Weitere \Matlab-Funktionen}



\chapter{Arbeitspunkt-Regelung}\label{cha:apr}

In diesem Kapitel wird die Modellierung des Gesamtsystems erläutert und auf dessen Implementierung in Simulink eingegangen.

\section{Einleitung}%\label{sec:}



\section{Aufbau in \Simulink}



\section{Implementierung in \Matlab}



\section{Anfangswert-Tests}



\section{QR Parameter Tests}



\section{System Parameter Tests}







%\input{./inc/Einfuehrung.tex}
%\input{./inc/Tipps.tex}
%\chapter{Allgemeine Hinweise und Informationen zum Erstellen einer schriftlichen Arbeit mit \LaTeX}
\label{cha:Hinweise_Latex}

%Dieses Kapitel gibt allgemeine Hinweise zur Erstellung einer wissenschaftlichen Arbeit mit dem Textsatzprogramm \LaTeX. Es ist inhaltlich identisch mit dem vorherigen Kapitel und enthält zusätzlich die \LaTeX-spezifischen Befehle, die bei der Umsetzung der Hinweise hilfreich sein könnten.

%Wird die Arbeit \emph{nicht} mit \LaTeX\ erstellt, sollte das vorherige Kapitel gelesen werden.

%Besonderes Augenmerk sollte auf Abschnitt~\ref{sec:Latex-Zitieren}, \textbf{Zitate und korrekte Zitierweisen} gelegt werden.

Dieses Kapitel gibt allgemeine Hinweise zur Erstellung einer wissenschaftlichen Arbeit mit dem Textsatzprogramm \LaTeX. Das Kapitel sollte auch von Studierenden gelesen werden, die sich gegen eine Erstellung der Arbeit mit \LaTeX\ entschieden haben. Da das Layout der Arbeit in diesem Fall gemäß den TUD Designvorgaben zusätzlich selbst erstellt werden muss, dient die vorliegende Anleitung auch als Vorlage.

%Wird die Arbeit \emph{nicht} mit \LaTeX\ erstellt, sollte das vorherige Kapitel gelesen werden.

%Besonderes Augenmerk sollte auf Abschnitt~\ref{sec:Latex-Zitieren}, \textbf{Zitate und korrekte Zitierweisen} gelegt werden.

\section{\LaTeX-Distribution}
\label{sec:Distribution}
Welche Distribution verwendet wird, hängt vom persönlichen Geschmack des Anwenders ab und es besteht grundsätzlich die Wahl zwischen \Miktex{}, das vor allem unter Windows weit verbreitet ist und \texlive{}, das eher unter Linux und Mac Anwendung findet.
Im Folgenden wird nur das Vorgehen für \Miktex{} unter Windows beschrieben, für \texlive{} sind die Abläufe ähnlich.

Zuerst ist die Distribution herunterzuladen und zu installieren, wobei im Allgemeinen die Standardeinstellungen ausreichend sind.


\section{Editor}
Für \LaTeX{} gibt es eine Vielzahl von freien und kommerziellen Texteditoren. Das \Texstudio ist ein freier open-source Editor, der sich bei uns am Institut bewährt hat.

\subsection{Einrichten von \Texstudio}
Es gibt prinzipiell zwei verschiedene Wege, um aus der \LaTeX-Quelldatei ein pdf zu erzeugen.
\begin{itemize}
	\item \LaTeX => PDF pdf\LaTeX{} wandelt die Quelldatei direkt in ein pdf.
	Dabei können als Bilder im Format .jpg, .png und .pdf eingebunden werden.
\end{itemize}
In \Texstudio{} kann die Werkzeugkette im Reiter \zitat{Erzeugen} in den Einstellungen angepasst werden.
Um das direkte Kompilieren mit pdf\LaTeX{} zu verwenden ist bei Kompiler \texttt{txs:///pdflatex} einzutragen, während für den Weg über Postscript \texttt{txs:///latex} einzutragen ist.
\Texstudio{} verwendet eine Hauptdatei für die Kompilierung, die entweder automatisch ermittelt wird, oder vom Benutzer festgelegt wird.

\subsubsection{Autovervollständigung}
Damit die Makros in den Paketen, die mit der Vorlage mitgeliefert werden, von \Texstudio{} beim Eintippen automatisch ergänzt werden, müssen die in den Ordnern der jeweiligen Pakete befindlichen cwl-Dateien in den Ordner \verb|%APPDATA%/TeXstudio/completion/user| kopiert werden und \Texstudio{} neu gestartet werden.
Durch Anlegen einer eigenen cwl-Datei können dort auch eigene Autovervollstädigungen definiert werden, wie in der Dokumentation unter \url{http://texstudio.sourceforge.net/manual/current/usermanual_en.html#CWLDESCRIPTION} nachzulesen ist.
Sollen die Vervollständigungen dauerhaft aktiviert werden und nicht nur, wenn \Texstudio{} das Laden der entsprechenden Pakete erkannt hat, ist unter \Texstudio{} konfigurieren->Vervollständigung ein Haken bei der gewünschten cwl-Datei zu setzen.

\subsubsection{Makros}
Da das Öffnen von Dateien, die mit dem Makro \texttt{\textbackslash{}inputtikz} geladen werden, in \Texstudio{} nicht mehr über die eingebaute Funktionalität möglich ist, kann das das Javascriptmakro \texttt{open\_includetikz.js} über \zitat{Makros->Makros bearbeiten} durch Kopieren in das sich öffnende Fenster hinzugefügt werden und ein beliebiges Tastenkürzel mit diesem assoziiert werden, damit sich die eingebundenen Bilddateien bei Eingabe des Tastenkürzels öffnen, wenn der Curser in einer Zeile mit dem Makronamen steht.



\section{Rechtschreibung}

Studentische Abschlussarbeiten am IAT sind nach den aktuell geltenden Regeln der deutschen Rechtschreibung zu verfassen, \cite{Duden}.

Die neue deutsche Rechtschreibung wird in \LaTeX\ mit dem Paket \verb|babel| über die Option \verb|ngerman| aktiviert.

Im Internet findet man unter \url{http://www.duden.de/} einen Crashkurs zur neuen deutschen Rechtschreibung.

Eine Überprüfung der Rechtschreibung über die in die Editoren eingebaute Rechtschreibprüfung hinaus, lässt sich in \Texstudio{} mit dem Languagetool erreichen.
Dieses kann unter \Texstudio{} im Reiter \zitat{Sprache prüfen} konfiguriert werden.
Dort ist unter \zitat{LT-Pfad} der Pfad zur jar-Datei \texttt{languagetool.jar} anzugeben und unter \zitat{LT-Argumente} \zitat{\texttt{org.languagetool.server.HTTPServer -p 8081}} und bei \zitat{Server-Adresse} \zitat{\texttt{http://local\-host:8081}} einzugeben.
Zum Einschalten der Überprüfung kann dann vor dem Starten von \Texstudio{} das Languagetool von Hand gestartet werden, oder über \zitat{Starte das Languagetool falls es nicht läuft} automatisch gestartet werden.


\section{Zitate und korrekte Zitierweise}
\label{sec:Latex-Zitieren}
Zitate sind wörtliche oder sinngemäße Wiedergaben von Gedanken, Ideen oder Meinungen anderer Autoren.
Werden Ideen oder Inhalte aus Quellen wörtlich oder sinngemäß in die eigene Arbeit übernommen, besteht die \textbf{Pflicht}, diese zu kennzeichnen.
Wird dies unterlassen, liegt ein \textbf{Täuschungsversuch} (Plagiat) vor und die Arbeit kann als \textbf{nicht bestanden} bewertet werden.

Laut § 38 Abs. 2 der \emph{Allgemeinen Prüfungsbestimmungen} liegt \glqq\ [...] ein Täuschungsversuch [...] vor, wenn eine falsche Erklärung nach §§ 22 Abs. 7, 23 Abs. 7 abgegeben worden ist oder ein anderes Werk, eine Bearbeitung eines anderen Werkes, eine Umgestaltung eines anderen Werkes ganz oder teilweise in der Prüfungsarbeit wiedergeben werden, ohne dieses zu zitieren (Plagiat).\grqq~\cite{APB}

\subsection*{Zitierfähigkeit}
Zitierfähig sind nur veröffentlichte Werke aus allgemein zugänglichen Quellen (Bücher, Artikel, \etc).
Quellen, bei denen die Verfügbarkeit nicht garantiert werden kann (Internetquellen) oder der Urheber nicht klar nachvollziehbar ist (Wikipedia, o\,Ä.), sind problematisch und sollten möglichst vermieden werden.
Wird doch solch eine Quelle zitiert, ist die aktuelle Version zum Zeitpunkt des Zitates beizulegen.

\subsection*{Wörtliche Zitate}
Wörtliche Zitate in ingenieurwissenschaftlichen Arbeiten sind unüblich.
Sollte doch ein wörtliches Zitat in die Arbeit übernommen werden, muss dieses buchstaben- und zeichengetreu, inklusive eventueller Rechtschreibfehler übernommen werden.
Das wörtliche Zitat wird in Anführungszeichen eingefasst.

\subsection*{Sinngemäße Zitate}
Weit häufiger werden in wissenschaftlichen Arbeiten Ideen oder Meinungen anderer Autoren sinngemäß übernommen.
Diese müssen durch einen Verweis auf die Quelle gekennzeichnet werden.
Durch die Position des Verweises muss der Umfang der sinngemäßen Übernahme klar hervorgehen.

\subsection*{Quellenangaben im Literaturverzeichnis}
Für die Darstellung der Verweise, als auch für die Darstellung der Quellen im Literaturverzeichnis gibt es verschiedene Zitierweisen.
Üblich sind die Harvard-Variante mit Autor und Veröffentlichungsjahr in runden Klammern, wie zum Beispiel (Isermann, 2001) oder eine fortlaufende Nummerierung in eckigen Klammern, wie in dieser Vorlage.
Das \texttt{biblatex} Paket ermöglicht verschiedene (im Deutschen übliche) Zitierweisen.
Die Sortierung des Literaturverzeichnisses kann alphabetisch (voreingestellt, oder \bspw durch die Paketoption \texttt{sorting=nyt}) oder nach dem Erscheinen der Verweise erfolgen (durch die Paketoption \texttt{sorting=none}).
Das Paket \texttt{iatsada} definiert einen Zitierstil basierend auf dem \texttt{numeric} Stil von \texttt{biblatex}, kann aber bei Bedarf angepasst werden, falls \bspw ein alphabetischer Stil (\texttt{alphabetic}) vorgezogen wird.

Ein Verweis wird mit \texttt{\textbackslash{}cite\{$\left\langle\text{label}\right\rangle$\}} eingefügt und mit einem festen Leerzeichen \verb|~| mit dem vorherigen Wort getrennt.
Schließt der Verweis einen Satz ab, folgt der Punkt \emph{hinter} dem Verweis.
Neben dem \texttt{\textbackslash{}cite} Befehl stehen im Paket \texttt{iatsada} weitere Makros zur Zitierung von Seitenzahlen wie \texttt{\textbackslash{}citep\{$\left\langle\text{label}\right\rangle$\}\{$\left\langle\text{page}\right\rangle$\}} oder \texttt{\textbackslash{}citerange\{$\left\langle\text{label}\right\rangle$\}\{$\left\langle\text{pageu}\right\rangle$\}\{$\left\langle\text{pageo}\right\rangle$\}} zur Verfügung, die vorgezogen werden sollte, da der Leser so, vor allem bei umfangreicheren Werken, schneller den zitierten Sachverhalt nachlesen kann.

Die Literaturliste wird \LaTeX{} über eine Datei im \textsc{Bib}\TeX{}-Format mit der Endung \texttt{bib} bekannt gemacht.
Für die Erstellung des Literaturverzeichnisses (Datei mit der Endung \texttt{bbl}) aus der Literaturliste bietet sich die Erweiterung \textsc{Biber} an, die mit der Paketoption \texttt{backend=biber}\footnote{\label{ftn:Fussnote1}Alternativ kann auch \texttt{backend=bibtex} verwendet werden, wobei dann die Werkzeugkette des verwendeten Editors entsprechend eingestellt werden muss.} in \texttt{biblatex} geladen werden kann und im Editor in die entsprechende Werkzeugkette integriert werden muss, damit \textsc{Biber} ausgeführt wird.
Alternativ kann das Literaturverzeichnis auch per Hand erstellt werden.
Viele Literaturverwaltungsprogramme, wie zum Beispiel \textsc{JabRef}, ermöglichen den direkten Export der Datenbank in das \textsc{Bib}\TeX{}-Format.


\section{Gliederung des Dokuments}
\label{sec:Latex-Gliederung}
Im Inhaltsverzeichnis wird die Gliederung der Arbeit dargestellt. Die Überschriften und Seitenangaben der Kapitel, Unterkapitel und Abschnitte müssen mit den Elementen im Inhaltsverzeichnis übereinstimmen. Überschriften sind kurz und prägnant zu formulieren und dürfen keine vollständigen Sätze sein. Gibt es Unterpunkte in der Gliederung, so müssen immer mindestens zwei davon existieren und inhaltlich auf der gleichen Ebene sein. Die einzelnen Punkte des Inhaltsverzeichnis müssen nummeriert werden. Die Übersichtlichkeit des Inhaltsverzeichnisses kann durch Einrücken der Unterpunkte erhöht werden.

Das Inhaltsverzeichnis wird in \LaTeX\ automatisch erstellt.
Innerhalb der einzelnen Kapitel \verb|\chapter{...}| werden weitere Unterteilungen mit den Befehlen \verb|\section{...}|, \verb|\subsection{...}| \usw vorgenommen.
Werden diese mit einem \verb|*| versehen, dann erhält der jeweilige Abschnitt keine Nummer und erscheint nicht im Inhaltsverzeichnis.
Dies kann in manchen Fällen nützlich sein.

Um eine korrekte Darstellung des Inhaltsverzeichnisses zu erhalten, muss \ggf mehrmals hintereinander kompiliert werden, da sich aufgrund von
Gleitobjekten Seitenzahlen ändern können.
Dreimaliges Kompilieren reicht in der Regel.

Bei Bildern und Tabellen, die in eine \verb|figure|- bzw.\ \verb|table|-Umgebung eingeschlossen sind, handelt es sich um sog.\ \emph{Gleitobjekte}, \dah sie erscheinen nicht an der Stelle, an der sie eingebunden werden, sondern oben oder unten auf einer Seite, siehe \zB \tabvref{tab:Sonderzeichen}.
Optional können in \verb|[]| noch Positionierungswünsche angegeben werden.
Mit dem Befehl \verb|\caption{...}| erhalten Bilder eine \emph{Unterschrift} und Tabellen eine \emph{Überschrift}.

Kapitel, Abschnitte, Bilder, Tabellen und Gleichungen können mittels \verb|\label{...}| benannt werden.
Dadurch ist es möglich, sie später mit \verb|\ref{...}| oder \verb|\pageref{...}| zu referenzieren.
Es empfiehlt sich, den Namen (Labels) eine Markierung voranzustellen, aus der hervorgeht, um welches Objekt es sich handelt.
Üblich sind \verb|cha:| für \glqq Chapter\grqq, \verb|sec:| für \glqq Section\grqq, \verb|fig:| für \glqq Figure\grqq, \verb|tab:| für \glqq Table\grqq\ und \verb|eq:| für \glqq Equation\grqq.
Ein Bild benennt man also \zB\ mit \verb|\label{fig:Ausgangssignal}|.
Für die verschiedenen Arten von Objekten bietet das Paket \texttt{iatsada} die in \tabref{tab:Referenzen} angegebenen speziellen Referenzierungsmakros.
\begin{table}
	\centering
	\caption{Labels und zugehörige Referenzen}
	\label{tab:Referenzen}  
	\begin{tabular}{llll}
		Typ			&	Prefix				&	Referenz			&	resultierender Text\\
    \midrule
		Chapter		&	\texttt{cha}		&	\verb|\charef|		&	\charef{cha:Hinweise_Latex}\\
		Section		&	\texttt{sec}		&	\verb|\secref|		&	\secref{sec:Latex-Gliederung}\\
		Anhang		&	\texttt{cha/sec}	&	\verb|\appref|	&	\appref{cha:Checkliste}\\
		Table		&	\texttt{tab}		&	\verb|\tabref|		&	\tabref{tab:Referenzen}\\
		Figure		&	\texttt{fig}		&	\verb|\figref|		&	\figref{fig:Standardregelkreis}\\
		Equation	&	\texttt{eq/equ}		&	\verb|\equref|		&	\equref{eq:Allgemein-Approx}\\
		Listing		&	\texttt{lst}		&	\verb|\lstref|		&	\lstref{lst:Listing1}\\
		Algorithm	&	\texttt{alg}		&	\verb|\algoref|		&	\algoref{lst:Listing1}\\
		Footnote	&	\texttt{ftn}		&	\verb|\ftnref|		&	\ftnref{ftn:Fussnote1}\\
    \midrule
		Table		&	\texttt{tab}		&	\verb|\tabvref|		&	\tabvref{tab:Systemparameter}\\
		Figure		&	\texttt{fig}		&	\verb|\figvref|		&	\figvref{fig:Standardregelkreis}\\
    \bottomrule
	\end{tabular}
\end{table}

\section{Bilder}
\label{sec:Latex-Bilder}

Wenn möglich, sollten Bilder als Vektorgrafik eingebunden werden, damit sichergestellt werden, dass alle Details beim Ausdrucken erhalten bleiben.
Es ist dabei auf eine ausreichende Strichstärke zu achten.
Ebenfalls sollten die im Bild verwendete Schrift die gleiche sein, wie im übrigen Dokument.

Werden doch Pixelgrafiken verwendet, so ist die richtige Wahl der Auflösung von besonderer Bedeutung.
Einerseits sollten die Bilder auf dem ausgedruckten Dokument gut aussehen, andererseits aber auch eine zügige Bildschirmdarstellung und kleine Dateigröße ermöglichen.


Für die Erstellung von Plots eignet sich das \LaTeX-Paket \texttt{pgfplots} sehr gut.
In \appref{cha:Anhang-Grafiken} sind weitere geeignete Programme zur Erstellung von Bildern und Plots mit ihren Eigenschaften aufgelistet.

Eine Grafik lässt sich am einfachsten mit dem Befehl \verb|\includegraphics{}| aus dem \verb|graphicx|-Paket einbinden.
Eine vollständige Beschreibung des Befehls und weiterer nützlicher Grafikbefehle findet man in der Dokumentation des \verb|graphicx|-Pakets.
Diese liegt -- wie die Beschreibung aller anderen \LaTeX-Pakete -- im \verb|doc|-Verzeichnis des \TeX-Systems und kann mit \texttt{texdoc $\left\langle\text{paketname}\right\rangle$} aufgerufen werden.
In welchem Format \verb|graphicx| die Grafiken benötigt, hängt davon ab, ob man \TeX\ in Verbindung mit \textsc{Dvips} verwendet oder stattdessen pdf\TeX, siehe \appref{cha:TexSystem}.

\textbf{pdf\TeX} (empfohlen) verarbeitet dagegen Grafiken im \emph{Portable-Document-Format} (\verb|*.pdf|) sowie die Pixelformate \texttt{jpeg} und \texttt{png}.
Den direkten Export von PDF-Grafiken bieten derzeit zwar nur wenige Programme an, sie lassen sich aber einfach aus dem EPS-Format mit Hilfe des Acrobat Distiller oder Ghostscript erzeugen.
Zum Einbinden von EPS-Dateien muss das Paket \texttt{epsfig} geladen sein.

Für \textbf{\TeX/Dvips} müssen alle Grafiken im \emph{Encapsulated-PostScript}-Format (\verb|*.eps|) vorliegen.
EPS-Dateien lassen sich aus praktisch jeder Software erzeugen und können sowohl Vektor- als auch Pixelgrafiken enthalten -- zusätzlich sind auch Preview-Grafiken möglich, was aber in der \TeX-Welt im Allgemeinen nicht erforderlich ist.

Bilder müssen zentriert sein (\verb|\centering|) und eine Bild\emph{unterschrift} (\verb|\caption{...}|) besitzen.
Um auf eine Abbildung zu referenzieren, kann auch sie mit einem \verb|\label{fig:...}| versehen werden.
Nur in besonderen Ausnahmefällen sollten Bilder mit Text umflossen werden.
Wurden Abbildungen einer Quelle entnommen, muss dies entsprechend mit einem Verweis auf die Quelle im Literaturverzeichnis gekennzeichnet werden.
Wenn dazu das Makro \verb|\cite| verwendet wird, so ist zusätzlich das optionale Argument von \verb|\caption| ohne den Literaturverweis zu verwenden, damit die Verlinkungen im Abbildungsverzeichnis nicht zerstört werden.
Werden Abbildungen eine Quelle nachempfunden, angepasst oder abgeändert, ist dies mit dem Zusatz \glqq in Anlehnung an...\grqq\ oder ähnlich anzugeben.

\begin{figure}[htp]
	\centering
	\input{./tikz/BSB_Beispiel.tikz}
	\caption{Standard-Regelkreis; Bild erstellt mit \texttt{TikZ}}
	\label{fig:Standardregelkreis}
\end{figure}

Generell ist die Arbeit (und insbesondere die Grafiken) so zu gestalten, dass sie auch schwarzweiß gedruckt werden kann.
Farbige Fotos und Screenshots verursachen dabei \iA keine zusätzlichen Probleme.
Werden jedoch \zB farbige Kurven in einem Diagramm dargestellt, hat dies folgende Konsequenzen:
\begin{itemize}
	\item Keine zu hellen Farben für die Linien verwenden, da diese sonst beim Drucken nicht zu erkennen sind.
	RGB-Grün $(0,1,0)$ ist tabu!
	\item Bei mehreren Kurvenverläufen darf deren Farbe nicht das einzige Unterscheidungsmerkmal sein: Entweder sind unterschiedliche Linienformen zu verwenden oder die Kurven im Diagramm beschriften.
\end{itemize}

Bei der Erstellung von Plots ist unbedingt die \href{http://de.wikipedia.org/wiki/DIN_461}{\texttt{DIN 461 Grafische Darstellung in Koordinatensystemen}} zu beachten.
Hilfreich in diesem Zusammenhang und überhaupt für die korrekte Schreibweise von Zahlen und Einheiten im Fließtext und in Formeln sind die Hinweise der TU-Chemnitz: \url{www.tu-chemnitz.de/physik/FPRAK/Grundsatz/Literatur/si_v1.pdf}.

\section{\textsc{tikz}-Externalisierung}
\label{sec:External}
Zum Erstellen von Bilder bietet es sich an, \texttt{tikz} und \texttt{pgfplots} zu verwenden.
Der große Vorteil davon liegt darin, dass die Grafiken direkt in \LaTeX{} kompiliert werden und somit die gleiche Schriftart- und -größe besitzen und die Liniendicke definiert ist.
Dadurch ergibt sich ein schönes Gesamtbild des Dokuments, das wirkt als wäre es aus einem Guss.
Durch das Kompilieren des Bildes direkt in \LaTeX{} kann sich jedoch auch ein Problem ergeben, nämlich dann, wenn die Datenmenge zu groß ist.
Der Kompiler beschwert sich dann mit einem\\
\texttt{[...] ! TeX capacity exceeded, sorry [main memory size=3000000]}\\
oder ähnlich.
Dies wird sehr schnell erreicht, wenn zum Beispiel viele Plots in einem Dokument vorhanden sind.
Abhilfe dagegen schafft das separate Kompilieren der Bilder und anschließende Einbilden derselben als pdf-Dateien.
Dabei ist es einerseits möglich, den in \LaTeX{} eingebauten Mechanismus des Paketes \texttt{tikz} zu verwenden, indem die \texttt{external} Bibliothek durch \verb|\usetikzlibrary{external}| eingebunden wird, die sich um alles Weitere kümmert.
Alternativ kann auch eine eigene Methode verwendet werden, die im Folgenden vorgestellt wird.

\subsection{\texttt{tikzexternal}-Externalisierung}
Damit die in \texttt{tikz} eingebaute Externalisierung funktioniert, muss das Aufrufen von Systembefehlen durch \LaTeX{} erlaubt sein, was sich je nach verwendeter Distribution durch das hinzufügen der Kommandozeilenparameter \zitat\texttt{{-shell-escape}} oder \zitat{\texttt{-enable-write18}} zum Kompilierbefehl erreichen lässt.
Unter \Texstudio{} kann dazu unter \zitat{Befehle} bei pdf\LaTeX{} und \LaTeX{} die entsprechende Zeile geändert werden.


\section{Tabellen}
Tabellen müssen ebenfalls zentriert sein und besitzen eine zentrierte Tabellen\emph{überschrift} (\verb|\caption{...}|).
Hier gelten die gleichen Regeln zur Quellenangabe wie bei den Bildern.
Ein Referenzieren wird auch hier mit \verb|\label{tab:...}| ermöglicht.

Damit Tabellen \glqq schön\grqq\ aussehen, empfiehlt es sich, einige Grundregeln zu beachten.
Es gilt das Prinzip: \emph{weniger ist mehr}.
So sollte auf die Verwendung von senkrechten Linien verzichtet werden und nur wichtige Zeilen, wie zum Beispiel Überschriften, Sinnabschnitte, Unterpunkte, \etc mit horizontalen Linien getrennt werden.
Das Paket \verb|booktabs| stellt Linientypen für den Kopf und Fuß einer Tabelle zur Verfügung.


\begin{table}[htb]
	\begin{center}
		\caption{Parameter}
		\label{tab:Systemparameter}
%		\begin{tabular}{cd{3}d{3}lp{3mm}cd{3}d{3}l}
		\begin{tabular}{ccclp{3mm}cccl}
			\toprule
					&	\multicolumn{1}{c}{Ref.}	&	\multicolumn{1}{c}{Mod.}	&	Einheit					&	&					&	\multicolumn{1}{c}{Ref.}	&	\multicolumn{1}{c}{Mod.}	&	Einheit\\\cmidrule(r){1-4}\cmidrule(l){6-9}
			$m_1$	&	4,0		&	4,63	&	\unit{kg}				&	&	$d_1$			&	0		&	0		&	\unit{\frac{N\,m\,s}{rad}}\\[1mm]
			$m_2$	&	10,1	&	11,15	&	\unit{kg}				&	&	$d_2$			&	0		&	0		&	\unit{\frac{N\,m\,s}{rad}}\\[1mm]
			$m_3$	&	45,7	&	42,5	&	\unit{kg}				&	&	$l_1$			&	0,5		&	0,45	&	\unit{m}\\[1mm]
			$J_1$	&	0,967	&	0,993	&	\unit{kg\,m^2}			&	&	$l_2$			&	1,5		&	1,59	&	\unit{m}\\[1mm]
			$J_2$	&	0,571	&	0,599	&	\unit{kg\,m^2}			&	&	$\varphi_{1,0}$	&	100		&	98,5	&	\unit{\degree}\\[1mm]
			$g$		&	9,81	&	9,81	&	\unit{\frac{m}{s^2}}	&	&	$\varphi_{2,0}$	&	5		&	4,66	&	\unit{\degree}\\
			\bottomrule
		\end{tabular}
	\end{center}
\end{table}

\section{Mathematische Formeln}

Mathematische Formeln, wie 
\begin{equation}
	\label{eq:Allgemein-Approx}
	\int\limits_0^\infty g(x)\ud{}x \approx \sum_{i=1}^n w_i\eexp{x_i} g(x_i)\;,
\end{equation}
werden eingerückt linksbündig dargestellt und nur dann nummeriert, wenn auf sie im Text verwiesen wird.
Für eine bessere Lesbarkeit ist es sinnvoll Gleichungen als Bestandteil des Satzes zu verwenden, so wie mit \equref{eq:Allgemein-Approx}, wobei die nötigen Satzzeichen mit \texttt{\textbackslash{}eqp\{$\left\langle\text{punctuation}\right\rangle$\}} gesetzt werden können.
Die Referenzierung einer erst später aufgeführten Gleichung sollte vermieden werden.

\emph{Alle} mathematischen Ausdrücke (auch wenn es nur einzelne Zeichen sind) werden im mathematischen Modus \verb|$...$| geschrieben, damit sie in der
richtigen Schrift erscheinen.
Hier gilt die Faustregel, dass gewöhnliche mathematische Größen \emph{kursiv} geschrieben werden, Ausdrücke mit konventioneller (feststehender) Bedeutung dagegen in normaler (steiler, aufrechter) Schrift, siehe~\cite{DIN1338}.
Es sind insbesondere Einheiten, Standardfunktionen und -operatoren sowie mathematische Konstanten steil zu schreiben,
\begin{equation*}
	\eexp{ax}\;,\qquad a+\iu{}b\;,\qquad
	\int\limits_{t=0}^{t=1\,\unit{s}} f(t)\cdot\sin(\omega t)\ud{}t\;.
\end{equation*}
Da der mathematische Modus zunächst alle Größen \emph{kursiv} setzt, müssen Ausdrücke, die in aufrechter Schrift erscheinen sollen, mit dem Befehl
\verb|\mathrm{...}| gekennzeichnet werden; bei Textteilen innerhalb einer Formel verwendet man besser \verb|\mbox{...}| oder \verb|\text{...}| (aus dem
\verb|amsmath|-Paket).
Für Standardfunktionen werden von \LaTeX\ die entsprechenden Befehle \verb|\sin|, \verb|\log|, \verb|\max| usw.\ zur Verfügung gestellt.
Konsequenter Weise muss dieses Prinzip auch auf Indizes angewendet werden, also \zB
\begin{equation*}
	\sum_{i,\,j}a_{ij}\,\sin(ijx)\;,\qquad\text{aber:}\qquad
	K_\mathrm{Regler} = 0,5\;.
\end{equation*}
Als Ausnahme von der oben genannten Faustregel werden große griechische Buchstaben meist \emph{nicht} kursiv geschrieben, so auch im mathematischen Modus von
\LaTeX.
Matrizen und Vektoren werden in \textbf{fetten} Buchstaben gesetzt.
Damit sie sich besser von den übrigen Symbolen abheben, werden auch sie nicht \textbf{\emph{fett-kursiv}} sondern \textbf{fett-steil} geschrieben.
Dazu definiert die Datei \texttt{commonmacros.tex} die Befehle
\begin{center}
	\verb|\newcommand*{\mat}[1]{{\ensuremath{\boldsymbol{\mathrm{#1}}}}}|\\
	und\\
	\verb|\newcommand*{\ve}[1]{\ensuremath{\boldsymbol{#1}}}|\,,
\end{center}
die man dann sowohl für Matrizen (Großbuchstaben, \zB\ \verb|\mat{A}|) als auch für Vektoren (Kleinbuchstaben, \zB\ \verb|\ve{x}|) verwenden kann.
\begin{equation*}
	\mat{A} = \begin{bmatrix}
		a_{11} & \ldots & a_{1n}\\
		\vdots & \ddots & \vdots\\
		a_{n1} & \ldots & a_{nn}
	\end{bmatrix}\;,\qquad
	\ve{x} = \begin{bmatrix}
		x_1\\
		x_2
	\end{bmatrix}\;,\qquad
	\ve{\beta}^\transp = \begin{bmatrix}\beta_1	&	\ldots	&	\beta_m\end{bmatrix}
\end{equation*}

Chemische Formelzeichen schreibt man grundsätzlich in aufrechter Schrift.
Variable Größen sind aber auch hier kursiv:
\begin{equation*}
	\mathrm{H_2 O}\;,\qquad
	\mathrm{NO}_x\;,\qquad
	\mathrm{Fe_2^{2+}Cr_2^{\vphantom{2+}}O_4^{\vphantom{2+}}}
\end{equation*}


Beim Referenzieren von Gleichungen muss diese nummeriert werden.
Ist die Gleichung
\begin{equation}
	\label{eq:Approx}
	\int\limits_0^\infty g(x)\ud{}x \approx \sum_{i=1}^n w_i\eexp{x_i} g(x_i)
\end{equation}
mit \verb|\label{eq:Approx}| bezeichnet, erzeugt die Referenz \verb|\equref{eq:Approx}| den Ausdruck \glqq \equref{eq:Approx}\grqq.
Kapitel, Abschnitte, Bilder und Tabellen bekommen keine Klammern, also \zB \verb|\charef{cha:Intro}| für \glqq \charef{cha:Intro}\grqq.
Es ist darauf zu achten, ob es sich um \emph{Kapitel} oder \emph{Abschnitte} handelt.
Vor \verb|\ref{...}| steht ein \emph{festes} Leerzeichen~\verb|~|, damit dort kein Umbruch erfolgen kann, was von den in \tabref{tab:Referenzen} beschriebenen Referenzen befolgt wird.
Literaturangaben werden mit \verb|\cite{...}| anstelle von \verb|\ref{...}| referenziert.

Werden Gleichungen oder Listen in den laufenden Text eingefügt, darf dazwischen kein Absatz (\dah eine Leerzeile im Quelltext) sein.
Um den Quelltext besser zu gliedern, kann an dieser Stelle eine Zeile mit einem \verb|%|-Zeichen eingefügt werden.
Eine Leerzeile darf nur dann im Quelltext stehen, wenn auch wirklich ein Absatz erwünscht ist.
Der Zeilentrenner \verb|\\| erzeugt übrigens keinen Absatz und darf im laufenden Text überhaupt nicht vorkommen.
Der Übersichtlichkeit halber empfiehlt es sich, spätestens nach jedem Satzende eine neue Zeile im Quelltext zu beginnen.


\section{Auszeichnungen und Hervorhebungen}
Wichtige Begriffe werden durch eine andere Schrift hervorgehoben (ausgezeichnet).
Man unterscheidet dabei integrierte und aktive Auszeichnungen.
Integrierte Auszeichnungen sollen erst beim Lesen wahrgenommen werden, sich aber ansonsten in den Text eingliedern.
Die typische Form einer integrierten Auszeichung ist die \emph{kursive} Schrift, die mit \verb|\textit{...}| oder \verb|\emph{...}| erzeugt wird.
Aktive Auszeichnungen sollen dagegen sofort beim Betrachten der Seite auffallen.
Der wichtigste Vertreter ist hier die \textbf{fette} Schrift, die man durch Verwendung von \verb|\textbf{...}| erhält.
In wissenschaftlichen Arbeiten werden vorwiegend integrierte Auszeichnungen benutzt.

Grundsätzlich sollte bei Auszeichnungen immer nur \emph{ein} Attribut geändert werden, also \emph{nicht} gleichzeitig \emph{\underline{\textbf{fett, kursiv und
unterstrichen}}}.
Programmcode und Befehle setzt man üblicherweise mit \verb|\texttt{...}| oder \verb+\verb|...|+ in \texttt{Schreibmaschinenschrift}, Namen gelegentlich mit \verb|\textsc{...}| in \textsc{Kapitälchen}.
Für manche Bezeichnungen kommt eine \textsf{\textbf{fette serifenlose}} Schrift \verb|\textsf{\textbf{...}}| in Frage.
Hier müssen ausnahmsweise \emph{zwei} Attribute geändert werden, da sich die \textsf{serifenlose} Schrift zu wenig vom übrigen Text abhebt.

\glqq Anführungszeichen\grqq\ (siehe \secref{sec:Sonderzeichen}) sind sparsam zu verwenden, \zB bei umgangssprachlichen Begriffen oder wörtlichen Zitaten.
\underline{Unterstreichen} und \mbox{S\,p\,e\,r\,r\,e\,n} sollen überhaupt nicht benutzt werden.
Es ist wichtig, sich zu Beginn der Arbeit zu überlegen, welche Begriffe in welcher Schrift gesetzt werden, und dies konsequent einzuhalten.
So können \bspw die Namen von Autoren mit \texttt{\textbackslash{}name\{$\left\langle\text{Name}\right\rangle$\}}, wie in \name{Dirac}'sche Deltafunktion, einheitlich in Kapitälchen gesetzt werden.


\section{Einbinden von Quellcode}
Wird Quellcode (\Matlab{}, C, \ldots) in der Arbeit angegeben, ist grundsätzlich eine Monospace-Schriftart zu verwenden, da nur so die Lesbarkeit des Codes
gewährleistet werden kann.

Quellcode (\Matlab{}, C, \ldots) kann auf verschiedene Arten eingebunden werden.
Allgemein sollte mittels \verb|\linespread{1}| der ursprüngliche \LaTeX-Zeilenabstand benutzt werden.
Manchmal kann es auch erforderlich sein, die Schrift zu verkleinern oder notfalls sogar die Seiten im Querformat zu beschreiben.
Im laufenden Text sollten nur kleinere Code-Fragmente abgedruckt sein, längere Programme gehören grundsätzlich in den Anhang oder in einen separaten Ordner.

Die einfachste Möglichkeit zum Einbinden von Quellcode ist die \verb|verbatim|-\bzw die \verb|verbatim*|-Umgebung.
Der Code wird in Schreibmaschinenschrift \emph{exakt} (inklusive aller Leer- und Sonderzeichen) so wiedergegeben, wie er im \LaTeX-Quelltext steht.

Komfortablere Möglichkeiten bietet das \verb|listings|-Paket, \zB Syntax-Highlighting mit verschiedenen Schriften oder das Einbinden externer Dateien.
Die Umschaltung auf den einfachen Zeilenabstand muss aber von Hand erfolgen, \zB mittels\\[\parskip]
\hspace*{2em}\verb|\lstset{\basicstyle=\linespread{1}\selectfont}|


\section{Abstände und Sonderzeichen}
\label{sec:Sonderzeichen}
\LaTeX\ interpretiert ein Leerzeichen \verb*| | im Quelltext als normalen Wortzwischenraum.
Nach Befehlen wird es jedoch ignoriert, da es dort nur das Ende des Befehls kennzeichnet.
Soll \zB in dem Satz \glqq\TeX\ ist toll!\grqq\ nach \glqq\TeX\grqq\ ein Leerzeichen erscheinen, dann muss im Quelltext entweder \verb*|\TeX\ | oder \verb|\TeX{}| geschrieben werden.
Im ersten Fall wird durch \verb*|\ | ein Leerzeichen erzwungen, im zweiten Fall wird die leere Umgebung \verb|{}| benutzt, um den Befehl \verb|\TeX| zu beenden.

Manchmal führt ein normales Leerzeichen zu unerwünschten Ergebnissen.
Bei fest verbundenen Begriffen benutzt man ein \emph{festes} Leerzeichens, \zB bei \verb|Dr.~Müller| oder \verb|3~Uhr|, das weder umgebrochen noch gedehnt werden
kann, s.\,a.\ \secref{sec:Latex-Gliederung}.
Bei zusammengesetzten Abkürzungen, beispielsweise \glqq\dah\grqq, \glqq\ua\grqq\ oder \glqq\zB\grqq, wird ein \emph{kleiner} Zwischenraum \verb|\,| verwendet.
Hinter dem zweiten Punkt sollte wieder ein \verb*|\ | stehen, damit dieser nicht als Satzende interpretiert wird.
Der kleine Zwischenraum \verb|\,| steht auch zwischen Zahl und Einheit bei physikalischen Größen, siehe \tabref{tab:Sonderzeichen}.

Im \emph{mathematischen} Modus wird das Komma als Aufzählungszeichen interpretiert und dahinter ein kleiner Abstand eingefügt.
Dies ist jedoch problematisch, da das Komma im Deutschen auch als \emph{Dezimal}komma verwendet wird.
Um den zusätzlichen Abstand zu unterdrücken schreibt man \zB \verb|$2{,}5x$| für \glqq $2{,}5x$\grqq.

Unterschiede sind auch bei den \glqq Strichen\grqq\ zu beachten.
Der \emph{Bindestrich} \verb|-| steht bei zusammengesetzten Wörtern oder Trennungen und wird ohne zusätzlichen Zwischenraum benutzt.
Der \emph{Gedankenstrich} \verb|--| steht bei eingeschobenen Satzteilen und als \glqq Bis-Strich\grqq.
Als Gedankenstrich wird er immer mit einem Leerzeichen davor und dahinter benutzt, als \glqq Bis-Strich\grqq\ ohne Leerzeichen.
Das \emph{Minuszeichen} \verb|$-$| gibt es nur im mathematischen Modus.
\tabref{tab:Sonderzeichen} zeigt Beispiele für die drei Fälle.

Die deutschen Anführungszeichen werden mit \verb|\glqq| und \verb|\grqq{}| \bzw \verb*|\grqq\ | gesetzt.
Keinesfalls dürfen stattdessen englische Anführungszeichen \verb|``...''| oder gar das Zoll-Zeichen \verb|"..."| benutzt werden.
Wie oben erläutert, muss der Befehl \verb|\grqq| mit \verb*|\ | oder mit \verb|{}| abgeschlossen werden, falls danach ein Leerzeichen folgen soll.

\begin{table}
  \caption{Die wichtigsten Abstände und Sonderzeichen.}
  \label{tab:Sonderzeichen}
  \setlength{\tabcolsep}{1em}\centering
  \begin{tabular}{lll}\hline
    Bezeichnung			&	Beispiel				&	Eingabe\\\hline
    Leerzeichen			&	\TeX\ ist toll!			&	\verb*|\TeX\ ist toll!|\\
					    &							&	\verb*|\TeX{} ist toll!|\\
    festes Leerzeichen	&	Dr.~Müller				&	\verb|Dr.~Müller|\\
    kleines Leerzeichen	&	d.\,h.\ 3,5\,km			&	\verb*|d.\,h.\ 3,5\,km|\\
    Bindestrich			&	\TeX-Datei				&	\verb|\TeX-Datei|\\
    Gedankenstrich		&	S.~153--165				&	\verb|S.~153--165|\\
    Minuszeichen		&	$y=5x-2$				&	\verb|$y=5x-2$|\\
    Anführungszeichen	&	\glqq Beispiel\grqq\	&	\verb*|\glqq Beispiel\grqq\ |\\
						&							&	\verb*|\glqq Beispiel\grqq{}|\\\hline
  \end{tabular}
\end{table}



\section{Definition eigener Befehle}
Die Möglichkeit eigene Befehle in \LaTeX{} zu definieren und zu verwenden erleichtert das Erstellen einer wissenschaftlichen Arbeit deutlich.
So dient ein eigener Befehl oft dazu, häufig verwendete Befehlsfolgen kürzer und schneller schreiben zu können.
Von zentraler Bedeutung ist außerdem, dass man diesen Befehl einfach ändern kann.
Hat man \zB alle Matrizen mit einem eigenen Befehl versehen, der diese fett formatiert, so lässt sich dies auch schnell für alle Matrizen wieder ändern.
Entscheidet man sich Matrizen mit einem Unterstrich zu kennzeichnen, so ist lediglich die Anpassung des entsprechenden Befehls notwendig.

Dies lässt sich auch auf Variablennamen übertragen.
Definiert man \zB für $\tilde{\hat{\mat{x}}}_\mathrm{b2}$ einen neuen Befehl \verb|\xb2|, so verkürzt sich zum einen der Schreibaufwand.
Zum anderen lässt sich selbst in der Endphase der Arbeit die Variable umbenennen, \zB in $\mat{z}_2$, indem lediglich der Befehl verändert wird.
Die sinnvolle Verwendung eines Befehls setzt damit voraus, dass er auch immer verwendet wird.

Beispielhafte selbst definierte Befehle, die teilweise hier am Institut verwendet werden, sind in \appref{cha:commonmacros} zu finden.

\section{Sonstiges}
\subsection*{PDF-Ausgabe}
Das \verb|hyperref|-Paket wird benutzt, um  die Ausgabe für das \emph{Portable Document Format} (PDF) zu optimieren.
Dies funktioniert sowohl mit \TeX/Dvips als auch mit pdf\TeX, jedoch kann es bei mehrzeiligen oder umgebrochenen Verlinkungen zu Problemen mit der Positionierung des Links bei \TeX/Dvips kommen.
Besondere Einstellungen sind dafür nicht erforderlich.

Im PDF-Dokument können dann alle Verweise auf Kapitel, Gleichungen, Bilder, Literatur \usw angeklickt werden.
Um eine gute Druckqualität zu gewährleisten, sind diese Links allerdings \emph{nicht} farblich hervorgehoben.

Außerdem werden mit dem Paket \texttt{bookmark} in das PDF-Dokument \emph{Bookmarks} (Lesezeichen) eingebettet, die später als Baumstruktur erscheinen und die Navigation erleichtern.
In den Bookmarks erscheint die Titelseite sowie alle Einträge des Inhaltsverzeichnisses.
Schließlich werden auch noch \emph{Pagelabels} (also \glqq wahre\grqq\ Seitenzahlen) erzeugt, die ebenfalls die Navigation erleichtern und von Vorteil sind, wenn nur Teile des Dokuments gedruckt werden.


\subsection*{Verwenden von \LaTeX-Paketen}
Durch das Einbinden von Zusatzpaketen kann \LaTeX\ angepasst und erweitert werden.
Die Pakete werden mit dem Befehl \verb|\usepackage{...}| eingebunden, \ggf können auch noch Optionen in \verb|[...]| angegeben werden.


%\chapter{Verzeichnisstruktur und vordefinierte Befehle der IAT-Vorlage}
\label{cha:Verzeichnisstruktur}
Es handelt sich bei diesem \LaTeX-Dokument um ein für studentische Arbeiten am Institut für Automatisierungstechnik vorbereitetes Dokument auf Basis der Klasse \verb|tudreport|, \dah die Schriftarten, Pakete und Klassen des TU Designs, wie es vom Fachgebiet Festkörperphysik oder dem Referat Kommunikation angeboten wird, müssen installiert sein, damit ein Dokument mit der Vorlage erstellt werden kann.
Es ist keine neue, abgeleitete Klasse definiert!
Eine Liste von nützlichen Befehlen, die das Paket \texttt{iatsada} zur Verfügung stellt, findet sich in der Dokumentation des Paketes.
Die Dokumentation lässt sich durch Kompilierung der Datei \texttt{iatsada.dtx} erzeugen.

Die Klasse \verb|tudreport| ist aus der Standard-Klasse \verb|scrreprt| abgeleitet und stellt nur wenige neue Befehle zur Verfügung; weitere Funktionen können bei
Bedarf durch Zusatzpakete eingebunden oder selbst definiert werden.
Die Klasse ist daher auch so aufgebaut, dass sie mit möglichst vielen Paketen zusammen arbeitet.
Im Wesentlichen wird das Layout angepasst, wie es in \cite{Richtlinien} festgelegt ist und sich für solche Arbeiten bewährt hat, \zB:
\begin{itemize}
	\item Es wird doppelseitig auf DIN-A4-Papier geschrieben.
	In die zu erstellende PDF-Version werden Bookmarks und Hyperlinks (nicht farbig!) integriert.
	\item Der Abstand der Zeilen beträgt das 1,25-fache des Standard-Abstands von \LaTeX.
	Da technische Arbeiten viele Formeln und Bilder enthalten, werden Absätze durch einen zusätzlichen vertikalen Zwischenraum statt durch einen Einzug getrennt.
	\item Kapitel beginnen immer auf einer neuen Seite.
	\item Die Titelseite hat ein festes Layout mit dem Logo der TU~Darmstadt.
	\item Durch Verwendung der Paketoption \texttt{onlycolorfront=true} des Paketes \texttt{iatsada} werden die Identitätsleisten auf allen Seiten nach dem Deckblatt in Graustufen aufgeführt um Farbe zu sparen.
\end{itemize}
%
%
\section{Verzeichnisse}

\begin{itemize}
	\item \verb|bib|\\Hier wird standardmäßig die Datei \verb|literature.bib| mit den Bibtex-Einträgen erwartet.
	\item \verb|Bilder|\\Vorgesehen für Bilder
	\item \verb|common|\\Allgemeinere Dateien, in die Teile der Definitionen ausgelagert sind, damit die Hauptdatei nicht überfrachtet wird.
	\item \verb|inc|\\Vorgesehen für tex-Dateien mit eigentlichem Inhalt
\end{itemize}


\subsection{Verzeichnis \texttt{common}}
Damit das Hauptdokument nicht überfrachtet wird, sind die folgenden längeren \glqq{}Abschnitte\grqq{} in die angegebenen Dateien im Unterverzeichnis \verb|common| ausgelagert:
\begin{itemize}
	\item \verb|commonmacros.tex|\\
		Definiert einige nützliche Befehle
	\item \verb|header_includes.tex|
		Einbinden der Konfigurationsdateien in der Präambel
	\item \verb|includes.tex|\\
		Beinhaltet alle \verb|\usepackage|-Befehle
	\item \verb|mymacros.tex|
		Definiert eigenen Makros
	\item \verb|pgfplotssetup.tex|
		Definiert Einstellungen und Makros für \texttt{pgfplots}
	\item \verb|pgfsetup.tex|
		Läd alle benötigten \texttt{tikz}-Bibliotheken
	\item \verb|preface.tex|\\
		Generiert die ersten Seiten der Arbeit (Aufgabenstellung, Erklärung, Inhaltsverzeichnis, \etc)
	\item \verb|SADA_Abstract.tex|\\
		Kurzfassung der Arbeit in deutscher und englischer Sprache.
	\item \verb|SADA_Aufgabenstellung.tex|\\
		Aufgabenstellung bei einer studentischen Arbeit. Achtung: für den FB16 muss für das offizielle Exemplar die im Original unterschriebene Aufgabenstellung an dieser Stelle mit gebunden werden.
	\item \verb|TikZ_BSBnormal.tex|
		Definiert einige Makros für Blockschaltbilder
\end{itemize}



%\input{./inc/Zusammenfassung.tex}



% =================================================================================
% Anhang
% =================================================================================
\appendix % Damit wird der Anhang begonnen. Die Kapitel werden ab jetzt mit Buchstaben nummeriert

%\chapter{AP-Regelung Systemparametertests}

\newcommand{\scalea}{0.62}
\begin{figure}[h]
	\centering
	\subfloat[\apaz]{ \includegraphics[scale=\scalea]{Bilder/SysParam Variation/m0/AP2.pdf}	}
	\hfil
	\subfloat[\apad]{	\includegraphics[scale=\scalea]{Bilder/SysParam Variation/m0/AP3.pdf}	}
	\\
	\subfloat[\apave]{ \includegraphics[scale=\scalea]{Bilder/SysParam Variation/m0/AP41.pdf} }
	\hfil
	\subfloat[\apavz]{ \includegraphics[scale=\scalea]{Bilder/SysParam Variation/m0/AP42.pdf}	}
	\caption{Maximale Startwerte -- Variation $m_0$}
	\label{fig:sysvarm0}
\end{figure}

\renewcommand{\scalea}{0.38}
\begin{figure}[h]
	\centering
	\subfloat[\apaz]{ \includegraphics[scale=\scalea]{Bilder/SysParam Variation/s1/AP2.pdf}	}
	%\hfil
	\subfloat[\apad]{	\includegraphics[scale=\scalea]{Bilder/SysParam Variation/s1/AP3.pdf}	}
	\\
	\subfloat[\apave]{ \includegraphics[scale=\scalea]{Bilder/SysParam Variation/s1/AP41.pdf} }
	%\hfil
	\subfloat[\apavz]{ \includegraphics[scale=\scalea]{Bilder/SysParam Variation/s1/AP42.pdf}	}
	\caption{Maximale Startwerte -- Variation $s_1$}
	\label{fig:sysvars1}
\end{figure}

\begin{figure}
	\centering
	\subfloat[\apaz]{ \includegraphics[scale=\scalea]{Bilder/SysParam Variation/s2/AP2.pdf}	}
	%\hfil
	\subfloat[\apad]{	\includegraphics[scale=\scalea]{Bilder/SysParam Variation/s2/AP3.pdf}	}
	\\
	\subfloat[\apave]{ \includegraphics[scale=\scalea]{Bilder/SysParam Variation/s2/AP41.pdf} }
	%\hfil
	\subfloat[\apavz]{ \includegraphics[scale=\scalea]{Bilder/SysParam Variation/s2/AP42.pdf}	}
	\caption{Maximale Startwerte -- Variation $s_2$}
	\label{fig:sysvars2}
\end{figure}

\begin{figure}
	\centering
	\subfloat[\apaz]{ \includegraphics[scale=\scalea]{Bilder/SysParam Variation/l1/AP2.pdf}	}
	%\hfil
	\subfloat[\apad]{	\includegraphics[scale=\scalea]{Bilder/SysParam Variation/l1/AP3.pdf}	}
	\\
	\subfloat[\apave]{ \includegraphics[scale=\scalea]{Bilder/SysParam Variation/l1/AP41.pdf} }
	%\hfil
	\subfloat[\apavz]{ \includegraphics[scale=\scalea]{Bilder/SysParam Variation/l1/AP42.pdf}	}
	\caption{Maximale Startwerte -- Variation $l_1$}
	\label{fig:sysvarl1}
\end{figure}

%\chapter{Befehle in \texttt{commonmacros.tex}}
\label{cha:commonmacros}
Hier sind im Folgenden kurz einige in \texttt{commonmacros.tex} definierte Befehle aufgelistet.

%
\section*{Einheiten}
Die folgenden Befehle funktionieren im Mathe- und Textmodus (\dah es wird im Textmodus automatisch für den Befehl in den Mathemodus umgeschaltet):
\begin{itemize}
	\item Einheit (Aufrechte Schrift im Mathemodus)\\ \verb|\unit{\frac{N}{m}}| $\rightarrow$ \unit{\frac{N}{m}}
	\item Zahl mit Einheit\\(Setzt \glqq{}kleines\grqq{} Leerzeichen zwischen Zahl und Einheit, Zahl und Einheit automatisch im Mathemodus, Einheit in aufrechter Schrift)\\ \verb|\valunit{34,3}{cm}| $\rightarrow$ \valunit{34,3}{cm}
	\item (Das aufrechte $\mu$ gibt es mit dem Befehl \verb|\upmu| aus dem Paket upgreek)\\ \verb|\valunit{4}{\upmu m}| $\rightarrow$ \valunit{4}{\upmu m}
\end{itemize}

\noindent Besondere Einheiten
\begin{itemize}
	\item Gradzeichen (Funktioniert im Text- und Mathemodus)\\ \verb|\degree| $\rightarrow$ \degree
	\item Grad Celsius (Funktioniert im Text- und Mathemodus)\\ \verb|\degC| $\rightarrow$ \degC
\end{itemize}


\section*{Vektoren und Matrizen}
\begin{itemize}
	\item Vektor\\ \verb|\ve{x}| $\rightarrow$ \ve{x}
	\item Matrix\\ \verb|\mat{A}| $\rightarrow$ \mat{A}
	\item Vektor Sonderzeichen\\ \verb|\ves{\lambda}| $\rightarrow$ \ves{\lambda}
	\item Matrix Sonderzeichen\\ \verb|\mas{\Lambda}| $\rightarrow$ \mas{\Lambda}
\end{itemize}
Wichtig: Mathematische Akzente müssen dabei geklammert werden!
\begin{itemize}
	\item \verb|$\dot{\ve{x}}$| $\rightarrow$  $\dot{\ve{x}}$
	\item \verb|$\dot{\tilde{\ve{x}}}$| $\rightarrow$ $\dot{\tilde{\ve{x}}}$
\end{itemize}
\verb|\ve{}| und \verb|\mat{}| \bzw \verb|\ves{}| und \verb|\mas{}| machen jeweils genau das gleiche. Die Unterscheidung dient nur zur besseren Lesbarkeit.

\begin{itemize}
	\item Transponiert-Zeichen (aufrechtes T)\\ \verb|$\mat{A}^\transp$| $\rightarrow$ $\mat{A}^\transp$
\end{itemize}


\section*{Funktionen und Abkürzungen}
\begin{itemize}
	\item Unterstreichen\\ \verb|$\ul{x}$| $\rightarrow$ $\ul{x}$
	\item Innenprodukt\\ \verb|$\inprod{f}{g}$| $\rightarrow$ $\inprod{f}{g}$
	\item Exponentialschreibweise\\ \verb|$45\E{-2}$| $\rightarrow$ $45\E{-2}$
	\item e-Funktion\\ \verb|$\eexp{t}$| $\rightarrow$ $\eexp{t}$
	\item Rang\\ \verb|$\rang{\mat{A}}$| $\rightarrow$ $\rang{\mat{A}}$
	\item Imaginäre Einheit (aufrechtes j)\\ \verb|$5+\iu 2$| $\rightarrow$ $5+\iu 2$
	\item \glqq{}Von-Bis-Punkte\grqq{} mit Kommas und schönen Abständen\\ \verb|$1 \todots n$| $\rightarrow$ $1 \todots n$
	\item i abgeleitet\\ \verb|$\doti$| $\rightarrow$ $\doti$
	\item Aufrechte Schrift (Abkürzung für \verb|\mathrm{}|) \\ \verb|$\mrm{abc}$| $\rightarrow$ $\mrm{abc}$
	\item Normaler Text in Formel (Abkürzung für \verb|\textnormal{}|)\\ \verb|$\tn{ab für}$| $\rightarrow$ $\tn{ab für}$
	\item Geklammerte Gruppe mit Subscript\\ \verb|$\grpsb{\frac{1}{2}}{x}$| $\rightarrow$ $\grpsb{\frac{1}{2}}{x}$
	\item Geklammerte Gruppe mit aufrechtem Subscript\\ \verb|$\grprsb{\frac{1}{2}}{x}$| $\rightarrow$ $\grprsb{\frac{1}{2}}{x}$
\end{itemize}



\section*{Ableitungen und Integrale}
\begin{itemize}
	\item Normale Ableitung\\ \verb|$\normd{f}{x}$| $\rightarrow$ $\normd{f}{x}$
	\item Materielle Ableitung\\ \verb|$\matd{f}{x}$| $\rightarrow$ $\matd{f}{x}$
	\item Partielle Ableitung\\ \verb|$\partiald{f}{x}$| $\rightarrow$ $\partiald{f}{x}$
	\item Beispiel höhere Ableitung\\ \verb|$\normd{^2 f}{x^2} \qquad \partiald{^2 f}{x \partial y}$| $\rightarrow$ $\normd{^2 f}{x^2} \qquad \partiald{^2 f}{x \partial y}$
	\item Normale Ableitung an\\ \verb|$\normdat{f}{x}{x=0}$| $\rightarrow$ $\normdat{f}{x}{x=0}$
	\item Materielle Ableitung an\\ \verb|$\matdat{f}{x}{x=0}$| $\rightarrow$ $\matdat{f}{x}{x=0}$
	\item Partielle Ableitung an\\ \verb|$\partialdat{f}{x}{x=0}$| $\rightarrow$ $\partialdat{f}{x}{x=0}$
	\item Aufrechtes \glqq{}d\grqq{} für Integral\\ \verb|$\ud$| $\rightarrow$ $\ud$
	\item Beispiel für Integral\\ \verb|$\int f(x) \ud x$| $\rightarrow$ $\int f(x) \ud x$
\end{itemize}


\section*{Transformationen}
\begin{itemize}
	\item \verb|$\Laplace{x}$| $\rightarrow$ $\Laplace{x}$
	\item \verb|$\InvLaplace{X}$| $\rightarrow$ $\InvLaplace{X}$
	\item \verb|$x \trans X$| $\rightarrow$ $x \trans X$
	\item \verb|$X \invtrans x$| $\rightarrow$ $X \invtrans x$
	\item \verb|$\FT{x}$| $\rightarrow$ $\FT{x}$
	\item \verb|$\FTabs{x}$| $\rightarrow$ $\FTabs{x}$
	\item \verb|$\IFT{x}$| $\rightarrow$ $\IFT{x}$
	\item \verb|$\DFT{x}$| $\rightarrow$ $\DFT{x}$
	\item \verb|$\DFTabs{x}$| $\rightarrow$ $\DFTabs{x}$
\end{itemize}




\section*{Verweise}
Verweise auf verschiedene Objekte mit passendem Text (\glqq{}Abbildung X\grqq{}, \glqq{}Tabelle X\grqq{}).
Dabei ist dann immer der komplette Text ein Hyperlink, und nicht nur die Zahl.
\begin{itemize}
	\item Abbildung\\ \verb|\figref{label}|
	\item Tabelle\\ \verb|\tabref{label}|
	\item Gleichung\\ \verb|\equref{label}|
	\item Definition\\ \verb|\defref{label}|
	\item Kapitel\\ \verb|\charef{label}|
	\item Anhang\\ \verb|\appendixref{label}|
	\item Abschnitt\\ \verb|\secref{label}|
	\item Listing\\ \verb|\lstref{label}|
	\item Algorithmus\\ \verb|\algoref{label}|
	\item Seite\\ \verb|\pagerefh{label}|
	\item Fußnote\\ \verb|\ftnref{label}|
\end{itemize}

\ZT auch auf Varioref basierend (\glqq{}Abbildung 23 auf dieser Seite\grqq{}, \glqq{}Abbildung 23 auf Seite 45\grqq)
\begin{itemize}
	\item Abbildung\\ \verb|\figvref{label}|
	\item Tabelle\\ \verb|\tabvref{label}|
	\item Gleichung\\ \verb|\equvref{label}|
\end{itemize}


\section*{Abkürzungen}
Abkürzungen mit Punkt \glqq{}dazwischen\grqq{} (wird mit kleinen Abständen gesetzt)\\
\verb|\dah| $\rightarrow$ \dah, \verb|\Dah| $\rightarrow$ \Dah, \verb|\iA| $\rightarrow$ \iA, \verb|\IA| $\rightarrow$ \IA, \verb|\ua| $\rightarrow$ \ua, \verb|\Ua| $\rightarrow$ \Ua, \verb|\uU| $\rightarrow$ \uU, \verb|\UU| $\rightarrow$ \UU, \verb|\zB| $\rightarrow$ \zB, \verb|\ZB| $\rightarrow$ \ZB, \verb|\zT| $\rightarrow$ \zT, \verb|\ZT| $\rightarrow$ \ZT

\vspace{1ex}
\noindent Abkürzungen mit Punkt, bei denen der Punkt nicht als Satzende interpretiert wird:\\
\verb|\bspw| $\rightarrow$ \bspw, \verb|\Bspw| $\rightarrow$ \Bspw, \verb|\bzw| $\rightarrow$ \bzw, \verb|\Bzw| $\rightarrow$ \Bzw,  \verb|\bzgl| $\rightarrow$ \bzgl, \verb|\ca| $\rightarrow$ \ca, \verb|\evtl| $\rightarrow$ \evtl, \verb|\ggf| $\rightarrow$ \ggf, \verb|\Ggf| $\rightarrow$ \Ggf, \verb|\usw| $\rightarrow$ \usw, \verb|\vgl| $\rightarrow$ \vgl, \verb|\Vgl| $\rightarrow$ \Vgl



\section*{Listingdefintionen}
\begin{itemize}
	\item \verb|Matlab_colored|
	\item \verb|Matlab_colored_smallfont|
\end{itemize}


\begin{lstlisting}[style=Matlab_colored, caption = {Beispiellisting, style=Matlab\_colored}, label={lst:Listing1}]
function [] = animierePunkt(inY, inX)

temp = length(inY);

%% [...]

%% -------------------------------------------------------------
for i=1:temp
    if i>1
        delete(p(i-1));
    end
    p(i) = plot(inX(i),inY(i),'Marker','o','MarkerSize',10);
    pause(0.025);
end
hold off;
\end{lstlisting}


\begin{lstlisting}[style=Matlab_colored_smallfont, caption = {Beispiellisting, style=Matlab\_colored\_smallfont}, label={lst:Listing2}]
function [] = animierePunkt(inY, inX)

temp = length(inY);

%% [...]

%% ---------------------------------------------------------------------
for i=1:temp
    if i>1
        delete(p(i-1));
    end
    p(i) = plot(inX(i),inY(i),'Marker','o','MarkerSize',10);
    pause(0.025);
end
hold off;
\end{lstlisting}

\section*{Sonstiges}
Latex gibt beim Umwandeln \zT Fehler aus, wenn Zeichen aus dem \texttt{textcomp}-Paket verwendet werden, da diese nicht in den TU-Schriften vorhanden sind.
Mit \verb|\textcompstdfont{}| wird die Schriftart für den Text im Argument explizit umgeschaltet, und so der Fehler vermieden:
\begin{itemize}
	\item \verb|\textcompstdfont{\textuparrow}| $\rightarrow$ \textcompstdfont{\textuparrow}
\end{itemize}



% =================================================================================
% Literaturverzeichnis
% =================================================================================
\cleardoublepage        % Auf eine leere Seite einfügen
\phantomsection         % Für Aufnahme ins Inhaltsverzeichnis
\addcontentsline{toc}{chapter}{\bibname}  % In Inhaltsverzeichnis von
                                          % Dokument und pdf aufnehmen
\printbibliography 
% =================================================================================

	
\end{document}
