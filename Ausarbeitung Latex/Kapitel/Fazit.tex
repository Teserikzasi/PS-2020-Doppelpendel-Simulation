\chapter{Fazit und Ausblick}

\section{Zusammenfassung und Fazit}

Diese Arbeit beschäftigt sich mit der Steuerung und Regelung eines Doppelpendels, das an einem Schlitten angebracht ist.
Das System wurde in dieser Arbeit rein simulativ betrachtet, wobei die Anwendung am realen Versuchsstand stets im Hintergrund behalten wurde.
Es wurde ein umfassendes Simulationsmodell in \ml/\sm\ aufgebaut, welches in zukünftigen Arbeiten weiterverwendet werden kann.

Das \spds\ und der Motor wurden sehr detailliert modelliert.
Im Vergleich zu bisherigen Arbeiten wurde beispielsweise die Induktionsspannung des Motors und die \crb\ berücksichtigt.
Die Herleitung der \bwgl\ des \spds s wurde symbolisch umgesetzt, wodurch das System sehr einfach erweitert werden kann.
Hierdurch konnten Fehler in den zuvor verwendeten Gleichungen aufgedeckt werden.
Das gesamte Simulationsmodell wurde sehr modular und funktional aufgebaut.
Somit können unterschiedliche Parameter einfach getestet und verglichen werden.

Anschließend wurde die Regelung an den vier \ap en betrachtet und insbesondere die Regelbarkeit untersucht.
Dazu wurde zunächst ein Regelungsmodell aufgebaut, das einen \ap-abhängigen \zsr\ und eine \vorst seinheit enthält.
Auch hier wurde auf eine saubere Implementierung geachtet, sodass einfach Änderungen und Vergleiche sowie ausführliche Auswertungen der Simulationsdaten möglich sind.
Um Tests/Testreihen effektiv auswerten zu können, wurden die \xots-Funktionen programmiert, welche das Simulationsmodell automatisch ausführen und auf denen weitere Tests aufbauen können.

Nach Verbesserungen in der \vorst\ und dem \beob\ wurde eine weitere Verbesserung der Regelung über eine Optimierung der Reglerparameter betrachtet.
Dazu werden die Güteparameter des \ricc-Reglers schrittweise variiert und deren Einfluss mithilfe von verschiedenen Diagrammen analysiert.
Durch die Optimierung wurden für alle \ap e Parameter ermittelt, die das System besser stabilisieren können als vorherige Werte.

In einem weiteren Schritt wurde der Einfluss der Parameter des \dpd s untersucht.
Dabei wurde wieder von den Test- und Auswertungsfunktionen Gebrauch gemacht, um direkt ein umfassendes Bild des Regelverhaltens zu erlangen.
Es konnte herausgefunden werden, dass Trägheits- und Geometrieparameter einen erheblichen Einfluss auf die Stabilisierbarkeit des Systems haben.
Die Ergebnisse sind allerdings mit Vorsicht zu betrachten, da die Systeme mit geänderten Parametern durch eine zusätzliche Optimierung der Reglerparameter unter Umständen ein besseres Verhalten aufweisen können.

Außerdem zeigte sich, dass die erhöhte \crb\ des ersten Pendelgelenks aufgrund des bei der Neukonstruktion hinzugekommenen Schleifrings einen deutlich negativen Einfluss hat.
Dies betrifft zum einen die Stabilisierbarkeit, aber auch das stationäre Verhalten, da sich dadurch ein nicht ausregelbarer Grenzzyklus einstellt.
Insbesondere in Kombination mit dem \beob\ ist die Regelbarkeit wenig robust.



\traj

  

\section{Ausblick}

Die in dieser Arbeit optimierte Regelung und Steuerung kann am realen Versuchsstand erprobt werden.
Möglicherweise führen die extra für den \beob, welcher ja am Versuchsstand eingesetzt wird, optimierten Reglerparameter wieder zu einer gut stabilisierenden \aprg.
Dabei ist zu beachten, dass für die Optimierung die Stabilität der Winkelausregelung im Vordergrund stand.
Für andere Szenarien wie horizontale Fahrten, müssen die Reglerparameter eventuell nach anderen Gesichtspunkten optimiert werden.

Um die Regelung weiter zu verbessern, könnten die bisherigen Regelungskonzepte überdacht und erweitert werden.
Aktuell basiert die \aprg\ auf einer \lin, sodass ein linearer \zsr\ und \beob\ verwendet werden kann.
Da die angesprochene \crb\ das System jedoch stark nichtlinear macht, sollte die Verwendung nichtlinearer Regelungen geprüft werden.
In einem ersten Schritt sollte der \beob\ um die nichtlinearen Effekte ergänzt werden, da sich ansonsten große Schätzfehler ergeben, die das Regelverhalten verschlechtern.
Alternativ kann die angesprochene Variante \diff\ als \ze\ möglicherweise bessere Ergebnisse erzielen, da hier keine \ap abhängigkeit oder weitere Modellungenauigkeiten beachtet werden müssen.
Da selbst bei der Variante \zm\ Schwierigkeiten wie der Grenzzyklus auftraten, sollte auch beim \zsr\ über nichtlineare Regelungskonzepte nachgedacht werden.
Im Gegensatz zum Schlitten lässt sich die \crb\ in den Gelenken nämlich nicht über eine Vorsteuerung kompensieren.
Eventuell lässt sich der Grenzzyklus aber auch durch eine weitere Optimierung der Reglerparameter verringern, wobei für eine Einflussanalyse auf dem erstellten Code aufgebaut werden kann.

Des Weiteren kann die Regelung durch Änderungen am System selbst beeinflusst werden.
Für eine Entscheidungsfindung und weitere Test- und Analysemöglichkeiten stellen die Ergebnisse dieser Arbeit eine Grundlage dar.
Eine direkte Aussage zum möglichen Anbringen zusätzlicher Gewichte am Pendel ist schwierig, da die Konstruktionsparameter teilweise direkt voneinander abhängen und erste Parametertests gegensinnige Tendenzen zeigten, die sich zudem je nach \ap\ unterscheiden.
Es muss daher stets ein Kompromiss getroffen werden.
Für weiterführende Analysen und Aussagen sind eine genauere Modellierung der Massenverteilung sowie weitere Tests und Auswertungen nötig.
Der in dieser Arbeit erstellte Code kann dafür verwendet werden.

Darüber hinaus kann auch eine grundsätzlich neue Konstruktion angedacht werden, die speziell auf besonders gute Regelbarkeit ausgelegt wird.
Mithilfe der Systemparametertests kann der Einfluss aller Parameter systematisch untersucht werden.
Da sich gezeigt hat, dass manche Parametervariationen die Stabilisierbarkeit deutlich erhöhen, kann durch Optimieren aller Parameter eine sehr gut regelbare Konstruktion erstellt werden.

Der Einfluss der Systemparameter des Motors wurde in dieser Arbeit nicht betrachtet.
Genauso gut könnten die Parametervariationen auch für Motorparameter durchgeführt werden.
Der Code müsste dafür nur geringfügig erweitert werden.
Somit könnte der Einfluss des Motors untersucht werden und unter Umständen bei zukünftigen Änderungen am System berücksichtigt werden.