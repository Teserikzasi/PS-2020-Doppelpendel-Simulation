\chapter{Fazit und Ausblick}

\section{Zusammenfassung und Fazit}

Diese Arbeit beschäftigt sich mit der Steuerung und Regelung eines Doppelpendels, das an einem Schlitten angebracht ist.
Das System wurde in dieser Arbeit rein simulativ betrachtet, wobei die Anwendung am realen Versuchsstand stets im Hintergrund behalten wurde.
Es wurde ein umfassendes Simulationsmodell in \ml/\sm\ aufgebaut, welches in zukünftigen Arbeiten weiterverwendet werden kann.

Das \spds\ und der Motor wurden sehr detailliert modelliert.
Im Vergleich zu bisherigen Arbeiten wurde beispielsweise die Induktionsspannung des Motors und die \crb\ berücksichtigt.
Die Herleitung der \bwgl\ des \spds s wurde symbolisch umgesetzt, wodurch das System sehr einfach erweitert werden kann.
Außerdem konnten dadurch Fehler in den zuvor verwendeten Gleichungen aufgedeckt werden.
Das gesamte Simulationsmodell wurde sehr modular und funktional aufgebaut.
Somit können sehr einfach unterschiedliche Parameter getestet und verglichen werden.

Anschließend wurde die Regelung an den vier \ap en betrachtet und insbesondere die Regelbarkeit untersucht.
Dazu wurde zunächst ein Regelungsmodell aufgebaut, das einen \ap-abhängigen \zsr\ und eine \vorst seinheit enthält.
Auch hier wurde auf eine saubere Implementierung geachtet, sodass einfach Änderungen und Vergleiche sowie eine ausführliche Auswertung der Simulationsdaten möglich sind.
Um Tests/Testreihen effektiv auswerten zu können, wurden die \xots-Funktionen programmiert, welche das Simulationsmodell automatisch ausführen und auf denen weitere Tests aufbauen können.

Nach Verbesserungen in der \vorst\ und dem \beob\ wurde eine weitere Verbesserung der Regelung über eine Optimierung der Reglerparameter betrachtet.
Dazu werden die Güteparameter des \ricc-Reglers schrittweise variiert und deren Einfluss mithilfe von verschiedenen Diagrammen analysiert.
Durch die Optimierung wurden Parametersätze für alle \ap e bestimmt, die das System besser stabilisieren können als vorige Werte.

In einem weiteren Schritt wurde der Einfluss der Parameter des \dpd s untersucht.
Dabei wurde wieder von den Test- und Auswertungsfunktionen Gebrauch gemacht, um direkt ein umfassendes Bild des Regelverhaltens zu erlangen.
Es konnte herausgefunden werden, dass Trägheits- und Geometrieparameter einen erheblichen Einfluss auf die Stabilisierbarkeit des Systems haben.

Außerdem zeigte sich, dass die \crb\ des ersten Pendelgelenks einen negativen Einfluss auf die Regelbarkeit und das stationäre Verhalten hat, insbesondere in Kombination mit der Verwendung eines \beob s.


\traj

  

\section{Ausblick}

