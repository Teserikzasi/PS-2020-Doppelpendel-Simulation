\chapter{Fazit und Ausblick}

\section{Zusammenfassung und Fazit}

Diese Arbeit beschäftigt sich mit der Steuerung und Regelung eines Doppelpendels, das an einem Schlitten angebracht ist.
Das System wurde in dieser Arbeit rein simulativ betrachtet, wobei die Anwendung am realen Versuchsstand stets im Hintergrund behalten wurde.
Es wurde ein umfassendes Simulationsmodell in \ml/\sm\ aufgebaut, welches in zukünftigen Arbeiten weiterverwendet werden kann. 

Im Vergleich zu den bisherigen Arbeiten wurden hierbei erstmalig die Induktionsspannung des Motors und die \crb\ berücksichtigt.
Die Herleitung der \bwgl\ des \spds s wurde symbolisch umgesetzt, wodurch das System leicht erweitert werden kann.
Hierdurch konnten Fehler in den Systemgleichungen, die bislang in den vergangenen Arbeiten verwendet worden waren, aufgedeckt werden.
Bei allen im Rahmen dieser Arbeit realisierten Implementierungen wurde auf eine sinnvolle Struktur und einen modularen Aufbau geachtet.
Der zugehörige \ml-Code wurde funktional aufgebaut, wodurch unterschiedliche Parameter einfach getestet und verglichen werden können.

Die Regelung wurde an den vier \ap en betrachtet und jeweils deren Regelbarkeit untersucht.
Dazu wurde zunächst ein Regelungsmodell aufgebaut, das einen \ap-abhängigen \zsr\ und eine \vorst seinheit enthält.
Um Testreihen effektiv auswerten zu können, wurden die \xots-Funktionen programmiert, welche das Simulationsmodell automatisch ausführen und auf denen weitere Tests aufbauen können.

Nach einzelnen Weiterentwicklungen der \vorst\ und des \beob s wurde eine weitere Erhöhung der Regelgüte durch Variation der Reglerparameter erreicht.
Dazu wurden die Güteparameter des \ricc-Reglers schrittweise variiert und deren Einfluss mithilfe verschiedener Diagramme analysiert.
Durch die Optimierung wurden für alle \ap e Parameter ermittelt, die das System gegenüber den vorherigen Werten besser stabilisieren können.

In einem weiteren Schritt wurde der Einfluss der Parameter des \dpd s auf die Arbeitspunktregelung untersucht.
Die Untersuchungen zeigen, dass Trägheits- und Geometrieparameter einen erheblichen Einfluss auf die Stabilisierbarkeit des Systems haben.
Die Ergebnisse sind jedoch unter dem Vorbehalt zu betrachten, dass die Systeme mit geänderten Parametern durch eine zusätzliche Optimierung der Reglerparameter unter Umständen ein besseres Verhalten aufweisen können. 

Außerdem zeigt sich, dass die erhöhte \crb\ des ersten Pendelgelenks aufgrund des bei der Neukonstruktion hinzugekommenen Schleifrings einen erkennbar negativen Einfluss hat.
Dies betrifft zum einen die Stabilisierbarkeit, aber auch das stationäre Verhalten, da sich dadurch ein nicht ausregelbarer Grenzzyklus einstellt.
Insbesondere in Kombination mit dem \beob\ ist die Regelbarkeit wenig robust.

Weiterhin wurden eine NMPC für die Trajektorienberechnung sowie eine Trajektorienfolgeregelung realisiert. Anknüpfend an die Ergebnisse der Vorgängerarbeit Fauvé \cite{fauve} konnte eine günstige Konfiguration der NMPC sowie eine geeignete Auslegung der Trajektorienfolgeregelung ermittelt werden, für die sich die betrachteten Aufschwungtrajektorien sowohl am Schlittenpendelmodell als auch am Gesamtmodell unter Berücksichtigung der Gegeninduktion des Motors stabilisieren lassen.

Die Untersuchung des Einflusses der Systemparameter auf die Trajektorienberechnung zeigt, dass besonders die Pendelmassen eine erkennbare Eingrenzung des Bereichs, in dem Trajektorien gefunden werden, aufweisen. Neben der Analyse der Verteilung gültiger und ungültiger Trajektorien sowie der Trajektoriengüte, wurden daher Begrenzungen der Wertebereiche für die untersuchten Parameter formuliert, die zur Orientierung für zukünftige konstruktive Veränderungen am Versuchsstand dienen können. 
Das ungünstige Verhalten der aktuellen Systemparameter in Bezug auf die Trajektorienberechnung wird durch die Untersuchung bestätigt, indem überwiegend keine Trajektorien gefunden werden. %unklar vgl alte und neue parameter
Eine Begründung durch die ermittelten Bereichsgrenzen ist jedoch nur teilweise möglich, da oft auch dann keine Trajektorie gefunden wird, wenn sich der Parameter des aktuellen Versuchsstands im empfohlenen Wertebereich befindet. Einen erkennbaren Veränderungsbedarf signalisieren die Untersuchungen in erster Linie für das Trägheitsmoment von Stab 2. 
Insgesamt sind die Ergebnisse jedoch aufgrund der vielfältigen Einflussgrößen auf die Optimierungslandschaft der NMPC zu problematisieren. %unklar

  

\section{Ausblick}

Die in dieser Arbeit optimierte Regelung und Steuerung kann am realen Versuchsstand erprobt werden.
Möglicherweise führen die extra für den \beob, welcher auch am Versuchsstand eingesetzt wird, optimierten Reglerparameter wieder zu einer gut stabilisierenden \aprg.
Dabei ist zu beachten, dass für die Optimierung die Stabilität der Winkelausregelung im Vordergrund stand.
Für andere Szenarien wie horizontale Fahrten, müssen die Reglerparameter gegebenenfalls nach anderen Gesichtspunkten optimiert werden.

Um die Arbeitspunktregelung weiter zu verbessern, könnten die bisherigen Regelungskonzepte überdacht und erweitert werden.
Aktuell basiert die \aprg\ auf einer \lin, sodass ein linearer \zsr\ und \beob\ verwendet werden kann.
Da die angesprochene \crb\ das System jedoch stark nichtlinear macht, sollte die Verwendung nichtlinearer Regelungen geprüft werden.
In einem ersten Schritt sollte der \beob\ um die nichtlinearen Effekte ergänzt werden, da sich ansonsten große Schätzfehler ergeben, die das Regelverhalten verschlechtern.
Alternativ kann die angesprochene Variante \diff\ als \ze\ möglicherweise bessere Ergebnisse erzielen, da hier keine \ap abhängigkeit oder weitere Modellungenauigkeiten beachtet werden müssen.
Da selbst bei der Variante \zm\ Schwierigkeiten wie der Grenzzyklus auftreten, sollten auch beim \zsr\ nichtlineare Regelungskonzepte in Erwägung gezogen werden.
Im Gegensatz zum Schlitten lässt sich die \crb\ in den Gelenken nämlich nicht über eine Vorsteuerung kompensieren.
Eventuell lässt sich der Grenzzyklus aber auch durch eine weitere Optimierung der Reglerparameter verringern, wobei für eine Einflussanalyse auf dem erstellten Code aufgebaut werden kann.

Des Weiteren kann die Regelbarkeit durch Änderungen am System selbst beeinflusst werden.
Für eine Entscheidungsfindung und weitere Test- und Analysemöglichkeiten stellen die Ergebnisse dieser Arbeit eine Grundlage dar.
Eine direkte Aussage zur Modifizierung des aktuellen Versuchsstands durch Anbringen zusätzlicher Gewichte am Pendel oder durch Neukonstruktion  ist problematisch, da die Konstruktionsparameter teilweise direkt voneinander abhängen und erste Parametertests gegensinnige Tendenzen zeigten, die sich zudem je nach \ap\ unterscheiden.
Es muss daher stets ein Kompromiss getroffen werden.
Für weiterführende Analysen und Aussagen sind eine genauere Modellierung der Massenverteilung sowie weitere Tests und Auswertungen nötig.

Die Ergebnisse der Parameteruntersuchungen, die für die Trajektorienberechnung mittels NMPC durchgeführt wurden, gelten für die im Rahmen dieser Arbeit definierten Voraussetzungen. %nicht wirklich aussagekräftig
Als nächster Schritt steht somit die Überprüfung der Gültigkeit bei einer Änderung der Voraussetzungen wie beispielsweise der Variation der Vergleichstrajektorie, der Kraftbeschränkung und den Konfigurationsmöglichkeiten der NMPC aus. Die ermittelten Bereichsverfehlungen der aktuellen Systemparameter der Versuchsstands, wie beispielsweise für das Massenträgheitsmoment von Stab 2, sind hierbei besonders zu beobachten und nach ausreichender Prüfung am Versuchsstand gegebenenfalls zu korrigieren. Es ist hierbei immer auf mögliche Kompromisse bezüglich der in dieser Arbeit untersuchten Einflüsse auf die Arbeitspunktregelung zu achten.

Der Einfluss der \crb\ in Bezug auf die Trajektorienberechnung wurde nicht näher untersucht. Erste Tests weisen jedoch auf einen erkennbar negativen Einfluss hin. Eine Variation der Reibwerte ist daher für zukünftige Arbeiten denkbar.

Die Güte der Trajektorien wurde in die Wahl der Bereichsgrenzen zunächst nicht näher einbezogen und in der Analyse durch Erfahrungswerte für die spätere Stabilisierbarkeit in der Simulation eingeordnet. Da die durchgeführten Untersuchungen zur Stabilisierbarkeit \bzw Regelbarkeit der berechneten Trajektorien sich auf die Tests zur Wahl einer günstigen Ausgangskonfiguration der NMPC für die anschließenden Parametertests im Rahmen dieser Arbeit beschränken, bleibt eine detailliertere Untersuchung der Stabilisierbarkeit in Bezug auf die Trajektoriengüte offen.

Um die Qualität der Trajektorien zu steigern wurde ein im Vergleich zur Vorgängerarbeit aufwendigeres Integrationsverfahren mit kleinerem Schrittfehler für die NMPC implementiert. Zukünftig wäre an dieser Stelle eine weitere Untersuchung denkbar, um zu entscheiden welche Integrationsverfahren sich für die Berechnung von Trajektorien mit Hilfe des NMPC-Verfahrens prinzipiell besonders eignen. 

Der Einfluss der Systemparameter des Motors wurde in dieser Arbeit nicht betrachtet.
Ebenso könnten die Parametervariationen daher auch für die Motorparameter durchgeführt werden.
Der Code müsste dafür nur geringfügig erweitert werden.
Somit könnte der Einfluss des Motors untersucht und unter Umständen bei zukünftigen Änderungen am System berücksichtigt werden.

%Der in dieser Arbeit erstellte Code ist allgemein für die Weiterverwendung in zukünftigen konzipiert.
Unsicherheiten