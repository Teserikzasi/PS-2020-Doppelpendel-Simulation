\chapter{Arbeitspunkt-Regelung}\label{cha:apr}

Nachdem im letzten Kapitel die Modellierung des Gesamtsystems erläutert wurde, stellt sich nun die Frage, wie sich das System regeln lässt. Dabei geht es einerseits um die Regelung an den 4 Arbeitspunkten \siehe{\secref{sec:aps}}, \dah das Stabilisieren und Halten dieser, sowie um \traj n, die das \dpd\ von einem \ap\ in einen anderen überführen (siehe folgendes \charef{cha:trj}).

Alle \ap -Regelungen vergangener Arbeiten basieren auf einer \lin\ und der anschließenden Auslegung eines \ricc-Reglers.
Die Güteparameter wurden dabei jedes Mal anders und meist heuristisch bestimmt.
Da am Versuchsstand nicht alle Zustände gemessen werden können, wurden meistens \beob\ mit Polplatzierung ausgelegt.
Kämmerer \cite{kämmerer} verglich diesen mit einem Zustandsvariablenfilter, mit welchem allerdings keine zufriedenstellende Ergebnisse gelangen.
Apprich \cite{apprich} und Kämmerer \cite{kämmerer} setzten Störgrößenbeobachter ein, die in den folgenden Arbeiten nicht mehr verwendet wurden.
Alle Arbeiten setzten Vorsteuerungsmethoden ein, da die \crb\ am Schlitten bei der Regelung stets problematisch war.
Chang \cite{chang} erstellte ein Simulationsmodell zur \ap-Regelung, das als Orientierung dient.

Beim Reglerentwurf wird zunächst dem Vorgehen vergangener Arbeiten gefolgt. 
Einzelne Aspekte in der Steuerung/Regelung werden anschließend verbessert.
Die Reglerparameter der vorigen Arbeiten werden überprüft und optimiert.
Außerdem wird der Einfluss des \beob\ anlysiert.

\begin{figure}[hb]
	\centering
		\includegraphics[width=0.4\textwidth]{Bilder/Regelung Vorgehen.pdf}
	\caption{Vorgehen bei der AP-Regelung }
	\label{fig:regvorg}
\end{figure}

\figref{fig:regvorg} stellt das prinzipielle Vorgehen bei der Untersuchung und Optimierung der \ap-Regelung dar.
Es werden zunächst Auswertungsmethoden implementiert, die einzelne Simulationstests evaluieren.
Mit den Anfangswerttests wird das Ermitteln der maximalen Winkelabweichung, ab welcher das System nicht mehr stabilisiert werden kann, automatisiert.
Dadurch können im nächsten Schritt die Güteparameter systematisch optimiert werden.
Abschließend wird die Regelbarkeit in Abhängigkeit der Systemparameter untersucht.

Regler können nach verschiedenen Gesichtspunkten wie Dynamik, Überschwingen, Energieverbrauch optimiert werden.
In dieser Arbeit wird bei der Untersuchung und Optimierung des \dpd s stets die Stabilität priorisiert.
Die Optimierung des Einschwingvorgangs steht nicht im Vordergrund.
Daher werden vor allem die instabilen \ap e 2, 3 und 4 untersucht.
Außerdem werden hauptsächlichen die maximalen Winkelabweichungen betrachtet, da diese wesentlich für die Stabiltät sind.
Horizontale Fahrten in x-Richtung sind zwar möglich, eventuell müssen dafür aber andere Reglerparameter eingesetzt werden.


\section{System-\lin\ und Analyse}%\label{sec:}

Das nicht-lineare System wird zunächst linearisiert, dessen \ewe\ analysiert, sowie auf Steuer- und Beobachtbarkeit überprüft.

\subsection{\lin}\label{subsec:lin}

Da das \zrm\ des \spds s \eqref{eq:zrm} nicht-linear ist, muss es um jeden \ap\ linearisiert werden, bevor lineare Regelungsmethoden angewandt werden können. 

\begin{align}
	\mat{A} &= \left. \frac{\partial \vef(\vex,u_R)}{\partial \vex} \right|_{\vex=\vexr}  
		= \left. \begin{bmatrix}
		0 & 1 & 0 & 0 & 0 & 0 \\
		0 & \partiald{a_2(\vex)}{\xop} & \partiald{a_2(\vex)}{\phe} & \partiald{a_2(\vex)}{\phep} & \partiald{a_2(\vex)}{\phz} & \partiald{a_2(\vex)}{\phzp} \\
		0 & 0 & 0 & 1 & 0 & 0 \\
		0 & \partiald{a_4(\vex)}{\xop} & \partiald{a_4(\vex)}{\phe} & \partiald{a_4(\vex)}{\phep} & \partiald{a_4(\vex)}{\phz} & \partiald{a_4(\vex)}{\phzp} \\
		0 & 0 & 0 & 0 & 0 & 1 \\
		0 & \partiald{a_6(\vex)}{\xop} & \partiald{a_6(\vex)}{\phe} & \partiald{a_6(\vex)}{\phep} & \partiald{a_6(\vex)}{\phz} & \partiald{a_6(\vex)}{\phzp} \\
	\end{bmatrix} \right._{\vex=\vexr}  \\
	\mat{B} &= \left. \frac{\partial \vef(\vex_R,u)}{\partial u} \right|_{u=u_R}
	= \ve{b}(\vexr) = \begin{bmatrix}
		0 \\ b_2(\vexr) \\ 0 \\  b_4(\vexr) \\ 0 \\  b_6(\vexr)
	\end{bmatrix}
\end{align}

Somit ergeben sich für jeden \ap\ unterschiedliche Systemmatrizen \mat{A} und Eingangsmatrizen \ve{B}. Beim \bss\ fällt zusätzlich die zweite Zeile und die zweite Spalte (bis auf die 1) weg, da dort $a_2=0$ ist und keine Abhängigkeit von \xop\ besteht \siehe{\tabref{tab:abh}}, außerdem ist $b_2=1$.
Die Ausgangsmatrix \mat{C} ist immer gleich, da die Ausgangsgleichung \eqref{eq:hx} linear ist.


\subsection{Eigenwerte}

Anhand der \ewe\ können Aussagen über die Dynamik eines Systems getroffen werden. 
Das \bss\ hat aufgrund der zweifachen Integration des Eingangs \xopp\ stets zwei \ewe\ in 0. 
Mit den Apprich-Parametern ergeben sich folgende \ewe:

\begin{table}[htbp]
	\centering
	\caption{\ewe\ des \bss s (Parameter: Apprich)}
		\begin{tabular}[t]{cccc}
			\toprule
			\ape & \apz & \apd & \apv \\
			\midrule
			$0$	&	$0$	&	$0$	&	$0$	\\
			$0$	&	$0$	&	$0$	&	$0$	\\
			$-0.2972 +11.4177\iu$ &    $7.8028						$	&	  $6.8596							$	&   $11.1308$	\\
			$-0.2972 -11.4177\iu$ &   $-7.9367						$	&   $-7.2314					$		&  $-11.7251$	\\
			$-0.0418 + 4.8515\iu$ &   $-0.1858 + 7.0392\iu$	&  $-0.0669 + 7.8676\iu$	&  $  4.8089$	\\
			$-0.0418 - 4.8515\iu$ &   $-0.1858 - 7.0392\iu$	& $ -0.0669 - 7.8676\iu	$	&  $ -4.8926$	\\
			\bottomrule
		\end{tabular}
	\label{tab:ewappr}
\end{table}
Im \ap\ 1 ergeben sich zwei konjugiert-komplexe Polpaare, die sich physikalisch mit der Schwingung der beiden Pendel begründen lassen. Sie befinden sich aufgrund der Dämpfung in der linken s-Halbebene, es ist der einzige stabile \ap. \ap\ 2 und 3 besitzen jeweils ein konjugiert-komplexes Polpaar, da dort das erste \bzw zweite Pendel nach unten zeigt und damit \qq{stabil} ist. Der Realteil ist bei \ap\ 3 deutlich kleiner, was auf die geringere Dämpfung des zweiten Pendelgelenks zurückzuführen ist (siehe \tabref{tab:spdparams}). Aufgrund der Instabilität des anderen Pendels ergibt sich jeweils ein positiv reeller \ew. In \ap\ 4 stehen beide Pendel oben, weswegen es dort zwei \ewe\ in der rechten s-Halbebene gibt.

\begin{table}[htbp]
	\centering
	\caption{\ewe\ des \bss s (Parameter: Ribeiro)}
		\begin{tabular}[t]{cccc}
			\toprule
			\ape & \apz & \apd & \apv \\
			\midrule
				$0$	&	$0$	&	$0$	&	$0$	\\
				$0$	&	$0$	&	$0$	&	$0$	\\
				$-174.3727$						&	$-172.9211$	&	$-173.2473$						&	$-175.4362$	\\
				$-0.3493$							&	$-0.3494$		&	$0.3471$							&	$0.3471$	\\
				$-0.6385 + 7.1578\iu$	&	$6.7715$		&	$-0.6443 + 7.2040\iu$	&	$6.7469$	\\
				$-0.6385 - 7.1578\iu$	&	$-7.6900$ 	&	$-0.6443 - 7.2040\iu$ &	$-7.6570$ \\
			\bottomrule
		\end{tabular}
	\label{tab:ewribe}
\end{table}

Mit den neuen Parametern (Ribeiro) ergibt sich ein anderes Bild (siehe \tabref{tab:ewribe}).
Bei \ap\ 1 und 2 verschwindet ein Polpaar, welches dem ersten Pendel zugeordnet wurde. Ein sehr schneller \ew\ kommt hinzu, außerdem ein langsamer reeller, der bei \ap\ 3 und 4 positiv ist. 
Die Ursache für diese Änderung liegt vermutlich in der Modellierung der \crb\ (siehe \secref{sec:crb}). Bei der \lin\ um die Ruhelage wird die Annäherungsfunktion der \mrm{signum}-Funktion um 0 linearisiert, wo sie sehr steil ist. Dies resultiert in einer sehr hohen Dämpfung. Daher ist es sinnvoll, die \crb\ der Pendelstäbe für den AP-Reglerentwurf zu vernachlässigen.
Wird die \crb\ zu 0 gesetzt, ergeben sich prinzipiell ähnliche Werte wie in \tabref{tab:ewappr}.


\subsection{Steuerbarkeit und Beobachtbarkeit}

Voraussetzung für eine Regelung ist die Steuerbarkeit des Systems. Diese kann für lineare Systeme oder linearisierte Systeme an einem \ap\ mit dem Steuerbarkeitskriterium nach \textsc{Kalman} überprüft werden \cite{AdamyRT2}. Außerdem sollte ein System beobachtbar sein, falls nicht alle Zustände bekannt sind und daher über einen Beobachter ermittelt werden.

Das \spds\ (\bss) ist an allen 4 \ap en vollständig steuer- und beobachtbar.


\section{Aufbau der Regelung}\label{sec:aufbaureg}

Die Aufgabe eines Reglers \bzw einer (Vor-)Steuerung ist es nun, anhand der Messgrößen $\vey=(\xo\ \phe\ \phz)^{\transp}$ \eqref{eq:hx} eine geeignete Steuerspannung $U_{\mrm{Steuer}}$ an den Motor vorzugeben. Dabei teilt sich die Steuerung/Regelung in mehrere Schritte auf. 

Die eigentliche Regelung bildet ein \zsr, welcher entweder am \krs\ oder am \bss\ ausgelegt ist. 
Dessen Stellgröße kann im ersten Fall (\fsoll) direkt an die Motorvorsteuerung gegeben werden.
Beim Beschleunigungssystem muss die Ausgangsstellgröße \asoll\ in einem weiteren Block zu einer Sollkraft bestimmt werden.
Diese Aufteilung in einen \ap regler am einfacheren \bss\ und eine unterlagerte Geschwindigkeitsregelung, um der störenden Reibung am Schlitten zu begegnen, hat sich in vergangenen Arbeiten als sinnvoll herausgestellt.


\subsection{Zustandsregler}\label{subsec:zsr} 

Die \aprg\ wird mit einem \zsr\ realisiert ($u=-\mat{K} \vexd$). 
Dieser regelt ein System grundsätzlich in den Ursprung, es muss also noch eine Zustandstransformation durchgeführt werden. 
Für jeden der \ap e (\ref{sec:aps}) wird der Zustand, der für den \zsr\ die Endlage darstellt, vorher von den Messgrößen abgezogen. 
Nach Berechnung der Stellgröße und Schätzung des Zustands wird der \ap\ wieder dazu addiert.

Die Verstärkung $\mat{K}$ wird durch Lösen des LQ-Problems bestimmt \cite{AdamyRT2}. Parameter für den Entwurf sind somit die positiv-definite Matrix 
$\mat{Q}=\diaga \left(\begin{matrix} q_1 & q_2 & q_3 & q_4 & q_5 & q_6 \end{matrix}\right)$
 zur Bewertung der Zustände und $R$ für die Stellgröße. 
Da es sich jeweils um ein \emph{linear-zeitinvariantes System} (LZI) handelt, ist die Matrix \mat{P} aus der \ricc-Gleichung, und damit auch \mat{K}, konstant.

Für jeden \ap\ muss ein eigener Regler entworfen werden, da jedem \ap\ ein anderes lineares System zugrunde liegt \siehe{\secref{subsec:lin}}.
Der AP-Regler ist somit nur in der Umgebung der Ruhelage gültig und kann zu große Abweichungen aufgrund der Nichtlinearität möglicherweise nicht ausregeln.
Außerdem sind für jeden \ap\ andere Güteparameter \mat{Q} und $R$ optimal. Die systematische simulative Optimierung erfolgt in \secref{sec:x0qr}.

Das Lösen der \ricc-Gleichung und die Berechnung von $\mat{K}$ für alle \ap e erfolgt automatisiert in \ml.


\subsection{Zustandsermittlung}\label{subsec:zse} 

Ein \zsr\ benötigt zur Regelung zu jedem Zeitpunkt die Information aller Zustände \vex\ \eqref{eq:vex} des Systems. Beim \spds\ werden die drei Zustände \xo, \phe\ und \phz\ gemessen. Deren Ableitungen, die Zustände \xop, \phep\ und \phzp\ müssen noch ermittelt werden. Dabei wird von drei Möglichkeiten ausgegangen, die in der Simulation miteinander verglichen werden können:
\begin{enumerate}
	\item Zustandsmessung
	\item Beobachter
	\item Differenzieren 
\end{enumerate}
Am Versuchsstand erfolgt die Zustandsermittlung durch einen \beob.

\subsubsection{\zm}
Bei der \zm\ wird von einer idealen Messung aller Zustände ausgegangen, \dah $\vexd=\vex$. 
Diese Variante wird in der Simulation meist zuerst eingesetzt, um die Regelung des Systems besser beurteilen zu können und spezielle Probleme durch die \ze\ im Nachhinein analysieren zu können.
Am realen Versuchsstand kann diese Methode nicht eingesetzt werden.

\subsubsection{\beob}\label{subsec:beob}
Der Schätzzustand $\vexd$ wird durch einen \emph{Luenberger-Beobachter} ermittelt. 
Dieser \qq{simuliert} ein lineares Schätzsystem (\mat{A}, \mat{B}, \mat{C}) und regelt Unterschiede in den Ausgangsgrößen \vey\ und \veyd\ mit der \beob-Matrix \mat{L} aus \cite{AdamyRT2}. 
Wie beim \zsr\ muss auch hier die Auslegung für jeden \ap\ separat erfolgen.
Damit gilt allerdings auch, dass der \beob\ nur in der Nähe dieser Ruhelage gültig ist, bei zu großen Abweichungen kann er aufgrund der \lin\ die Zustände nicht mehr richtig schätzen.

\mat{L} wird in dieser Arbeit mittels Polplatzierung ausgelegt. 
Einflussparameter sind somit die Pole des Beobachters $\vep_b$\,. 
Diese sollten üblicherweise weiter links liegen als die Pole des geregelten Systems $\vep_r$\,. 
Sie können aber nicht beliebig weit links liegen, da sonst Messrauschen verstärkt wird.

In \cite{brehl} werden sie beispielsweise hintereinander auf der negativen reellen Achse verteilt:
	\[
	\vep_b=\begin{bmatrix}
		-40 & -41 & -42 & -43 & -44 & -45
	\end{bmatrix}
\]
Im Allgemeinen ist es aus oben genannten Gründen sinnvoll, die \beob-Pole in Abhängigkeit der Pole des geschlossenen Regelkreises zu definieren.
Bei \cite{chang} werden sie nach
	\[
	\vep_b = \vep_r - 25
\]
bestimmt, wodurch allerdings komplexe \beob-Pole möglich sind. In dieser Arbeit werden sie daher meist folgendermaßen berechnet:
	\[
	\vep_b = \Re\left\{{\vep_r \cdot 5}\right\}
\]
%Im \ml-Code kann die Definition geändert werden und die verschiedenen Auslegungen miteinander verglichen werden.

Neben dem Messvektor \vey\ ist auch die Stellgröße $u$ Eingang des \beob. 
Dafür könnte man direkt die geforderte Stellgröße des \zsr\ $u_\mrm{reg}$ verwenden.
Allerdings wird letztere bei bestimmten Regelabweichungen häufig deutlich über der maximalen Stellgröße liegen, sodass die Ausgangsstellgröße spätestens beim Motor begrenzt wird.
Wenn der Beobachter dies nicht mitbekommt und daher sein System mit der unbegrenzten Regler-Stellgröße simuliert, gibt es zwangsläufig Abweichungen zwischen \beob-System und realem System.
Der Zustand wird nicht mehr richtig geschätzt, folglich ergibt sich ein schlechteres Regelverhalten, bis hin zur Instabilität.

Aus diesen Gründen ist es sinnvoll, eine Vorsteuerung zu implementieren \siehe{\secref{sec:motvorst}}, die die Stellgröße schon vor der Ausgabe an den Motor begrenzt und die tatsächliche Stellgröße zurück an den \beob\ gibt, sodass dieser das reale System besser nachbilden kann.
Somit weicht die Stellgröße nur noch bei Stellgrößeneinbrüchen anderer Art \siehe{\secref{subsec:dcMotor}} ab.


\subsubsection{\diff}
Die übrigen Zustände werden durch Differentiation der Messgrößen und \evtl Tiefpass-Filterung gebildet:
\begin{align}
	\vexd=\begin{bmatrix}
		1	&	0	&	0	\\
		\ddt	&	0	&	0	\\
		0	&	1	&	0	\\
		0	&	\ddt	&	0	\\
		0	&	0	&	1	\\
		0	&	0	&	\ddt	\\
	\end{bmatrix} \vey  = \begin{bmatrix}
		\xom \\ \ddt \xom \\ \phem \\ \ddt \phem \\ \phzm \\ \ddt \phzm
	\end{bmatrix}
\end{align}
In der Simulation kam es hierbei zu Problemen, da für eine möglichst glatte und fehlerfreie Differentiation eine kleine Schrittweite notwendig ist, außerdem hat sich die Simulation in manchen Fällen aufgehängt. Daher wird diese Variante in dieser Arbeit nicht weiter verfolgt.

Trotzdem sei angemerkt, dass diese Möglichkeit am realen Versuchsstand in Kombination mit einer sinnvollen Filterung der Messgrößen und den Ableitungen möglicherweise Vorteile gegenüber einem Beobachter darstellt. 
Es ist kein Einschwingen der Startzustände nötig, es gibt keine Probleme bei Stellgrößenstörungen und die Differentiation muss nicht für jeden \ap\ einzeln ausgelegt werden.


\subsection{Vorsteuerung F/a und a/v-Regler}

Ist der AP-Regler nach dem \bss\ ausgelegt, gibt dieser als Stellgröße eine Sollbeschleunigung \asoll\ für den Schlitten vor. 
Aus dieser muss die geforderte Motorkraft bestimmt werden.
Sie kann zum Großteil mit einer Vorsteuerung bestimmt werden, die Differenz wird mit einem Geschwindigkeitsvergleich ausgeregelt.


\subsubsection{a/v-Regler}
Die Sollgeschwindigkeit \vsoll\ wird durch Integration von \asoll\ berechnet.
Dadurch kann ein Soll/Ist-Vergleich mit der realen \bzw geschätzten Schlittengeschwindigkeit \xop\ durchgeführt werden.
Die Regelung wird durch einen P-Regler umgesetzt:
	\[
	\fav= \kpv (\vsoll-\xopd)
\]
Brehl \cite{brehl} ging für \kpv\ beispielsweise von 500 aus, während im \sm-Modell von \cite{chang} 150 verwendet wird.
Der Parameter scheint am realen Modell einen starken Einfluss auf die Stabilität und ein mögliches \qq{Ratterverhalten} zu haben.
In dieser Arbeit wird ein Wert von 150 angenommen und der Parameter nicht weiter untersucht, da es in der Simulation meist nur geringe Abweichungen von der Sollgeschwindigkeit gibt.

Als problematischer stellte sich hingegen ein möglicher \emph{Windup-Effekt} heraus: 
Ist die Sollbeschleunigung größer als wegen der \sgb\ tatsächlich erreicht werden kann, wird \vsoll\ immer weiter aufintegriert und damit auch \fav, obwohl die Motorkraft bereits maximal ist. 
Wenn das System wieder aus der \sgb\ ist und die Vorsteuerungskraft eigentlich das Vorzeichen wechselt, ist der I-Anteil noch vorhanden und muss erst abgebaut werden.
In dieser Zeit übersteigt \fav\ daher \fvorst\ und die Stellgröße geht somit praktisch in die falsche Richtung, was die Regelung sehr schnell instabil werden lassen kann.
Aus diesem Grund wird \asoll\ vor der Integration mit einem Sättigungsglied auf 
	\[
	a_\mrm{soll,max}=\frac{\fmax}{m}
\] 
begrenzt.

\subsubsection{\vorst\  (M,C,D)}\label{sec:vorstmcd}
Der \avr\ alleine könnte die Schlittenkraft zwar regeln, allerdings kann der Hauptteil der Kraft mit einer Vorsteuerung bestimmt werden, wodurch die Dynamik besser ist.
Die Vorsteuerung besteht im Wesentlichen aus der Trägheitskraft, da bei reiner Betrachtung des Schlittens $\fsoll=m_0 \asoll$ gilt. 
Wenn davon ausgegangen wird, dass die Reibungsparameter hinreichend genau bestimmt wurden, kann auch die Reibung des Schlittens kompensiert werden.
Während \cite{brehl} nur die Trägheitskraft vorsteuerte, wird in \cite{chang} die Vorsteuerung folgendermaßen berechnet:
	\[
	\fvorst=m \asoll + \Fco \, \sign{\vsoll} + d_0 \xop
\]

\subsubsection{\vorst\  mit Systemgleichung}
Auch die obige Gleichung bestimmt die benötigte Kraft nicht exakt, da es je nach Winkellage der Pendel rückwirkende Kräfte auf den Schlitten gibt.
Man kann daher einen Schritt weiter gehen und die benötigte Kraft zur Beschleunigung des Schlittens direkt aus der entsprechenden Bewegungsgleichung herleiten. 
Auch wenn am realen Versuchsstand die Beschleunigung aufgrund von Ungenauigkeiten nicht exakt mit \asoll\ übereinstimmt, wird dem \avr\ dadurch weitere \qq{Arbeit abgenommen}.

Die Beziehung zwischen der Schlittenbeschleunigung und der Kraft am Schlitten wird durch die zweite Zustandsgleichung \eqref{eq:zrmF} beschrieben:
\begin{align}
	a=\xopp=f_2(\vex,F)
	\label{eq:xpp}
\end{align}
Diese Gleichung kann leicht nach der Kraft umgestellt werden:
\begin{align}
	\fsoll=\frac{\asoll-a_2(\vex)}{b_2(\vex)}
\end{align}
Damit diese Berechnung für beliebige Systemparameter und somit unterschiedliche Funktionen $a_2$ und $b_2$ erfolgen kann, wird die Bestimmung der Vorsteuerkraft in dieser Arbeit symbolisch in \ml\ automatisiert.

\subsubsection{Ermittlung reale Beschleunigung}
In \secref{subsec:beob} wurde erläutert, dass es sinnvoll ist, die reale Stellgröße an den \beob\ zu geben.
Im folgenden Vorsteuerungsblock (\secref{sec:motvorst}) wird die vor dem Motor begrenzte Kraft \freal\ zurückgegeben, da die \sgb\ \fmax\ bekannt ist.
Ist der \beob\ allerdings am \bss\ ausgelegt, geht er bei der Stellgröße von der Beschleunigung aus.
Somit muss aus \freal\ zunächst \areal\ bestimmt werden.

Dies kann ähnlich wie bei der Kraftvorsteuerung durch die Trägheit geschehen, was allerdings ungenau ist und der \beob\ demzufolge von einer falschen Eingangsgröße ausgeht.
Daher wird die Ermittlung der tatsächlichen Beschleunigung wie im vorigen Abschnitt über die Systemgleichung vorgenommen.
Es gilt \eqref{eq:xpp}.



\subsection{Motor Vorsteuerung}\label{sec:motvorst}

Bisher wurde für die Regelung die Ermittlung der Kraft, die am Schlitten wirken soll, betrachtet.
Die Schnittstelle zum Motor besteht durch Vorgabe der Steuerspannung.
Die ausgegebene Spannung stellt daher letztendlich die eigentliche Stellgröße dar und muss in Abhängigkeit der Sollkraft bestimmt werden.

Auch wenn das Motorsystem für die Simulation genauer modelliert wird \siehe{\secref{sec:mot}}, handelt es sich (abgesehen von einer sehr schnellen Tiefpass-Dynamik) im Wesentlichen um ein reines Verstärkungsglied, das mit der Konstante $K_\mrm{MotorGain}$ \eqref{eq:motgain} beschrieben wird. % \siehe{\secref{subsec:motorparams}}.
Die \vorst\ rechnet daher mit der Konstante die Sollkraft in die Sollspannung um.
	\[
	\usoll=\frac{\fsoll}{K_\mrm{MotorGain}}
\]

In der Motor Vorsteuerung wird auch die Stellgrößenbegrenzung berücksichtigt. 
Wie schon in \secref{subsec:beob} erläutert wurde, wird die Motorspannung vor der Ausgabe begrenzt.
Die tatsächliche Stellgröße kann dadurch an den \beob\ gegeben werden.
%maximale Kraft \siehe{\secref{subsec:paramshist}}




\section{Simulationsmodell in \Simulink}

Die \sm-Modelle und \ml-Funktionen zur \ap-Regelung befinden sich im Ordner \texttt{Regelung}.

\begin{figure}
	\centering
		\includegraphics[width=1.00\textwidth]{Bilder/Simulink/ap_regelung_test ohne outofc.PNG}
	\caption{Aufbau der \ap-Regelung}
	\label{fig:simapr}
\end{figure}

Die \ap-Regelung wird wie in \secref{sec:aufbaureg} beschrieben als \sm-Modell implementiert \siehe{\figref{fig:simapr}}.
Das Gesamtmodell-Modul \siehe{\secref{sec:gesamtslx}} wird hier eingebunden und stellt die Regelstrecke dar.
Davor befinden sich drei Blöcke der Regelung/Vorsteuerung.
Alternativ kann auch zum \ap-Regler am Kraftsystem umgeschaltet werden.

\subsubsection{APRegler.slx}
Ist eine Art \qq{Wrapper} für den eigentlich \zsr, um die \ap-Transformation durchzuführen.
Erwartet als Parameter die Daten zum \ap.

\subsubsection{Zustandsregler.slx}
Implementiert einen gewöhnlichen \zsr, bei dem alle Arten der \ze\ \siehe{\secref{subsec:zse}} möglich sind \siehe{\figref{fig:simzsr}}.
Die Auswahl wird durch ein \emph{Variant Subsystem} realisiert, das durch die Variable \texttt{simparams.Zustandsermittlung} angesteuert wird \siehe{\figref{fig:simzse}}.
Die Stellgröße für den \beob\ kann zwischen intern und extern umgeschaltet werden.
Benötigt als Parameter die Verstärkung \texttt{K} sowie eventuelle Daten für den \beob.

\begin{figure}
	\centering
		\includegraphics[width=0.8\textwidth]{Bilder/Simulink/zsr.PNG}
	\caption{\zsr\ in \sm}
	\label{fig:simzsr}
\end{figure}

\begin{figure}
	\centering
		\includegraphics[width=0.4\textwidth]{Bilder/Simulink/zustandsermittlung.PNG}
	\caption{\ze\ in \sm}
	\label{fig:simzse}
\end{figure}

\subsubsection{Beobachter.slx}
Implementiert einen Standard-\beob. 
Eingänge sind \texttt{u} und \texttt{y}, Ausgang sind die geschätzten Zustände \texttt{x\_est}.
Die Matrizen des linearen Systems, sowie \texttt{L} und die \beob startwerte werden über die Maske übergeben.

\subsubsection{MotorVorsteuerung.slx}
Enthält die in \secref{sec:motvorst} beschriebene Vorsteuerung zum Motor.

\subsubsection{Vorsteuerung\_Fa\_avRegler.slx}


\begin{figure}
	\centering
		\includegraphics[width=0.7\textwidth]{Bilder/Simulink/Fa_vorst.PNG}
	\caption{Aufbau der Vorsteuerung}
	\label{fig:simfav}
\end{figure}

\subsubsection{SchlittenVorsteuerung.slx}
Enthält die \qq{alte} Variante der Vorsteuerung \siehe{\secref{sec:vorstmcd}}.
Dabei werden die Schlittenparameter ($m_0$, \Fco\ und $d_0$) verwendet. 

\subsubsection{SchlittenGleichungKraft.slx und SchlittenGleichungBeschleunigung.slx}


\subsection{\init}



\subsection{Weitere \Matlab-Funktionen}

\subsubsection{SimAP}


\section{Anfangswert-Tests}\label{sec:x0test}

Ziel dieser Arbeit ist die "`Regelbarkeit"' des Systems zu untersuchen.
Dafür müssen zunächst Kriterien definiert werden, inwiefern das System unter festgelegten System- und Entwurfsparametern regelbar ist.

Dazu wird wieder \figref{fig:regvorg} herangezogen.
Ist ein System fest ausgelegt, können die verschiedenen \ap e mit bestimmten Startwerten getestet werden.
Dabei gibt es jedoch unzählige Kombinationsmöglichkeiten, die nicht alle getestet werden können.
Es muss daher eine Auswahl an Startwerten getroffen werden (beispielsweise wird immer nur ein Zustand gesetzt, alle anderen sind am \ap).
Selbst dann ist es mit manuellem Testen und Ausprobieren aufwändig, ein umfassendes Bild vom Regelverhalten zu erlangen.
Die Optimierung der "`äußeren Schleifen"' wird somit langwierig und auf diese Weise nicht zielführend.

Daher wird die "`untere Schleife"' automatisiert.
Wird ein \ap-Test durchgeführt, können die Simulationsdaten ausgewertet werden und gewisse Gütemaße und Performance-Indikatoren bestimmt werden, um die Regelgüte zu beschreiben.
Im Vordergrund steht aber meist die Stabilität und damit die Frage, bei welchen Werten diese nicht mehr vorhanden ist.


\subsection{Auswertung eines \ap-Tests}

Wenn \texttt{SimAP} aufgerufen wird, werden neben den Simulationsdaten auch die Ergebnisse von \texttt{Auswertung/AP/APAuswertung} zurückgegeben.
Dort werden \ua folgende Gütemaße, die an das übliche quadratische \ricc-Gütemaß angelehnt sind, berechnet:
\begin{subequations} \begin{align} 
	J_x &= \int_{0}^{\tend} \dvex^\transp(t) \, \mat{Q} \, \dvex(t)  \, \ud t = \int_{0}^{\tend} \Delta x_\mrm{norm}(t)  \, \ud t  \\
	J_{x,\mrm{est}} &= \int_{0}^{\tend} \dvexest^\transp(t)  \,\mat{Q}  \,\dvexest(t)  \, \ud t  \\
	J_u &= \int_{0}^{\tend}  R \cdot \Delta u(t)^2 \, \ud t	\\
	J_F &= \int_{0}^{\tend}  R \cdot F(t)^2 \, \ud t
\end{align} \end{subequations}
mit
\begin{align*}
	\mat{Q} &= \diag{25, 1, 50, 1, 50, 1 } \\
	R &= 1 
\end{align*}
In \ml\ werden sie als Summe berechnet (der Einfachheit halber konstante Schrittweite vorausgesetzt).

Für die Beurteilung der Stabilität, also ob der Zustand am Ende der Simulationszeit noch in der Nähe des \ap es ist, wird
$\Delta x_\mrm{norm}(\tend)$
verwendet. Ist dieser skalare Wert unter einer festgelegten Schranke, wird von einer Stabilisierung des \ap es ausgegangen.

Außerdem wird für jeden Zustand die Maximalabweichung vom \ap\ bestimmt, da diese ein wichtiges Gütekriterium ist, insbesondere die Schlittenposition, da diese Beschränkungen unterliegt \siehe{\secref{sec:schlbes}}.

Des Weiteren kann anhand von $\Delta x_\mrm{norm}(t)$ eine Einschwingzeit bestimmt werden.
Diese ist dort, wo der Norm-Verlauf eine gewisse Schranke nicht mehr überschreitet.


\subsection{x0-Test}

Die Funktion \texttt{Regelung/x0\_Tests/x0\_Test.m} stellt die Basis für einen Anfangswerttest dar.
Für den angegebenen \ap\ ermittelt sie mit einzelnen Simulationen und steigenden Anfangsauslenkungen vom \ap\ schrittweise die maximalen Startwerte.
Dabei wird mit obigen Methoden ermittelt, ob die Stabilisierung erfolgreich war.
Falls nicht, wird abgebrochen und der Maximalwert sowie einige der Gütemaße für allen Einzelergebnisse zurückgegeben.

Die Schrittweite \texttt{y\_st} = [ \texttt{x\_st} \texttt{phi1\_st} \texttt{phi2\_st} ] gibt die Schrittweite an, mit der die Tests ausgeführt werden.
Eine kleinere Schrittweite löst zwar die Ergebnisse besser auf, führt aber zu einer höheren Rechenzeit.

Die Ergebnisse eines Anfangswerttests an \apv\ sind beispielhaft in \figref{fig:x0test} dargestellt.
Auf der Abszisse wird der Startwert der Auslenkung (hier \phz) aufgetragen.
In Abhängigkeit von dieser werden die maximalen Abweichungen der drei Ausgänge vom \ap\ dargestellt.
In diesem Fall kann das System bis zu einem Startwert von $\phzo=\valdeg{19}$ stabilisiert werden.
Die Startauslenkung wird auch als schwarze Linie eingezeichnet. 
Die Maximalabweichung ist stets oberhalb oder auf dieser.

\begin{figure}[htbp]
	\centering
		\includegraphics[width=0.7\textwidth]{Bilder/x0test/appr-x0-ap42.pdf}
	\caption{Anfangswerttest \apv, Auslenkung \phz}
	\label{fig:x0test}
\end{figure}


\subsection{Kritische Anfangswert-Tests}

Wenn immer nur ein Zustand ausgelenkt wird, ergeben sich pro \ap\ 3 Tests, also insgesamt 9 Tests (unter Vernachlässigung des stabilen \ape).
Es hat sich dabei gezeigt, dass Abweichungen von \xo\ kein Problem bezüglich der Stabilität darstellen.
Außerdem sind bei \ap\ 2 und 3 Auslenkungen des jeweils "`stabilen"' Pendels unkritisch.

Aus diesem Grund sind für die Untersuchung der Stabilität nur die folgenden vier (kritischen) Anfangswert-Tests relevant:
\begin{itemize}
	\item \ap\ 2, Auslenkung \phz
	\item \ap\ 3, Auslenkung \phe
	\item \ap\ 4, Auslenkung \phe
	\item \ap\ 4, Auslenkung \phz
\end{itemize}
In \texttt{x0\_Tests/x0\_Test\_APs.m} werden diese vier Tests durchgeführt.


\newcommand{\scalee}{0.48}

\begin{figure}
	\centering
	\subfloat[\apaz]{ \includegraphics[scale=\scalee]{Bilder/Parameter neu (Ribeiro) Creg off/AP2.pdf} }
	\hfil
	\subfloat[\apad]{	\includegraphics[scale=\scalee]{Bilder/Parameter neu (Ribeiro) Creg off/AP3.pdf} }
	\\
	\subfloat[\apave]{ \includegraphics[scale=\scalee]{Bilder/Parameter neu (Ribeiro) Creg off/AP41.pdf} }
	\hfil
	\subfloat[\apavz]{ \includegraphics[scale=\scalee]{Bilder/Parameter neu (Ribeiro) Creg off/AP42.pdf} }
	\caption{Maximalabweichungen -- Vergleich QR-Parameter (System Ribeiro)}
	\label{fig:qrvglrib}
%\vspace{15pt}
\end{figure}




\section{QR Parameter Tests}\label{sec:x0qr}




%Bei \cite{chang} wurde eine Zufallssuche zum Finden von geeigneten Reglerparametern durchgeführt.
%Dort wurde aber lediglich eine feste Anfangsauslenkung von \phz\ getestet und bei der Auswertung max_x, Endwert der Winkel

\section{System Parameter Tests}\label{sec:x0sys}

System Ribeiro

\newcommand{\scaleq}{0.62}
\begin{figure}
	\centering
	\subfloat[\apaz]{ \includegraphics[scale=\scaleq]{Bilder/SysParam Variation/m1/AP2.pdf}	}
	\hfil
	\subfloat[\apad]{	\includegraphics[scale=\scaleq]{Bilder/SysParam Variation/m1/AP3.pdf}	}
	\\
	\subfloat[\apave]{ \includegraphics[scale=\scaleq]{Bilder/SysParam Variation/m1/AP41.pdf} }
	\hfil
	\subfloat[\apavz]{ \includegraphics[scale=\scaleq]{Bilder/SysParam Variation/m1/AP42.pdf}	}
	\caption{Maximale Startwerte -- Variation $m_1$}
	\label{fig:sysvarm1}
\end{figure}

\begin{figure}
	\centering
	\subfloat[\apaz]{ \includegraphics[scale=\scaleq]{Bilder/SysParam Variation/m2/AP2.pdf}	}
	\hfil
	\subfloat[\apad]{	\includegraphics[scale=\scaleq]{Bilder/SysParam Variation/m2/AP3.pdf}	}
	\\
	\subfloat[\apave]{ \includegraphics[scale=\scaleq]{Bilder/SysParam Variation/m2/AP41.pdf} }
	\hfil
	\subfloat[\apavz]{ \includegraphics[scale=\scaleq]{Bilder/SysParam Variation/m2/AP42.pdf}	}
	\caption{Maximale Startwerte -- Variation $m_2$}
	\label{fig:sysvarm2}
\end{figure}

\renewcommand{\scaleq}{0.55}

\begin{figure}
	\centering
	\subfloat[\apaz]{ \includegraphics[scale=\scaleq]{Bilder/SysParam Variation/J1/AP2.pdf}	}
	\hfil
	\subfloat[\apad]{	\includegraphics[scale=\scaleq]{Bilder/SysParam Variation/J1/AP3.pdf}	}
	\\
	\subfloat[\apave]{ \includegraphics[scale=\scaleq]{Bilder/SysParam Variation/J1/AP41.pdf} }
	\hfil
	\subfloat[\apavz]{ \includegraphics[scale=\scaleq]{Bilder/SysParam Variation/J1/AP42.pdf}	}
	\caption{Maximale Startwerte -- Variation $J_1$}
	\label{fig:sysvarJ1}
\end{figure}

\begin{figure}
	\centering
	\subfloat[\apaz]{ \includegraphics[scale=\scaleq]{Bilder/SysParam Variation/J2/AP2.pdf}	}
	\hfil
	\subfloat[\apad]{	\includegraphics[scale=\scaleq]{Bilder/SysParam Variation/J2/AP3.pdf}	}
	\\
	\subfloat[\apave]{ \includegraphics[scale=\scaleq]{Bilder/SysParam Variation/J2/AP41.pdf} }
	\hfil
	\subfloat[\apavz]{ \includegraphics[scale=\scaleq]{Bilder/SysParam Variation/J2/AP42.pdf}	}
	\caption{Maximale Startwerte -- Variation $J_2$}
	\label{fig:sysvarJ2}
\end{figure}

