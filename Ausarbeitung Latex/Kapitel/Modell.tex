\chapter{Modellierung}\label{cha:modell}

In diesem Kapitel wird die Modellierung des Gesamtsystems erläutert und auf dessen Implementierung in \Simulink\ eingegangen.

\section{Modell des \spd-Systems}

Die Modellierung des \spd-Systems orientiert sich zunächst an den Modellen der vergangenen Arbeiten. Diese bezogen sich meist auf die Herleitung von \cite{modpen}. Dabei gibt es die Variante \emph{Kraftsystem}, das als Eingang die Kraft annimmt, welche am Schlitten wirkt, sowie das vereinfachte \emph{Beschleunigungssystem}, das direkt die Beschleunigung des Schlittens als Eingang erhält.

\subsection{Koordinaten}

%\begin{figure}[bp]
	%\centering
		%\includegraphics[width=0.7\textwidth]{Bilder/intro.pdf}
	%\caption{Idee des zentralen Reglerentwurfs}
	%\label{fig:intro}
%\end{figure}

Die Minimal-Koordinaten sind 
\begin{align*}
	q_0 = x_0  \\
	q_1 = \varphi_1  \\
	q_2 = \varphi_2
\end{align*}

\subsection{Herleitung der Bewegungsgleichungen}

Um auf die Bewegungsgleichungen des Systems zu gelangen, wird in \cite{modpen} der \emph{Lagrange}-Formalismus verwendet.

\subsection{Ruhelagen}



\subsection{\crb}

Da die \crb sowohl des Schlittens als auch der Pendelstäbe einen wesentlichen Einfluss zu haben scheint, darf diese nicht vernachlässigt werden. In den bisherigen Modellierungen wurde höchstens die \crb des Schlittens berücksichtigt. Da jedoch durch die Neukonstruktion des \dpd s die Messsignalübertragung (zur Vermeidung einer Kabelaufwickelung) über einen Schleifring realisiert wurde, besteht die Vermutung, dass dieser für eine erhöhte \crb verantwortlich ist. Dies würde das System bereits um einen sehr kleinen Arbeitsbereich nicht-linear machen, was die Regelung erschwert.

Im vorigen Projektseminar \cite{ribeiro} wurde die Reibung der Pendelstäbe mittels Identifikation ermittelt, aufgeteilt auf den viskosen und den Coulombanteil.

Die Formel der Gleitreibung lautet eigentlich 
	\[
	F_c = F_{c0}  \cdot  \sign{\dot{x}} ,
\]
allerdings führt diese Implementierung aufgrund der signum-Funktion zu Komplikationen in der Simulation. Daher wird der Verlauf bei sehr niedrigen Geschwindigkeiten mit der $\tanh$ Funktion angenähert.


\section{Motor-Modell}\label{sec:mot}

Das Motormodell besteht aus den drei Baugruppen Spannungs-Strom-Wandler, Gleichstrommotor und Getriebe. Die Modellierung der Baugruppen basiert auf den in Franke \cite{franke} erstmalig aufgestellten Gleichungen, die auch in den nachfolgenden Arbeiten zum Versuchsstand zur Anwendung kamen. Das Modell des Gleichstrommotors wird im Rahmen dieser Arbeit nun zusätzlich um die Berücksichtigung der Gegeninduktion des Motors erweitert.

\subsection{Spannungs-Strom-Wandler}

Bei dem am Versuchsstand eingesetzten Spannungs-Strom-Wandler handelt es sich um einen Servoverstärker, der ursprünglich zur Drehzahlregelung von Gleichstrommotoren vorgesehen war. Entsprechend folgt der Verstärker dem Prinzip einer übergeordneten Drehzahlregelung mit unterlagerter Stromregelung. Für die Anwendung am Versuchsstand ist der übergeordnete Drehzahlregelkreis jedoch aufgetrennt und in einen Spannungs-Strom-Wandler umfunktioniert worden. Dieser dient zur Vorgabe eines konstanten Ankerstroms mittels Pulsweitenmodulation (PWM) der Zwischenkreisspannung des Wandlers proportional zur eingehenden Steuerspannung. Das stationäre Verhalten kann daher durch einen Proportionalitätsfaktor $K_{UI}$ beschrieben werden.
\begin{align}
	I_a = K_{UI} \cdot U_{\mrm{Steuer}}
\end{align}

Gemäß Franke \cite{franke} lässt sich die Dynamik des Wandlers durch ein \mrm{PT_1}-Glied modellieren, sodass sich für den Wandler im Laplace-Bereich die Gleichung
\begin{align} \label{eq:UI}
	I_a(s) = \frac{K_{UI}}{1+T_{UI} \cdot s} \cdot U_{\mrm{Steuer}}(s)
\end{align}
ergibt.


\subsection{Gleichstrommotor}\label{subsec:dcMotor}

Bei dem verwendeten Motor handelt es sich um eine fremderregte Gleichstrommaschine, wobei die magnetische Erregung durch einen Permanentmagneten erzeugt wird \cite{franke}. Das elektromagnetische Drehmoment des Gleichstrommotors ist näherungsweise proportional zum Ankerstrom \cite{binder}. Hierbei wird vorausgesetzt, dass keine magnetische Sättigung vorliegt.
\begin{align}
	M_e = K_I \cdot I_a
\end{align}

Neben dem elektromagnetischen Drehmoment sind außerdem parasitäre Reibmomente zu berücksichtigen. Die Modellierung der Reibung von Motor und Getriebe wird zusammen mit der Schlittenreibung in \secref{sec:spdModell} behandelt. Ein rückwirkendes Moment durch die Federkraft des Riemens wird auf Grund der Annahme unendlicher Riemensteifigkeit vernachlässigt. 

Weiterhin wird das Drehmoment durch die Gegeninduktion geschwächt. Dieser Effekt ist in den Motormodellen von Franke \cite{franke} und den Nachfolgearbeiten bisher nicht berücksichtigt worden. Da die Erfahrung am realen Versuchsstand jedoch gezeigt hat, dass eine Modellierung der Gegeninduktion sinnvoll erscheint, wird diese im Rahmen dieser Arbeit in das Motormodell integriert.

Zum besseren Verständnis des Effekts wird zunächst das physikalische Prinzip der Gegeninduktion betrachtet. Fließt ein Strom durch den ruhenden Anker, der sich im Magnetfeld der Permanentmagneten befindet, so wirkt senkrecht zu den Richtungen des Stroms und des Magnetfelds die Lorenzkraft auf die in der Ankerwicklung befindlichen Ladungsträger (Drei-Finger-Regel bzw. Rechte-Hand-Regel in technischer Stromrichtung). Durch das entstehende Drehmoment beginnt der Anker zu rotieren. Auf Grund der Rotation bewegen sich die Ladungsträger nun zusätzlich zur eigentlichen Stromrichtung auch in Rotationsrichtung des Ankers. Auf diese Bewegungskomponente kann nun erneut das Prinzip der Lorentzkraft angewendet werden. Die resultierende zusätzliche Kraftkomponente, die in Abhängigkeit der Rotationsgeschwindigkeit auf die Ladungsträger wirkt, zeigt nun gegen die Stromrichtung (Lenz'sche Regel). Der resultierenden Ladungsbeschleunigung kann ein Spannungsabfall über der Ankerwicklung zugeordnet werden, die sogenannte induzierte Gegenspannung oder Gegen-EMK (Gegen-Elektromagnetische-Kraft). Der in Folge sinkende Ankerstrom reduziert das Drehmoment des Motors. \cite{binder}

Am Versuchsstand wird dieser Effekt bei geringen Winkelgeschwindigkeiten durch den Stromregler des Spannungs-Strom-Wandlers kompensiert. Ab einer bestimmten Winkelgeschwindigkeit bei konstantem Sollstrom wird die induzierte Gegenspannung größer als die maximale Zwischenkreisspannung des Spannungs-Strom-Wandlers, sodass nicht mehr ausgeregelt werden kann. Nun nimmt der Ankerstrom und damit das Drehmoment mit steigender Winkelgeschwindigkeit ab bis die Leerlaufdrehzahl erreicht ist. 

\begin{figure}[htbp]
	\centering
		\includegraphics[width=0.50\textwidth]{Bilder/Motor/Stromkennlinie.pdf}
	\caption{Stromkennlinie}
	\label{fig:Stromkennlinie}
\end{figure}

Auf Grund der integrierten Strombegrenzung zum Schutz von Motor und Verstärker wird der vom Wandler bereitstellbare Strom zusätzlich begrenzt. Der dadurch parametrierte Maximalstrom des Wandlers wird durch den Stromregler zunächst konstant gehalten bis die Strombegrenzung der Gegen-EMK ab einer bestimmten Winkelgeschwindigkeit $\omega_\mrm{EMK}$ die des Wandler unterschreitet (siehe Abbildung \ref{fig:Stromkennlinie}). An dieser Stelle tritt ein "`Knick"' im Stromverlauf auf. Für den Stromregler wird dabei vereinfachend angenommen, über ausreichend hohe Dynamik und Stellenergie zu verfügen, um den Strom bis zur EMK-Grenze zu jedem Zeitpunkt konstant halten zu können.

Es werden zunächst nur positive Winkelgeschwindigkeiten betrachtet. 
Der Abschnitt des Verlaufs in Abbildung \ref{fig:Stromkennlinie} mit konstantem maximalen Strom wird durch
\begin{equation}
	I_a  = I_{w\mrm{, max}} = \mrm{const.}, 0 \leq \omega \leq \omega_{\mrm{EMK}}
	\label{eq:Iconst}
\end{equation}
beschrieben, wobei $I_{w\mrm{, max}}$ der Begrenzungsstrom des Wandlers ist.

\begin{figure}[h]
	\centering
		\begin{circuitikz}[european]
	\draw (0,0) 
		to[short, -, i=$I_a$] (1,0)
		to[R=$R$] (3.5,0)
		to[L=$L$] (6,0)
		to[V=$U_i$] (6,-2)
		-- (0,-2);
	\draw
		node[ocirc] (A) at (0,0) {}
		node[ocirc] (B) at (0,-2) {};
		%(A) to[open, v=$U_a$] (B);	
		\begin{scope}[shorten >= 10pt,shorten <= 10pt]
			\draw[->] (A) -- node[left] {$U_a$} (B);
		\end{scope}  
\end{circuitikz}
	\caption{Ersatzschaltbild eines Gleichstrommotors}
	\label{fig:dcESB}
\end{figure}

Der Verlauf des Bereichs mit zeitlich abnehmender maximaler Stromstärke wird aus dem Ersatzschaltbild des Gleichstrommotors in Abbildung \ref{fig:dcESB} hergeleitet. Aus dem zweiten Kirchhoff'schen Gesetz (Maschenregel) ergibt sich
\begin{equation}
	U_a =  RI_a + L \normd{I_a(t)}{t} + U_i \ . 
	\label{eq:dcMasche}
\end{equation}

Für die Kennlinie des maximalen Stroms liegt der Tastgrad (engl.: \textit{duty cylcle}) der pulsweitenmodulierten Ankerspannung bei $100\%$, sodass 
\[
	U_{a \mrm{, max}}  = const.
\]
gilt. Die elektrische Zeitkonstante des Motors, die mit den Angaben des Datenblatts aus Franke \cite{franke} zu 
\[
	\tau_{\mrm{el}} = \frac{L}{R} \approx 0,0023 \mrm{s} \ll 1 \mrm{s} \ ,
\]
berechnet werden kann, ist ausreichend klein, dass näherungsweise von stationärem Betrieb ausgegangen werden kann und für die selbstinduzierte Spannung 
\[
	L \normd{I_a(t)}{t} \approx 0
\]
gilt. Die induzierte Gegenspannung wird gemäß \cite{binder} durch Proportionalität zur Winkelgeschwindigkeit
\[
	U_i  = K_I \cdot \omega \ 
\]
modelliert.
Damit kann Gleichung \eqref{eq:dcMasche} zur Geraden
\begin{equation}
	I_{a \mrm{, max}}(\omega) = \frac{U_a}{R} - \frac{K_I}{R} \omega \ , \qquad \omega > \omega_{\mrm{EMK}} 
	\label{eq:Ivar}
\end{equation}
umgeformt werden.
Die Beschreibung des Bereichs negativer Winkelgeschwindigkeiten kann aus den Gleichungen \eqref{eq:Iconst} und \eqref{eq:Ivar} durch eine Punktspiegelung des in \figref{fig:Stromkennlinie} dargestellten Verlaufs in den dritten Quadranten gewonnen werden.

Zusammenfassend ergibt sich für den Verlauf des maximal verfügbaren Stroms  
\[
I_{a \mrm{, max}}(\omega) = \left\{\begin{array}{lll}
														-\frac{U_a}{R} - \frac{K_I}{R} \omega, & \omega < -\omega_{\mrm{EMK}}   \\ \\
														I_{w\mrm{, max}} \cdot \sign{\omega} 	& -\omega_{\mrm{EMK}} \leq \omega \leq \omega_{\mrm{EMK}} \\ \\				
														\frac{U_a}{R} - \frac{K_I}{R} \omega, & \omega > \omega_{\mrm{EMK}} \end{array}\right. . \
\] 



\subsection{Getriebe}

%\begin{figure}[h]
	%\centering
		%\includegraphics[width=0.5\textwidth]{Bilder/Motor/Getriebe.PNG}
	%\caption{Getriebe \cite{franke}}
	%\label{fig:Getriebe}
%\end{figure}

Die Rotorbewegung des Motors wird über das Getriebe und das Antriebszahnrad auf den Zahnriemen weitergegeben. Auf Grund der hohen Steifigkeit des mit eingebetteten Stahlseilen unterstützten Riemens wird dieser wie in Apprich \cite{apprich} als unendlich starr angenommen, sodass die Kinematik zwischen der Winkelgeschwindigkeit des Motors und der Schlittengeschwindigkeit ohne Federkopplung durch 
\[
	\xop = K_G \cdot r_{32} \cdot \omega
\]
modelliert werden kann.
 Hierbei wird die Getriebeübersetzung $K_G$ über das Zahnverhältnis 
\[
	K_G =  \frac{Z_{16}}{Z_{60}}
\]
berechnet, während der Antriebsradius $r_{32}$ als Abstand zwischen der neutralen Faser des Riemens und der Drehachse definiert ist.
Die Kraft 




\section{Modellparameter}\label{sec:mparams}

Bevor in den Abschnitten \ref{subsec:motorparams} und \ref{subsec:spdparams} die in dieser Arbeit verwendeten Parametersätze für Motor- und Schlittendoppelpendel-Modell vorgestellt werden, soll im folgenden Abschnitt zunächst ein Überblick über die Entwicklung der in den vergangenen Arbeiten verwendeten Modellparameter gegeben werden.  

\subsection{Begründeter Stand der Modellparameter}\label{subsec:paramshist}

Die für die Modellierung erforderlichen Systemparameter des Versuchsstands wurden erstmalig 1997 von Franke \cite{franke} durch Messungen identifiziert. Die Antriebseinheit aus Spannungs-Strom-Wandler, Motor, Getriebe und Riemen ist seitdem nicht verändert worden. Daher repräsentieren die von Franke \cite{franke} identifizierten Modellparameter in Bezug auf die Antriebseinheit weiterhin den aktuellen Stand (siehe Abschnitt \ref{subsec:motorparams})
   
Während zuvor noch ein Einfachpendel verwendet worden war, konstruierte Apprich \cite{apprich} 2009 erstmalig ein Doppelpendel für den Versuchsstand. Von den Änderungen betroffen war neben den Pendeln auch die Schlittenmasse, da auch der obere Teil des Schlittens neu konstruiert wurde. Die Modellparameter für die neuen Pendelstäbe wurden, anders als bei Franke \cite{franke}, nicht gemessen, sondern aus dem CAD-Modell abgeleitet. Dies betrifft die Massen, Trägheitsmomente, Längen und Schwerpunkte der beiden Stäbe. Die Masse des Schlittens wurde von Franke \cite{franke} übernommen. Durch Messungen wurden lediglich die viskose und trockene Reibung des Schlittens gegenüber den Schienen erneut identifiziert. Im selben Jahr wurde von Kämmerer \cite{kämmerer} die viskose Lagerreibung $d_1$ zwischen Stab 1 und dem Schlitten als fehlender Modellparameter durch Messungen ergänzt. Die viskose Lagerreibung $d_2$ zwischen Stab 1 und Stab 2 wurde rechnerisch bestimmt, da das Lager gegen Ende der Arbeit getauscht werden musste. Die viskose Dämpfung des Schlittens, die von Apprich \cite{apprich} zuvor gemessen worden war, wurde durch einen deutlich höheren Schätzwert ersetzt. Außerdem wurde erstmalig die Masse von Schlitten und Antrieb zu einer schlittenseitig wirkenden Gesamtmasse zusammengefasst und ebenfalls als Schätzwert ausgewiesen.

Die Reibwerte $d_1$ und $d_2$ wurden 2011 durch Kisner \cite{kisner} erneut bestimmt. Durch die \textit{Prediction-Error Minimization Method} aus der \textit{System Identification Toolbox} von \Matlab\ wurden die Parameter $d_1$ und $d_2$ so variiert, dass die quadratische Fehlersumme minimal wird. Der Fehler wird durch die Differenz zwischen den gemessenen und den vom Modell vorhergesagten zeitlichen Winkelverläufen $\varphi_1(t)$ und $\varphi_2(t)$ bei ruhendem Schlitten und frei gewählter Anfangsauslenkung beschrieben. Für die Optimierung wurden zudem Anfangsschätzwerte für $d_1$ und $d_2$ gewählt, die nicht näher begründet wurden. Es wird davon ausgegangen, dass es sich um Erfahrungswerte handelt, da sie einerseits nicht mit den zuletzt von Kämmerer \cite{kämmerer} bestimmten Werten übereinstimmen, jedoch andererseits zu guten Ergebnissen am Versuchsstand führten. Statt der optimierten Werte wurden in den Nachfolgearbeiten die Anfangsschätzwerte weiterverwendet. Der Reibwert $d_0$ der viskosen Schlittenreibung, der nicht Gegenstand der Optimierung war, wurde mit einem deutlich höheren Wert als bei Kämmerer \cite{kämmerer} angegeben. Er ist auf Grund der ausbleibenden Erläuterung vermutlich als Erfahrungswert zu verstehen. Die zuvor von Kämmerer \cite{kämmerer} geschätzte effektive Gesamtmasse von Schlitten und Antrieb wurde nach unten korrigiert, wobei keine explizite Begründung vorliegt. Der neue Wert ist jedoch plausibel und wird daher ebenfalls als erfahrungsbasierter Schätzwert verstanden. Er ist in den weiteren Arbeiten nicht mehr verändert worden, sodass er als aktueller Stand zu betrachten ist. Bei Kisner \cite{kisner} wurde erstmalig auch eine Begrenzung der Stellkraft von $F_{\mrm{max}}=400 N$ bezüglich des Schlittens angegeben, jedoch nicht näher begründet. 2011 wurde außerdem von Noupa \cite{noupa} sowohl die viskose als auch die trockene Schlittenreibung gemessen, wobei besonders für die viskose Reibung eine hohe Richtungsabhängigkeit beobachtet wurde. Die gemessenen Werte wurden in den weiteren Arbeiten jedoch nicht weiter beachtet.

Auf Grund eines Austauschs des Lagers von Stab 2 wurde 2014 von Brehl \cite{brehl} eine erneute Identifikation der viskosen Dämpfungskonstanten $d_2$ messungsbasiert durchgeführt. Dabei wurden auch Länge und Masse von Stab 2 gemessen, womit auch das Massenträgheitsmoment neu berechnet wurde. In den Nachfolgearbeiten wurde jedoch nur die Dämpfungskonstante weiterverwendet, während für Länge, Masse und Trägheitsmoment weiterhin die CAD-Werte von Apprich \cite{apprich} verwendet wurden.  

Chang \cite{chang} konstruierte im Sommersemester 2019 ein neues Doppelpendel, wobei Schlitten und Antrieb nicht verändert worden sind. Die neuen Modellparameter wurden wieder aus dem CAD-Modell abgeleitet. Die Angabe des Schwerpunkts $s_2$ von Stab 2 scheint in der Ausarbeitung jedoch zu fehlen. Die Dämpfungskonstanten $d_1$ und $d_2$ wurden von den Vorgängern übernommen, da bei der Konstruktion die gleichen Rillenkugellager gewählt wurden wie bei Apprich \cite{apprich}. Für $d_1$ wurde der Schätzwert von Kisner \cite{kisner} und für $d_2$ der Messwert von Brehl \cite{brehl} übernommen, wobei in Changs Ausarbeitung versehentlich für beide Parameter Apprich \cite{apprich} als Quelle zitiert wird. Die viskose und die trockene Reibung des Schlittens wurden durch Messung selbst bestimmt. Wie bei Noupa \cite{noupa} wurde bei der viskosen Reibung eine auffällige Richtungsabhängigkeit festgestellt. Die Berücksichtigung der Richtungsabhängigkeit mit einer Vorsteuerung führte jedoch zu einer Verschlechterung des Systemverhaltens. Daher wurde schließlich die linksseitige Dämpfungskonstante für beide Seiten übernommen. Da das neu konstruierte Pendel noch nicht für die weiteren Bestandteile der Arbeit, wie die Auslegung der Regelung und deren Erprobung am Versuchsstand, zur Verfügung stand, wurde weiterhin auf die CAD-Werte von Apprich \cite{apprich} zurückgegriffen. Der Betrag der maximalen Stellkraft $F_{\mrm{max}}$ wurde zudem von dem von Kisner \cite{kisner} zuletzt genannten Wert von $400 N$ auf $421 N$ erhöht. Der neue Wert lässt sich rechnerisch nachvollziehen, wie Abschnitt \ref{subsec:motorparams} entnommen werden kann.

Im Rahmen einer Verifikation der von Chang \cite{chang} angegebenen Modellparameter für das neue Doppelpendel, wurden im Wintersemester 2019/2020 durch Ribeiro \cite{ribeiro} die Parameter aus dem CAD-Modell erneut abgeleitet. Da hierbei nicht genauer auf den Anlass der Neubestimmung eingegangen wurde, sich die neu bestimmten Parameter jedoch deutlich von den Vorherigen unterscheiden, wird angenommen, dass sich die von Chang \cite{chang} angegebenen Werte im Rahmen der Verifikation als fehlerbehaftet herausstellt hatten. 
Darüber hinaus wurde durch Messung die Reibung der Pendelgelenke identifiziert. Hierbei wurde neben der viskosen Reibung erstmalig auch die trockene Reibung in den Gelenken ermittelt. Die Parameter von Ribeiro \cite{ribeiro} zu den Pendelstäben sind somit aktueller Stand. Dagegen übernommen wurde weiterhin die von Kisner \cite{kisner} geschätzte Gesamtmasse mit Schlitten und Antrieb. Für die maximale Stellkraft wurde im Gegensatz zur Vorgängerarbeit Chang \cite{chang} statt $421 N$ wieder $400 N$ angenommen. Es wird vermutet, dass dadurch eine Stellkraftreserve für die Regelung vorgehalten werden sollte.

\subsection{Parameter des Motor-Modells}\label{subsec:motorparams}

Gemäß Abschnitt \ref{subsec:paramshist} werden die Modellparameter für das Motormodell franke \cite{franke} entnommen. 

Die Steuerspannung $U_{\mrm{Steuer}}$ am Eingang des Spannungs-Strom-Wandlers darf maximal $\pm10 V$ betragen. Ein Betrag größer als $10 V$ sollte vermieden werden, da sonst der Impulsstrom über den zulässigen Wert ansteigen und der Stromregler zerstört werden kann.

Die Verstärkung $K_{UI}$ des Spannungs-Strom-Wandlers kann durch ein Potentiometer bis zu einem Wert von etwa $2 \ A/V$ eingestellt werden, sodass über die maximale Steuerspannung der maximal zulässige Impulsstrom von $20A$ erreicht wird. Die zuletzt dokumentierte Einstellung liegt bei
\[
	K_{UI} = 1,87 \ \frac{A}{V} \ .
\]
Der Wert beinhaltet bereits eine Idealisierung, da Messungen von franke \cite{franke} gezeigt haben, dass die reale Verstärkung am Versuchsstand eine leichte Richtungsabhängigkeit bezüglich des Vorzeichens aufweist.

Alle weiteren verwendeten Parameter des Motormodells sind in Tabelle \ref{tab:motorparams} aufgeführt.

\begin{table}[h]
	\centering
		\begin{tabular}[t]{cc} \
					Tabelle & Dummy \\
		\end{tabular}
	\caption{Parameter - Motormodell}
	\label{tab:motorparams}
\end{table}

Die Zeitkonstante $T_{UI}$ wird als ausreichend klein angesehen, dass die Dynamik des Wandlers vernachlässigt werden kann.
Damit lässt sich mit
\[
	K_{\mrm{staticgain}} = K_{UI} \cdot K_I \cdot \frac{1}{K_G} \cdot \frac{1}{r_{32}}
\]
 die maximale Stellkraft
\[
	F_{\mrm{max}} = U_{\mrm{Steuer, max}} \cdot K_{\mrm{staticgain}} = 420,75 N
\]
berechnen. 

\subsection{Parameter des Schlittendoppelpendels}\label{subsec:spdparams}



Apprich \cite{apprich}

Chang \cite{chang}

Brehl \cite{brehl}

Ribeiro \cite{ribeiro}


\section{Aufbau in \Simulink}



\section{Implementierung in \Matlab}