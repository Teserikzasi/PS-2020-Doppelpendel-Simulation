\chapter{Einleitung}\label{cha:intro}

Das Schlittendoppelpendel ist ein nichtlineares System, an dem interessante steuerungs- und regelungstechnische Verfahren untersucht werden können. 
Am Fachgebiet Regelungstechnik und Mechatronik (rtm) der TU Darmstadt wird dazu ein Versuchsstand betrieben, der Untersuchungen und Demonstrationen am Einfach- und Doppelpendel ermöglicht. 

Eine Übersicht über die in der Vergangenheit durchgeführten Arbeiten zu diesem Versuchsstand kann Chang \cite{chang} entnommen werden. Im Zuge einer Neukonstruktion des Pendels im Jahr 2019 durch Chang \cite{chang} haben sich die Systemparameter geändert. Erste Erfahrungen mit dem neuen System zeigen im Vergleich zu den Vorherigen jedoch ein ungünstigeres Verhalten in Bezug auf die regelungstechnischen Eigenschaften. Dies legt eine Untersuchung des Einflusses der Systemparameter auf die Regelbarkeit des Systems nahe.

Da es bei der Trajektorienfolgeregelung aufgrund der Beschränkung einiger Systemzustände immer wieder zu Schwierigkeiten bei der Berechnung und Umsetzung von Trajektorien kam, wurde durch Fauvé \cite{fauve} der Ansatz der Modellprädiktiven Regelung (engl.: \textit{Model Predictive Control})(MPC), die im Gegensatz zu den klassischen Regelungsverfahren eine Berücksichtigung der Systembeschränkungen ermöglicht, für das bestehende Pendelsystem untersucht. Obwohl sich der Ansatz aufgrund ungenügender Echtzeitfähigkeit nicht für die Regelung eignete, erwies er sich in Bezug auf die Berechnung von Trajektorien für Arbeitspunktwechsel als vielversprechend. Erste Versuche die erfolgreich berechneten Trajektorien in der Simulation mit Hilfe einer Trajektorienfolgeregelung zu stabilisieren, gelangen jedoch nicht. Daher bleibt zu zeigen, dass die mittels NMPC gefundenen Trajektorien am System stabilisiert werden können. Zudem wird vermutet, dass neben der Regelbarkeit auch das Finden von Trajektorien durch die Systemparameter beeinflusst wird. 

Ziel dieser Arbeit ist daher die Untersuchung des Einflusses der Systemparameter auf Steuerung und Regelung des Schlittendoppelpendels. Das System wird dazu im Rahmen dieser Arbeit rein simulativ betrachtet. Aus diesem Grund soll in einem ersten Schritt ein ausführliches Simulationsmodell des Versuchsstands aufgebaut werden. Da in der Vergangenheit mehrfach verschiedene Systemparameter angenommen wurden, soll nun ein begründeter Parameterstand unter Berücksichtigung der vorherigen Arbeiten recherchiert werden. Anschließend ist eine Regelung auf Basis der Vorgängerarbeiten auszulegen, mit deren Hilfe auch die Untersuchung des Einflusses der Systemparameter auf die Regelbarkeit des Systems in den vier typischen Arbeitspunkten des Doppelpendels erfolgen soll. Der Einfluss wird zudem auch in Bezug auf die Trajektorienberechnung untersucht. Hierzu ist die bestehende NMPC für die Parameteruntersuchungen sowie den zukünftigen Einsatz als Instrument für die Trajektoriensuche weiterzuentwickeln. Um an die Arbeit von Fauvé \cite{fauve} anzuknüpfen, soll zunächst eine Trajektorienfolgeregelung entworfen werden, mit der zu zeigen ist, dass die mittels NMPC gefundenen Trajektorien in der Simulation stabilisiert werden können. 
