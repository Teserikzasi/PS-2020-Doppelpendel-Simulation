\chapter{Einleitung}\label{cha:intro}

Das Doppelpendel ist ein nichtlineares System, an dem interessante steuerungs- und regelungstechnische Methoden untersucht werden können. 
Am Fachgebiet Regelungstechnik und Mechatronik (rtm) der TU Darmstadt existiert dazu ein Versuchsstand, bei dem ein Doppelpendel an einem Schlitten befestigt ist, welcher über einen Elektromotor gesteuert werden kann.
Daran sind Untersuchungen und Demonstrationen am (inversen) Einfach- und Doppelpendel möglich. 

Eine Übersicht über die in der Vergangenheit durchgeführten Arbeiten zu diesem Versuchsstand kann \cite{chang} und \cite{ribeiro} entnommen werden. Im Zuge einer Neukonstruktion des Doppelpendels durch Chang \cite{chang} im Jahr 2019 haben sich die Systemparameter geändert. Erste Erfahrungen mit dem neuen System zeigen im Vergleich zu dem Vorherigen jedoch ein ungünstigeres Verhalten in Bezug auf die regelungstechnischen Eigenschaften. 
Vermutet wird dabei, dass dies mit den Systemparametern und der erhöhten trockenen Reibung des ersten Pendelgelenks, das mit einem Schleifringkontakt konstruiert wurde, zusammenhängt. 
%Dies legt eine Untersuchung des Einflusses der Systemparameter auf die Regelbarkeit des Systems nahe.

Ziel dieser Arbeit ist daher die systematische Untersuchung des Einflusses der Systemparameter auf Steuerung und Regelung des \spds s, welches im Rahmen dieser Arbeit rein simulativ betrachtet wird. Aus diesem Grund soll in einem ersten Schritt ein detailliertes Simulationsmodell des Versuchsstands aufgebaut werden. 
Da in der Vergangenheit mehrfach verschiedene Systemparameter angenommen worden waren, soll nun ein begründeter Stand der Parameter unter Berücksichtigung der vorherigen Arbeiten recherchiert werden. 

Anschließend ist eine Regelung in den vier Arbeitspunkten des Doppelpendels auf Basis der Vorgängerarbeiten auszulegen und die Stabilität und Regelgüte mit verschiedenen Reglerparametern zu untersuchen. 
Außerdem wird der Einfluss des \beob s analysiert.
%Es werden Kriterien festgelegt, um die „Regelbarkeit“ des Systems zu bewerten.
Darauf aufbauend soll der Einfluss der Systemparameter auf die Regelbarkeit des Systems untersucht werden.

Außerdem sollen Verfahren zur Ermittlung von Trajektorien für Arbeitspunktwechsel implementiert und erweitert werden.
Der Einfluss der Systemparameter wird zudem auch in Bezug auf die Berechnung der Trajektorien untersucht. %betonung deutlicher
Hierzu ist die bestehende Modellprädiktive Regelung für die Parameteruntersuchungen sowie den zukünftigen Einsatz als Instrument für die Trajektoriensuche weiterzuentwickeln. 
Um an die Arbeit von Fauvé \cite{fauve} anzuknüpfen, soll eine Trajektorienfolgeregelung entworfen werden, mit der zu zeigen ist, dass die gefundenen Trajektorien stabilisiert werden können. 

%Bei allen im Rahmen dieser Arbeit realisierten Implementierungen soll auf einen sinnvolle Struktur und einen modularen Aufbau geachtet werden.
%Dadurch soll die Wiederverwendbarkeit in zukünftigen Arbeiten gewährleistet werden.