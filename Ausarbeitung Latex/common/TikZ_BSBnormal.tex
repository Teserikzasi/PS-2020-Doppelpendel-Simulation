% TikZstyles f�r Blockschaltbilder

\renewcommand{\TikZscale}{1}
\tikzset{every picture/.style={node distance=4\mm, >=stealth'}}

%% Bl�cke
	% Rechteckige Bl�cke
	\tikzstyle{block}      = [draw, semithick, rectangle, minimum height=8\mm, minimum width=8\mm, inner sep=3pt]
	\tikzstyle{NLblock}    = [draw, semithick, rectangle, minimum height=8\mm, minimum width=8\mm, inner sep=3pt, double distance=1.2pt]
	\tikzstyle{PICblock}   = [draw, semithick, rectangle, minimum height=8\mm, minimum width=8\mm, inner sep=2pt]
	\tikzstyle{NLPICblock} = [draw, semithick, rectangle, minimum height=8\mm, minimum width=8\mm, inner sep=3pt, double distance=1.2pt]
	\tikzstyle{noblock}	   = [rectangle, inner sep=-0.6pt]

	% Dreieckige Bl�cke
	\tikzstyle{Rgain}			 = [draw, semithick, isosceles triangle, inner sep=1pt, minimum height=8\mm, isosceles triangle apex angle=60]
	\tikzstyle{Lgain}			 = [draw, semithick, isosceles triangle, inner sep=1pt, minimum height=8\mm, isosceles triangle apex angle=60, shape border rotate=180]
	\tikzstyle{Ugain}			 = [draw, semithick, isosceles triangle, inner sep=1pt, minimum height=8\mm, isosceles triangle apex angle=60, shape border rotate=90]
	\tikzstyle{Dgain}			 = [draw, semithick, isosceles triangle, inner sep=1pt, minimum height=8\mm, isosceles triangle apex angle=60, shape border rotate=-90]

	% Runde Bl�cke
	\tikzstyle{sum}   	   = [draw, semithick, circle, inner sep=1pt, minimum size=3\mm]
	\tikzstyle{branch}		 = [draw, circle, inner sep=0pt, minimum size=1\mm, fill=black]
	\tikzstyle{BRANCH}		 = [coordinate]


%% Verbindungselemente
	% Linien mit Pfeil
	\tikzstyle{to}  		= [->, thick]
	\tikzstyle{toNL}		= [->, thick, shorten >=0.9pt]
	\tikzstyle{NLto}		= [->, thick, shorten <=0.9pt]
	\tikzstyle{NLtoNL}	= [->, thick, shorten <=0.9pt, shorten >=0.9pt]
	
	\tikzstyle{TO}  		= [semithick, double distance=2pt, shorten >=2mm, decoration={markings,mark=at position 1 with {\arrow[semithick]{open triangle 60}}}, preaction={decorate},postaction={draw, line width=2pt, white, shorten >= 1.5mm}]
	
	\tikzstyle{TONL}  		= [semithick, double distance=2pt, shorten >=2.2mm, decoration={markings,mark=at position 1 with {\arrow[semithick]{open triangle 60}}, transform={xshift=-0.7pt}}, preaction={decorate},postaction={draw, line width=2pt, white, shorten >= 1.5mm}]
	
	\tikzstyle{NLTO}  		= [semithick, double distance=2pt, shorten <=0.9pt, shorten >=2mm, decoration={markings,mark=at position 1 with {\arrow[semithick]{open triangle 60}}}, preaction={decorate}, postaction={draw, line width=2pt, white, shorten >= 1.5mm}]
	
	\tikzstyle{NLTONL}  		= [semithick, double distance=2pt, shorten <=0.9pt, shorten >=2.2mm, decoration={markings,mark=at position 1 with {\arrow[semithick]{open triangle 60}}}, preaction={decorate},postaction={draw, line width=2pt, white, shorten >= 1.5mm}]
	
	\tikzstyle{innerWhite} = [semithick, white,line width=2pt, shorten >= 2mm, shorten <= 2mm]
	
	%\tikzstyle{TO}  		= [semithick, double distance=2pt, shorten >=2mm, decoration={markings,mark=at position 1 with {\arrow[semithick]{open triangle 60}}}, postaction={decorate}]
		%
	%\tikzstyle{TONL} = [semithick, double distance=2pt, shorten >=2.2mm, decoration={markings,mark=at position 1 with {\arrow[semithick]{open triangle 60}}, transform={xshift=-0.7pt}}, postaction={decorate}]
	%
	%\tikzstyle{NLTO} = [semithick, double distance=2pt, shorten <=0.79pt, shorten >=2mm, decoration={markings,mark=at position 1 with {\arrow[semithick]{open triangle 60}}}, postaction={decorate}]
	%
	%\tikzstyle{NLTONL} = [semithick, double distance=2pt, shorten <=0.79pt, shorten >=2.2mm, decoration={markings,mark=at position 1 with {\arrow[semithick]{open triangle 60}}, transform={xshift=-0.7pt}}, postaction={decorate}]
	

	% Linien ohne Pfeil 
	\tikzstyle{line}       = [thick]
	\tikzstyle{lineNL}     = [thick, shorten >= 0.6pt]
	\tikzstyle{NLline}     = [thick, shorten <= 0.6pt]
	\tikzstyle{NLlineNL}	 = [thick, shorten <= 0.6pt, shorten >= 0.6pt]
	
	% Doppellinien ohne Pfeil
	\tikzstyle{LINE}       = [semithick, double distance=2pt]
	\tikzstyle{LINENL}     = [semithick, double distance=2pt, shorten >= 0.9pt]
	\tikzstyle{NLLINE}     = [semithick, double distance=2pt, shorten <= 0.9pt]
	\tikzstyle{NLLINENL}	 = [semithick, double distance=2pt, shorten <= 0.9pt, shorten >= 0.9pt]
	
	\tikzstyle{neg}				 = [postaction={decorate,decoration={markings, mark=at position 1 with{\draw[-](-2pt,-3pt)--(-2pt,-7pt);}}}]
	
	\tikzstyle{labelabove} = [above, anchor=base, yshift=0.75ex]
	\tikzstyle{labelbelow} = [below, anchor=base, yshift=-2ex]
	\tikzstyle{labelright} = [right]
	\tikzstyle{labelleft}  = [left]
	
	
	% Pins und Labels
	\tikzstyle{every pin edge}	= [<-, thick]
	\tikzstyle{every pin}	 			= [pin distance=5\mm]
	\tikzstyle{every label}			= [font=\small]
	\tikzstyle{terminal}				= [coordinate]
	
	% extra Symbole f�r circuits-Library
	\tikzstyle{jack}	= [draw, circle, minimum size=1.5mm, inner sep=0]
