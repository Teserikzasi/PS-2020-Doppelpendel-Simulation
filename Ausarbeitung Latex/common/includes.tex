
\usepackage{iflang}
\usepackage{xspace}


\usepackage{bibgerm}		% Für deutsche Literaturverwaltung


\usepackage{graphicx}		% zum Einbinden von Postscript

\usepackage{subfig}			% Für Unterabbildungen
%	\captionsetup[subtable]{position=top}
\usepackage{rotating}		% Zum Drehen von Objekten
\usepackage{placeins}		% Für \FloatBarrier


\usepackage{booktabs}
\usepackage{array}			% Für Zellentyp "m{}" in tabular-Umgebungen (Vertikal zentriert)
\usepackage{ltxtable} 	% Vereinigt TabularX und Longtable
\usepackage{multirow}		% Für mehrzeilige Felder in Tabellen
\usepackage{cellspace}	% Für gescheiten Abstand von Formeln zu Tabellen-
                        % rändern
              

\usepackage{amsmath}		% Mehr mathematischen Formelsatz
\usepackage{amsfonts}
\usepackage{amssymb}
\usepackage{icomma}			% Damit nach Dezimalkommas kein Abstand eingefügt wird
							          % (in math-Umgebungen)
                        
% =========================================================
% Workaround für Problem mit amsmath und doppelten Akzenten
\makeatletter
\protected\def\mathaccentV#1#2#3#4#5%
  {%
    \ifmmode
      \mathaccentV@do{#2}{#3}{#4}{#5}%
    \else
      \@xp\nonmatherr@\csname #1\endcsname
    \fi
  }
\def\mathaccentV@do#1#2#3#4%
  {%
    \global\let\macc@nucleus\@empty
    \mathaccent"\accentclass@#1#2#3{#4}\macc@nucleus
  }
\makeatother
% =========================================================

\usepackage{upgreek}		% Für nicht-kursive kleine griechischen Buchstaben


\usepackage[ngerman]{varioref}







\usepackage{color}
%\usepackage[outdir=./]{epstopdf}