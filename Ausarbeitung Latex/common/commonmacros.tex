% Diese Datei dient zum Definieren nützlicher Befehle.
% Sie soll lediglich als Beispiel dienen, wie Befehle definiert werden, und welche Befehle nützlich sein können.

% Inhalt
% ======
%	Makros für Referenzen (Abbildungen, Zitate, ...)
%	Makros für Abbildungen
%	Makros für Einheiten, Exponenten
%	Makros für Formeln
%	Makros für Entwurf
%   Definitionen für Umgebungen

% Allgemeine Abkürzungen
% ======================
	\newcommand{\bzw}{bzw.\@\xspace}
	\newcommand{\Bzw}{Bzw.\@\xspace}
	\newcommand{\bzgl}{bzgl.\@\xspace}
	\newcommand{\ca}{ca.\@\xspace}
	\newcommand{\dah}{d.\thinspace{}h.\@\xspace}
	\newcommand{\Dah}{D.\thinspace{}h.\@\xspace}
	\newcommand{\ds}{d.\thinspace{}s.\@\xspace}
	\newcommand{\evtl}{evtl.\@\xspace}
	\newcommand{\ua}{u.\thinspace{}a.\@\xspace}
	\newcommand{\Ua}{U.\thinspace{}a.\@\xspace}
	\newcommand{\uU}{u.\thinspace{}U.\@\xspace}
	\newcommand{\UU}{U.\thinspace{}U.\@\xspace}
	\newcommand{\usw}{usw.\@\xspace}
	\newcommand{\etc}{etc.\@\xspace}
	\newcommand{\va}{v.\thinspace{}a.\@\xspace}
	\newcommand{\Vgl}{Vgl.\@\xspace}
	\newcommand{\vgl}{vgl.\@\xspace}
	\newcommand{\zB}{z.\thinspace{}B.\@\xspace}
	\newcommand{\ZB}{Zum Beispiel\xspace}
	\newcommand{\sa}{s.\thinspace{}a.\@\xspace}
	\newcommand{\ia}{i.\thinspace{}a.\@\xspace}
	\newcommand{\bspw}{bspw.\@\xspace}
	\newcommand{\Bspw}{Bspw.\@\xspace}
	\newcommand{\ggf}{ggf.\@\xspace}
	\newcommand{\Ggf}{Ggf.\@\xspace}
	\newcommand{\zT}{z.\thinspace{}T.\@\xspace}
	\newcommand{\ZT}{Z.\thinspace{}T.\@\xspace}
	\newcommand{\iA}{i.\thinspace{}A.\@\xspace}
	\newcommand{\IA}{I.\thinspace{}A.\@\xspace}

	\newcommand{\ie}{i.\thinspace{}e.\@\xspace}
	\newcommand{\Ie}{I.\thinspace{}e.\@\xspace}
	\newcommand{\eg}{e.\thinspace{}g.\@\xspace}






% Makros für Referenzen (Abbildungen, Zitate, ...)
% ================================================

	% Referenzierung auf Abbildungen, Tabellen, etc. (Hyperref-fähig)
	\newcommand{\figref}[1]{\hyperref[#1]{\figurename\ \ref*{#1}}}
	\newcommand{\tabref}[1]{\hyperref[#1]{\tablename\ \ref*{#1}}}
	\newcommand{\equref}[1]{\hyperref[#1]{Gl.~(\ref*{#1})}}
	\newcommand{\defref}[1]{\hyperref[#1]{Definition~\ref*{#1}}}
	\newcommand{\thrref}[1]{\hyperref[#1]{Satz~\ref*{#1}}}
	\newcommand{\figvref}[1]{\hyperref[#1]{\figurename\ }\vref{#1}}
	\newcommand{\tabvref}[1]{\hyperref[#1]{\tablename\ }\vref{#1}}
	\newcommand{\eqvref}[1]{\hyperref[#1]{Gl.~(\ref*{#1}) auf Seite~\pageref*{#1}}}
	\newcommand{\pagerefh}[1]{\hyperref[#1]{Seite~\pageref*{#1}}}
	
	\newcommand{\charef}[1]{\hyperref[#1]{\chaptername~\ref*{#1}}}
  \newcommand{\secref}[1]{\hyperref[#1]{Abschnitt~\ref*{#1}}}
	\newcommand{\appref}[1]{\hyperref[#1]{Anhang~\ref*{#1}}}

	\newcommand{\lstref}[1]{\hyperref[#1]{Listing~\ref*{#1}}}
	\newcommand{\algoref}[1]{\hyperref[#1]{Algorithmus~\ref*{#1}}}
	\newcommand{\ftnref}[1]{\hyperref[#1]{Fußnote~\ref*{#1}}}
  
  
% Makros für Abbildungen
% ======================

% Textbausteine
% =============
	% Produktnamen
	\newcommand*{\Matlab}{\textsc{Matlab}}
	\newcommand*{\Matlabreg}{\textsc{Matlab}\textsuperscript{\tiny \textregistered}}
	\newcommand*{\MatSim}{\textsc{Matlab/Simulink}}
	\newcommand*{\Simulink}{\textsc{Simulink}}
	\newcommand*{\Simulinkreg}{\textsc{Simulink}\textsuperscript{\tiny \textregistered}}
	
	% Das Makro |\name|\marg{person} formatiert einen Personennamen bspw. eines Erfinders oder Entdeckers gemäß |\name{Euler}| \arrow\ \name{Euler}.
	\newcommand*{\name}[1]{\textsc{#1}}



% Makros für Einheiten, Exponenten
% ================================

	\newcommand*{\unit}[1]{\ensuremath{\mathrm{#1}}}
	
	% Wert mit Einheit (mit kleinem Leerzeichen dazwischen), aus Text- UND Math-Modus
	\newcommand*{\valunit}[2]{\ensuremath{#1\,\mrm{#2}}}


	% "°C", im Text- oder Mathe-Modus
	\newcommand*{\degC}{
		\ifmmode
			^\circ \mrm{C}%
		\else
			\textdegree C%
		\fi}

	\newcommand*{\degree}{
		\ifmmode
			^\circ%
		\else
			\textdegree%
		\fi}
	
	% Für Exponentenschreibweise ( Anwendung: 123\E{3} )
	\newcommand*{\E}[1]{\ensuremath{\cdot 10^{#1}}}
	
	\newcommand*{\eexp}[1]{\ensuremath{\mathrm{e}^{#1}}}
	\newcommand*{\iu}{\ensuremath{\mathrm{j}}}

	\newcommand*{\todots}{\ensuremath{,\,\hdots,\,}}


% Makros für Formeln
% ==================

	% Definition für Vektor und Matizen
    \newcommand*{\mat}[1]{{\ensuremath{\boldsymbol{\mathrm{#1}}}}}
    \newcommand*{\ma}[1]{{\ensuremath{\boldsymbol{\mathrm{#1}}}}}
    \newcommand*{\mas}[1]{\ensuremath{\boldsymbol{#1}}}
    \newcommand*{\ve}[1]{\ensuremath{\boldsymbol{#1}}}
    \newcommand*{\ves}[1]{\ensuremath{\boldsymbol{\mathrm{#1}}}}

	\newcommand*{\AP}{\ensuremath{\mathrm{AP}}}
	\newcommand*{\doti}{\ensuremath{(i)^\cdot}}
	
	\newcommand*{\inprod}[2]{\ensuremath{\langle #1,\,#2 \rangle}}
	
	\newcommand*{\ul}[1]{\underline{#1}}

	% gerades "d" (z.B. für Integral)
	\newcommand*{\ud}{\ensuremath{\mathrm{d}}}
	
	% normaler Text in Formeln
	\newcommand*{\tn}[1]{\textnormal{#1}}
	
	% nicht-kursive Schrift in Formeln
	\newcommand*{\mrm}[1]{\ensuremath{\mathrm{#1}}}
	
	% gerades "T" für Transponiert
	\newcommand*{\transp}{\ensuremath{\mathrm{T}}}
	
	% gerades "rg"
	\newcommand*{\rang}{\ensuremath{\operatorname{rg}}}

	% Für geklammerte Ausdrücke mit Index (Subscript)
	% (einmal mit kursiven Index, einmal mit geradem Index)
	\newcommand*{\grpsb}[2]{\ensuremath{\left(#1\right)_{#2}}}
	\newcommand*{\grprsb}[2]{\ensuremath{\left(#1\right)_{\mathrm{#2}}}}

	% Ableitungen und Integrale
		% "normale" Ableitung (mit geraden "d"s)
		\newcommand*{\normd}[2]{\ensuremath{\frac{\mathrm{d}#1}{\mathrm{d}#2}}}
		\newcommand*{\normdat}[3]{\ensuremath{\left.\frac{\mathrm{d} #1}{\mathrm{d} #2}\right|_{#3}}}
	
		% Materielle Ableitung
		\newcommand*{\matd}[2]{\ensuremath{\frac{\mathrm{D} #1}{\mathrm{D} #2}}}
		\newcommand*{\matdat}[3]{\ensuremath{\left.\frac{\mathrm{D} #1}{\mathrm{D} #2}\right|_{#3}}}
	
		% Partielle Ableitung
		\newcommand*{\partiald}[2]{\ensuremath{\frac{\partial #1}{\partial #2}}}
		\newcommand*{\partialdat}[3]{\ensuremath{\left.\frac{\partial #1}{\partial #2}\right|_{#3}}}
	
	
	% Transformationen
	\newcommand*{\FT}[1]{\ensuremath{\mathcal{F}\left(#1\right)}}
	\newcommand*{\FTabs}[1]{\ensuremath{\left|\mathcal{F}\left(#1\right)\right|}}
	\newcommand*{\IFT}[1]{\ensuremath{\mathcal{F}^{-1}\left(#1\right)}}
	\newcommand*{\DFT}[1]{\ensuremath{\mathrm{DFT}\left(#1\right)}}
	\newcommand*{\DFTabs}[1]{\ensuremath{\left|\mathrm{DFT}\left(#1\right)\right|}}
	\newcommand*{\Laplace}[1]{\ensuremath{\mathcal{L}\left(#1\right)}}
	\newcommand*{\InvLaplace}[1]{\ensuremath{\mathcal{L}^{-1}\left(#1\right)}}
	\newcommand*{\invtrans}{\ensuremath{\quad\bullet\!\!-\!\!\!-\!\!\circ\quad}}
	\newcommand*{\trans}{\ensuremath{\quad\circ\!\!-\!\!\!-\!\!\bullet\quad}}


	% Manche textcomp-Zeichen funktionieren mit dem TU-Design nicht, diese können dann mit diesem
	% Befehl gesetzt werden.
	\newcommand*{\textcompstdfont}[1]{{\fontfamily{cmr} \fontseries{m} \fontshape{n} \selectfont #1}}
	


% =================================================================================
% Definitionen für Tabellen
% =================================================================================
% Spaltenstil zum Ausrichten von Zahlen am Dezimaltrennzeichen
\newcolumntype{d}[1]{D{.}{,}{#1}}

%\newcolumntype{L}[1]{>{\raggedright\arraybackslash}m{#1}}
%\newcolumntype{C}[1]{>{\centering\arraybackslash}m{#1}}
%\newcolumntype{R}[1]{>{\raggedleft\arraybackslash}m{#1}}

