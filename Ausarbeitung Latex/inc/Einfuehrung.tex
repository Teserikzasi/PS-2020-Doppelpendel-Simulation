\chapter{Einführung}\label{cha:Intro}
%
Beim Anfertigen einer Abschlussarbeit steht man als Studierender meist zum
ersten Mal vor dem Problem, einen längeren wissenschaftlichen Text mit Bildern,
Gleichungen und Referenzen schreiben zu müssen. Dafür bietet sich das
Textsatz-System \LaTeX\ an, zu dessen Vorteilen die weitgehende Trennung von
Inhalt und Layout gehören. Damit sich Studierende mehr mit dem \emph{Inhalt} der
Arbeit beschäftigen können, stellt das IAT ein \LaTeX-Dokument zur Verfügung, in
dem das \emph{Layout} der \verb|tudreport|-Klasse für das IAT angepasst wurde. Auch der vorliegende Text ist beispielhaft mit dieser Klasse geschrieben. \LaTeX-Grundlagen findet man \zB\ in \cite{Kopka}
oder \cite{Schmidt}. Allgemeines über
wissenschaftliche Arbeiten findet man in \cite{Friedrich}; als Einstieg in
die Typografie ist \cite{Willberg} sehr zu empfehlen.

Die Arbeit kann in Deutsch oder wahlweise auch in Englisch verfasst werden.

In der vorliegenden Anleitung werden in Kapitel \ref{cha:Hinweise} zuerst allgemeine Hinweise und Tipps zur Durchführung einer wissenschaftlichen Arbeit gegeben. Im Anschluß daran gibt Kapitel~\ref{cha:Hinweise_Latex} allgemeine Hinweise zum Erstellen der schriftlichen Arbeit. Es werden die \LaTeX-spezifischen Befehle vorgestellt, mit denen die Hinweise umgesetzt werden können. Die Hinweise in Kapitel~\ref{cha:Hinweise_Latex} sind auch für Studierende relevant, die sich gegen die Erstellung der Arbeit mit \LaTeX\ entscheiden. Da das Layout der Arbeit in diesem Fall zusätzlich selbst erstellt werden muss, dient die vorliegende Anleitung als Vorlage. In Kapitel \ref{cha:Verzeichnisstruktur} wird schließlich beschrieben, wie mit der \LaTeX-Vorlage gearbeitet werden sollte. 

Im Anhang findet sich eine Checkliste die vor Abgabe der Arbeit unbedingt abgearbeitet werden sollte, um zu prüfen, ob alle Vorgaben und Hinweise beachtet wurden.

