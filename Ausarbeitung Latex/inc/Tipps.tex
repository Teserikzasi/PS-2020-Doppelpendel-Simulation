\chapter{Tipps für die Durchführung einer wissenschaftlichen Arbeit}
\label{cha:Hinweise}
%
%Eine wissenschaftliche Abschlussarbeit kann im Allgemeinen in die folgenden 4 Phasen gegliedert werden.
%
\section*{1. Phase: Einarbeitung und Literaturrecherche}
\label{sec:Einarbeitung}
\addcontentsline{toc}{section}{1. Phase: Einarbeitung und Literaturrecherche}
\begin{itemize}
	\item dient zur Verdeutlichung der gegebenen Problemstellung
	\item führt den Bearbeiter auf den \glqq{}State-of-the-Art\grqq{} hin $\rightarrow$ Literatur
	\item zeigt Möglichkeiten zur Problemlösung auf, die dann in Phase 2 genutzt werden können
	\item dient der Ideenfindung
	\item regelmäßige Treffen mit dem Betreuer
\end{itemize}


\section*{Generelles zu Treffen mit dem Betreuer}
\label{sec:TreffenBetreuer}
\addcontentsline{toc}{section}{Generelles zu Treffen mit dem Betreuer}
\begin{itemize}
	\item persönliche Treffen sollten gut vorbereitet werden (z.B. Fragenkatalog)
	\item Ergebnisse des Treffens und Anmerkungen des Betreuers festhalten (Notizen)
\end{itemize}


\subsection*{Literaturrecherche und -management}
\label{sec:Literaturrecherche}
\addcontentsline{toc}{subsection}{Literaturrecherche und -management}
Zur Sammlung und Verwaltung von Literatur dienen sog. Literaturdatenbanken. Diese Programme ermöglichen oft die direkte Verwendung als Literaturverzeichnis im späteren Text.

\emph{Empfehlung: JabRef (Freeware, auf Java-Basis, für alle Plattformen)}

Das Suchen und Lesen der aktuellsten Literatur ist wichtig für eine gute Übersicht über die
Problemstellung. Über die ULB ist es möglich viele Online-Quellen zu nutzen, um Texte auch als
pdf Dokumente erhalten zu können. Die für unseren Bereich wichtigsten Seiten sind:
\begin{itemize}
	\item \href{http://rzblx1.uni-regensburg.de/ezeit/}{Elektronische Zeitschriftenbibliothek} (DB aller Zeitschriften und Berechtigungen)
	\item \href{http://ieeexplore.ieee.org}{IEEEXplore} (Zugang zu allen Medien der IEEE)
	\item \href{http://www.sciencedirect.com}{Sciencedirect} (Zugang zu Zeitschriften aus dem ELSEVIER Verlag)
	\item \href{http://citeseerx.ist.psu.edu}{CiteSeer} (freie DB mit Zugang zu vielen Volltexten)
	\item \href{http://www.ifac-control.org/publications/ifac-papersonline.net}{IFAC-PapersOnline} (Archiv mit Artikeln aller IFAC Konferenzen)
	\item \href{http://link.springer.com/}{Springer E-Books} (Archiv des Springer Verlags mit vielen eBooks und Zeitschriften)
\end{itemize}
auch mal auf den Webseiten der Autoren schauen und natürlich \glqq{}googlen\grqq{}


\section*{2. Phase: Bearbeitung des Problems}
\label{sec:Bearbeitung}
\addcontentsline{toc}{section}{2. Phase: Bearbeitung des Problems}
\begin{itemize}
	\item dient zur Lösung der Problemstellung
	\item kreativste und anstrengendste Phase der Arbeit
	\item Verwendung professioneller Hilfsmittel (Programme wie Matlab oder Mathematica etc.)
	\item Treffen wenn Bedarf besteht, keine Regelmäßigkeit mehr
	\item sowohl die Implementierung als auch die Ausarbeitung/Präsentation sollten spätestens ab dieser Phase mittels eines Versionskontrollsystems wie \bspw \textsc{Git} oder \textsc{SVN} versioniert werden und regelmäßig zur Datensicherung auf einen externen Speicher übertragen werden, damit ältere (möglicherweise noch funktionsfähige) Stände wiederhergestellt werden können und keine Daten bei Zerstörung oder Verlust des Arbeitsrechners verloren gehen
\end{itemize}


\section*{3. Phase: Ergebnisse zusammen stellen}
\label{sec:Ergebnisse}
\addcontentsline{toc}{section}{3. Phase: Ergebnisse zusammen stellen}
\begin{itemize}
	\item dient der Reorganisation der Arbeit
	\item sämtliche Ergebnisse werden festgehalten
	\item Strukturierung und Gliederung der Ergebnisse, so dass die nächste Phase (Schreiben) gut durchgeführt werden kann
	\item wenige, längere Treffen zur Ergebnisbesprechung mit dem Betreuer
\end{itemize}


\subsection*{Ergebnissicherung}
\label{sec:Ergebnissicherung}
\addcontentsline{toc}{subsection}{Ergebnissicherung}
\begin{itemize}
	\item alles zusammentragen was erreicht wurde $\rightarrow$ guter Überblick notwendig (S.O.)
	\item auch Programmcode ist ein Ergebnis $\rightarrow$ verständlich kommentieren (Englisch)
	\item auf  Wiederverwendbarkeit von Grafiken achten (Linienstärke, Farbe, Beschriftung, \ldots)
	\item nur noch kleine Änderungen durchführen (z.B. Parametereinstellungen)
\end{itemize}
\emph{Ergebnisse sollten für sich sprechen und für jeden verständlich sein (ohne Erklärung)}


\section*{4. Phase: Dokumentation und Präsentation}
\label{sec:Dokumentation}
\addcontentsline{toc}{section}{4. Phase: Dokumentation und Präsentation}
\begin{itemize}
	\item Verfassen der Arbeit und erstellen der Präsentation
	\item letzte Verfeinerungen an den Ergebnissen (falls notwendig)
	\item schwierigste Phase der Arbeit
	\item regelmäßige Treffen zur Korrektur des Textes / der Präsentation
\end{itemize}

\newpage


\subsection*{Wissenschaftliches Schreiben}
\label{sec:Wissenschaftliches Schreiben}
\addcontentsline{toc}{subsection}{Wissenschaftliches Schreiben}
\begin{itemize}
	\item nicht am Layout der Arbeit aufhalten $\rightarrow$ Vorlage verwenden
	\item sehr aufwendiger Prozess von Schreiben - Verbessern - neu Schreiben - \ldots
	\item Aufwand darf nicht unterschätzt werden (Richtwert: 1-2 Seiten / Tag)
	\item einfach erst einmal aufschreiben -  korrigiert wird dann später
	\item Struktur einer wissenschaftlichen Arbeit ist vorgegeben
	\begin{itemize}
		\item \textbf{Titelseite} mit Art der Arbeit, Titel, Namen des Autors sowie Abgabedatum.
		\item \textbf{Aufgabenstellung} wird vom Betreuer der Arbeit zur Verfügung
		  gestellt.
		\item \textbf{Erklärung} zur Selbständigkeit. Der Text ist vorgegeben und wird bei Verwendung der Vorlage automatisch erzeugt.
		\item \textbf{Kurzfassung} der Arbeit.
		  Der Umfang soll so bemessen sein, dass die englische Version
		  (\textbf{Abstract}) auf die gleiche Seite passt.
		\item \textbf{Inhaltsverzeichnis} wird in \LaTeX{} durch \verb|\tableofcontents| automatisch
		  erzeugt.
		\item \textbf{Symbole und Abkürzungen}. Dieses Verzeichnis erstellt man am Besten von Hand. Die Einteilung in lateinische und griechische Symbole und Formelzeichen kann nach Bedarf geändert werden (zum Beispiel nach Kapiteln oder Konzepten) oder ganz weggelassen werden
		\item \textbf{Hauptteil der Arbeit}, in einzelne Kapitel und Abschnitte unterteilen.
		\item \textbf{Anhang}. Hier können Abschnitte stehen, die beim Lesen der Arbeit
		  stören würden, \zB\ Programmcode, technische Daten oder lange mathematische
		  Beweise.
		\item \textbf{Literaturverzeichnis} wird entweder von Hand erstellt oder automatisch generiert (in \LaTeX{} \zB mit.
		  \textsc{Bib}\TeX )
		\item \label{itm:Ordner} \textbf{Zusätzliches Material} wie \zB\ der
		  vollständige Programmcode eines Software-Projekts gehört nicht in die Arbeit,
	  sondern kann in einem separaten Ordner abgelegt werden.
	\end{itemize}
\end{itemize}
$P_\text{rägnanz}\; O_\text{rdnung}\; E_\text{infachheit}\; M_\text{otivation}$ (einfache, kurze, strukturierte und anregende Sätze)


\subsection*{Verteidigung und Präsentation}
\label{sec:Verteidigung}
\addcontentsline{toc}{subsection}{Verteidigung und Präsentation}
\begin{itemize}
	\item Inhalt der Arbeit auf ca. 10 wesentliche Punkte reduzieren
	\item je Punkt eine Folie maximal zwei
	\item je Folie 1-2 min Gesprächszeit
	\item Struktur der Präsentation:
	\begin{itemize}
		\item Motivation / Einleitung (Aufgabenstellung)
		\item Grundlagen
		\item Lösungsweg
		\item Ergebnisse
		\item Zusammenfassung (und Ausblick)
	\end{itemize}
	\item komplexe Sachverhalte durch Abbildungen verdeutlichen
	\item nur Stichpunkte schreiben, keine ganzen Sätze
	\item klare, einfach nachvollziehbare Notation (z.B. bei Variablen) verwenden
	\item mehrmaliges Üben des Vortrags (\zB vor dem Spiegel, vor der Familie, \ldots)
\end{itemize}
\emph{nichts weglassen was auf einer Folie steht, gerne zusätzliche Dinge erwähnen}


\section*{Organisatorisches}
\label{sec:Organisatorisches}
\addcontentsline{toc}{section}{Organisatorisches}
\begin{itemize}
	\item Bewertung
	\begin{itemize}
		\item Arbeitsstil: 40\% - Selbständigkeit, Verständnis, Kreativität, Fleiß, Zusammenarbeit mit Betreuer, Systematik
		\item Ergebnisse: 20\% - Qualität, Nutzbarkeit, Innovations-, Erfüllungsgrad
		\item Ausarbeitung: 30\% - Aufbau, äußere Form, Sprache, Grundlagen, Vollständigkeit
		\item Vortrag: 10\% - Inhalt, Stil, Folien, Vorführung, Diskussion
	\end{itemize}
	\item jeder Studierende sollte an mindestens einem Regelungstechnischen Seminar teilnehmen
	\item Abgabetermin ist ein fixer Termin $\rightarrow$ Prüfungsleistung (Durchfallen möglich)
	\item jeder Studierende erhält auf Wunsch einen eigenen Account für unsere Rechner (Passwort legt Admin fest)
	\item zur Erstellung der Ausarbeitung und der Präsentation wird die Verwendung des Textsatzprogramms \LaTeX empfohlen, das Template für die Ausarbeitung erhaltet Ihr vom Betreuer
\end{itemize}


\section*{Allgemeines zu Hilfsmitteln bei der Erstellung und Bearbeitung}
\label{sec:Allgemeines}
\addcontentsline{toc}{section}{Allgemeines zu Hilfsmitteln bei der Erstellung und Bearbeitung}
\begin{itemize}
	\item es gibt viele Bücher die bei der Erstellung von wissenschaftlichen Arbeiten helfen
	\item viele verwendete Programme bieten ausführliche Hilfen an (z.B. \Matlab, Mathematica)
	\item bei Problemen immer zuerst in der Hilfe schauen, dann "googlen", dann Betreuer fragen
\end{itemize}


\subsection*{Empfohlene Programme}
\label{sec:Empfohlene Programme}
\addcontentsline{toc}{subsection}{Empfohlene Programme}
\begin{itemize}
	\item Literaturdatenbank $\rightarrow$ JabRef (http://jabref.sourceforge.net/)
	\item Ausarbeitung und Präsentation $\rightarrow$ \LaTeX / PowerPoint
	\item Simulation und Regelungstechnik $\rightarrow$ \MatSim
	\item symbolische und numerische Berechnungen $\rightarrow$ Mathematica
	\item Plots $\rightarrow$ \texttt{pgfplots}
	\item Blockschaltbilder $\rightarrow$ \texttt{TikZ}
	\item Versionskontrolle $\rightarrow$ \textsc{Git}
\end{itemize}
