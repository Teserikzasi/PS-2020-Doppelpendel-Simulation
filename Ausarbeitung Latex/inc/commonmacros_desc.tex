\chapter{Befehle in \texttt{commonmacros.tex}}
\label{cha:commonmacros}
Hier sind im Folgenden kurz einige in \texttt{commonmacros.tex} definierte Befehle aufgelistet.

%
\section*{Einheiten}
Die folgenden Befehle funktionieren im Mathe- und Textmodus (\dah es wird im Textmodus automatisch für den Befehl in den Mathemodus umgeschaltet):
\begin{itemize}
	\item Einheit (Aufrechte Schrift im Mathemodus)\\ \verb|\unit{\frac{N}{m}}| $\rightarrow$ \unit{\frac{N}{m}}
	\item Zahl mit Einheit\\(Setzt \glqq{}kleines\grqq{} Leerzeichen zwischen Zahl und Einheit, Zahl und Einheit automatisch im Mathemodus, Einheit in aufrechter Schrift)\\ \verb|\valunit{34,3}{cm}| $\rightarrow$ \valunit{34,3}{cm}
	\item (Das aufrechte $\mu$ gibt es mit dem Befehl \verb|\upmu| aus dem Paket upgreek)\\ \verb|\valunit{4}{\upmu m}| $\rightarrow$ \valunit{4}{\upmu m}
\end{itemize}

\noindent Besondere Einheiten
\begin{itemize}
	\item Gradzeichen (Funktioniert im Text- und Mathemodus)\\ \verb|\degree| $\rightarrow$ \degree
	\item Grad Celsius (Funktioniert im Text- und Mathemodus)\\ \verb|\degC| $\rightarrow$ \degC
\end{itemize}


\section*{Vektoren und Matrizen}
\begin{itemize}
	\item Vektor\\ \verb|\ve{x}| $\rightarrow$ \ve{x}
	\item Matrix\\ \verb|\mat{A}| $\rightarrow$ \mat{A}
	\item Vektor Sonderzeichen\\ \verb|\ves{\lambda}| $\rightarrow$ \ves{\lambda}
	\item Matrix Sonderzeichen\\ \verb|\mas{\Lambda}| $\rightarrow$ \mas{\Lambda}
\end{itemize}
Wichtig: Mathematische Akzente müssen dabei geklammert werden!
\begin{itemize}
	\item \verb|$\dot{\ve{x}}$| $\rightarrow$  $\dot{\ve{x}}$
	\item \verb|$\dot{\tilde{\ve{x}}}$| $\rightarrow$ $\dot{\tilde{\ve{x}}}$
\end{itemize}
\verb|\ve{}| und \verb|\mat{}| \bzw \verb|\ves{}| und \verb|\mas{}| machen jeweils genau das gleiche. Die Unterscheidung dient nur zur besseren Lesbarkeit.

\begin{itemize}
	\item Transponiert-Zeichen (aufrechtes T)\\ \verb|$\mat{A}^\transp$| $\rightarrow$ $\mat{A}^\transp$
\end{itemize}


\section*{Funktionen und Abkürzungen}
\begin{itemize}
	\item Unterstreichen\\ \verb|$\ul{x}$| $\rightarrow$ $\ul{x}$
	\item Innenprodukt\\ \verb|$\inprod{f}{g}$| $\rightarrow$ $\inprod{f}{g}$
	\item Exponentialschreibweise\\ \verb|$45\E{-2}$| $\rightarrow$ $45\E{-2}$
	\item e-Funktion\\ \verb|$\eexp{t}$| $\rightarrow$ $\eexp{t}$
	\item Rang\\ \verb|$\rang{\mat{A}}$| $\rightarrow$ $\rang{\mat{A}}$
	\item Imaginäre Einheit (aufrechtes j)\\ \verb|$5+\iu 2$| $\rightarrow$ $5+\iu 2$
	\item \glqq{}Von-Bis-Punkte\grqq{} mit Kommas und schönen Abständen\\ \verb|$1 \todots n$| $\rightarrow$ $1 \todots n$
	\item i abgeleitet\\ \verb|$\doti$| $\rightarrow$ $\doti$
	\item Aufrechte Schrift (Abkürzung für \verb|\mathrm{}|) \\ \verb|$\mrm{abc}$| $\rightarrow$ $\mrm{abc}$
	\item Normaler Text in Formel (Abkürzung für \verb|\textnormal{}|)\\ \verb|$\tn{ab für}$| $\rightarrow$ $\tn{ab für}$
	\item Geklammerte Gruppe mit Subscript\\ \verb|$\grpsb{\frac{1}{2}}{x}$| $\rightarrow$ $\grpsb{\frac{1}{2}}{x}$
	\item Geklammerte Gruppe mit aufrechtem Subscript\\ \verb|$\grprsb{\frac{1}{2}}{x}$| $\rightarrow$ $\grprsb{\frac{1}{2}}{x}$
\end{itemize}



\section*{Ableitungen und Integrale}
\begin{itemize}
	\item Normale Ableitung\\ \verb|$\normd{f}{x}$| $\rightarrow$ $\normd{f}{x}$
	\item Materielle Ableitung\\ \verb|$\matd{f}{x}$| $\rightarrow$ $\matd{f}{x}$
	\item Partielle Ableitung\\ \verb|$\partiald{f}{x}$| $\rightarrow$ $\partiald{f}{x}$
	\item Beispiel höhere Ableitung\\ \verb|$\normd{^2 f}{x^2} \qquad \partiald{^2 f}{x \partial y}$| $\rightarrow$ $\normd{^2 f}{x^2} \qquad \partiald{^2 f}{x \partial y}$
	\item Normale Ableitung an\\ \verb|$\normdat{f}{x}{x=0}$| $\rightarrow$ $\normdat{f}{x}{x=0}$
	\item Materielle Ableitung an\\ \verb|$\matdat{f}{x}{x=0}$| $\rightarrow$ $\matdat{f}{x}{x=0}$
	\item Partielle Ableitung an\\ \verb|$\partialdat{f}{x}{x=0}$| $\rightarrow$ $\partialdat{f}{x}{x=0}$
	\item Aufrechtes \glqq{}d\grqq{} für Integral\\ \verb|$\ud$| $\rightarrow$ $\ud$
	\item Beispiel für Integral\\ \verb|$\int f(x) \ud x$| $\rightarrow$ $\int f(x) \ud x$
\end{itemize}


\section*{Transformationen}
\begin{itemize}
	\item \verb|$\Laplace{x}$| $\rightarrow$ $\Laplace{x}$
	\item \verb|$\InvLaplace{X}$| $\rightarrow$ $\InvLaplace{X}$
	\item \verb|$x \trans X$| $\rightarrow$ $x \trans X$
	\item \verb|$X \invtrans x$| $\rightarrow$ $X \invtrans x$
	\item \verb|$\FT{x}$| $\rightarrow$ $\FT{x}$
	\item \verb|$\FTabs{x}$| $\rightarrow$ $\FTabs{x}$
	\item \verb|$\IFT{x}$| $\rightarrow$ $\IFT{x}$
	\item \verb|$\DFT{x}$| $\rightarrow$ $\DFT{x}$
	\item \verb|$\DFTabs{x}$| $\rightarrow$ $\DFTabs{x}$
\end{itemize}




\section*{Verweise}
Verweise auf verschiedene Objekte mit passendem Text (\glqq{}Abbildung X\grqq{}, \glqq{}Tabelle X\grqq{}).
Dabei ist dann immer der komplette Text ein Hyperlink, und nicht nur die Zahl.
\begin{itemize}
	\item Abbildung\\ \verb|\figref{label}|
	\item Tabelle\\ \verb|\tabref{label}|
	\item Gleichung\\ \verb|\equref{label}|
	\item Definition\\ \verb|\defref{label}|
	\item Kapitel\\ \verb|\charef{label}|
	\item Anhang\\ \verb|\appendixref{label}|
	\item Abschnitt\\ \verb|\secref{label}|
	\item Listing\\ \verb|\lstref{label}|
	\item Algorithmus\\ \verb|\algoref{label}|
	\item Seite\\ \verb|\pagerefh{label}|
	\item Fußnote\\ \verb|\ftnref{label}|
\end{itemize}

\ZT auch auf Varioref basierend (\glqq{}Abbildung 23 auf dieser Seite\grqq{}, \glqq{}Abbildung 23 auf Seite 45\grqq)
\begin{itemize}
	\item Abbildung\\ \verb|\figvref{label}|
	\item Tabelle\\ \verb|\tabvref{label}|
	\item Gleichung\\ \verb|\equvref{label}|
\end{itemize}


\section*{Abkürzungen}
Abkürzungen mit Punkt \glqq{}dazwischen\grqq{} (wird mit kleinen Abständen gesetzt)\\
\verb|\dah| $\rightarrow$ \dah, \verb|\Dah| $\rightarrow$ \Dah, \verb|\iA| $\rightarrow$ \iA, \verb|\IA| $\rightarrow$ \IA, \verb|\ua| $\rightarrow$ \ua, \verb|\Ua| $\rightarrow$ \Ua, \verb|\uU| $\rightarrow$ \uU, \verb|\UU| $\rightarrow$ \UU, \verb|\zB| $\rightarrow$ \zB, \verb|\ZB| $\rightarrow$ \ZB, \verb|\zT| $\rightarrow$ \zT, \verb|\ZT| $\rightarrow$ \ZT

\vspace{1ex}
\noindent Abkürzungen mit Punkt, bei denen der Punkt nicht als Satzende interpretiert wird:\\
\verb|\bspw| $\rightarrow$ \bspw, \verb|\Bspw| $\rightarrow$ \Bspw, \verb|\bzw| $\rightarrow$ \bzw, \verb|\Bzw| $\rightarrow$ \Bzw,  \verb|\bzgl| $\rightarrow$ \bzgl, \verb|\ca| $\rightarrow$ \ca, \verb|\evtl| $\rightarrow$ \evtl, \verb|\ggf| $\rightarrow$ \ggf, \verb|\Ggf| $\rightarrow$ \Ggf, \verb|\usw| $\rightarrow$ \usw, \verb|\vgl| $\rightarrow$ \vgl, \verb|\Vgl| $\rightarrow$ \Vgl



\section*{Listingdefintionen}
\begin{itemize}
	\item \verb|Matlab_colored|
	\item \verb|Matlab_colored_smallfont|
\end{itemize}


\begin{lstlisting}[style=Matlab_colored, caption = {Beispiellisting, style=Matlab\_colored}, label={lst:Listing1}]
function [] = animierePunkt(inY, inX)

temp = length(inY);

%% [...]

%% -------------------------------------------------------------
for i=1:temp
    if i>1
        delete(p(i-1));
    end
    p(i) = plot(inX(i),inY(i),'Marker','o','MarkerSize',10);
    pause(0.025);
end
hold off;
\end{lstlisting}


\begin{lstlisting}[style=Matlab_colored_smallfont, caption = {Beispiellisting, style=Matlab\_colored\_smallfont}, label={lst:Listing2}]
function [] = animierePunkt(inY, inX)

temp = length(inY);

%% [...]

%% ---------------------------------------------------------------------
for i=1:temp
    if i>1
        delete(p(i-1));
    end
    p(i) = plot(inX(i),inY(i),'Marker','o','MarkerSize',10);
    pause(0.025);
end
hold off;
\end{lstlisting}

\section*{Sonstiges}
Latex gibt beim Umwandeln \zT Fehler aus, wenn Zeichen aus dem \texttt{textcomp}-Paket verwendet werden, da diese nicht in den TU-Schriften vorhanden sind.
Mit \verb|\textcompstdfont{}| wird die Schriftart für den Text im Argument explizit umgeschaltet, und so der Fehler vermieden:
\begin{itemize}
	\item \verb|\textcompstdfont{\textuparrow}| $\rightarrow$ \textcompstdfont{\textuparrow}
\end{itemize}