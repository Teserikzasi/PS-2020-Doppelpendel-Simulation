\chapter{Checkliste}
\label{cha:Checkliste}

Von der Arbeit sind zwei gedruckte Exemplare (doppelseitig; gebunden mit dunkelblauem oder schwarzem Karton; vorne Klarsichtfolie; Klebe- oder Leimbindung, keine Spiralbindung) für die Bibliothek und den Betreuer abzugeben, sowie eine Datei im \emph{Portable Document Format} (PDF).
Die gedruckten Exemplare müssen jeweils eine CD oder DVD mit allen Materialien zum Reproduzieren der Ergebnisse der Arbeit enthalten, \dah den Quelltext (\bspw Matlabdateien und Simulinkmodelle) der Implementierung, den Quelltext der Ausarbeitung und des Vortrages sowie die elektronisch vorhandene zitierte (und möglicherweise in Betracht gezogene aber nicht zitierte) Literatur.
Darüber hinaus ist eine elektronische Version der Arbeit im PDF/A Format in \href{https://tubama.ulb.tu-darmstadt.de/}{\textsc{TUBaMa}} abzugeben, die separat durch Auskommentieren von \texttt{\textbackslash{}SADATUbamafalse} in der Hauptdatei erzeugt werden muss.
Dazu sollten folgende Punkte überprüft werden:

% Der folgende Befehl ändert die Aufzählungszeichen in Kästen. Diese Definition
% ist im gesamten restlichen Dokument wirksam.

\renewcommand{\labelitemi}{%
  \setlength{\fboxsep}{.6ex}\raisebox{.7ex}{\framebox{}}}

\textbf{Im Quelltext}
\begin{itemize}
	\item Wurden alle der Literatur entnommenen Stellen mit Literaturverweisen belegt?
	\item keine wörtlichen Zitate, falls doch müssen diese in Anführungszeichen!
	\item Alle Literaturangaben im Literaturverzeichnis vollständig? (Autor(en), Titel, Verlag/Journal/Konferenzband/\etc, Jahr, \usw)
	\item Alle Seitenangaben im Inhaltsverzeichnis korrekt?
	\item Keine einzelnen Abschnitte/Unterabschnitte im Literaturverzeichnis?
	\item Prägnante Kapitelnamen? (keine ganzen Sätze!)
	\item Alle Bilder haben Bild\emph{unterschriften}, alle Tabellen haben \emph{Überschriften}?
	\item Wurden die Hinweise aus \charef{cha:Hinweise_Latex} berücksichtigt
	(allgemeiner Aufbau, Abstände, Sonderzeichen, \ldots)\vspace{.1em}?
	\item Sind die Bilder gut erkennbar und alle Elemente beschriftet\vspace{.1em}?
	Passt die Schriftgröße in den Bildern zum Text\vspace{.1em}? Sitzen die
	Gleitobjekte an der richtigen Stelle\vspace{.1em}?
	\item Treten beim Aufruf von \TeX\ bzw.\ pdf\TeX\ Fehler oder Warnungen
	auf\vspace{.1em}? Sind \emph{alle} Bilder, Tabellen und Literaturstellen im
	Text zitiert\vspace{.1em}? Sind falsche oder doppelte Referenzen
	vorhanden\vspace{.1em}? Dies lässt sich anhand der Log-Datei feststellen.
\end{itemize}

\textbf{In der PDF-Datei}
\begin{itemize}
	\item Die Datei ist im \textbf{doppelseitigen} Layout mit Hypertext-Elementen zu
	erstellen; die Optionen \verb|draft|, \verb|oneside| und \verb|nohyperref|
	sind \textbf{nicht} aktiviert. Die Seitengröße des PDF-Doukuments überprüfen:
	210\,mm $\times$ 297\,mm (DIN A4).
	\item Sind die PDF-Bookmarks und Seitenzahlen vorhanden\vspace{.1em}? Zumindest
	im Vorspann sollte überprüft werden, ob die Bookmarks auf die richtige Seite
	verweisen.
	\item Sind die PDF-Infofelder (in Acrobat Reader: Datei $\rightarrow$
	Dokumenteigenschaften) richtig eingetragen\vspace{.1em}? Notfalls mit
	\verb|\hypersetup{...}| korrigieren.
	\item In den Bookmarks und Infofeldern können nicht alle Zeichen dargestellt
	werden. In einem solchen Fall \zB\ \verb|\texorpdfstring| (aus
	\verb|hyperref|) verwenden.
	\item Ist die für \textsc{TUBaMa} erzeugte Datei eine gültige PDF/A Datei?
\end{itemize}

\textbf{Im Ausdruck}
\begin{itemize}
	\item Die Arbeit ist \textbf{doppelseitig}, vorzugsweise schwarzweiß auf einem
	Laserdrucker auszudrucken. Sind Alle Grafiken gut zu erkennen (Farbe, Linienstärke, \etc)\vspace{.1em}?
	Wurden alle Sonderzeichen korrekt gedruckt\vspace{.1em}?
	\item Wurde eine Klebe- oder Leimbindung zum Binden verwendet?
	\item Beim Ausdrucken aus dem Acrobat Reader darf die Seitenanpassung
	\textbf{nicht} aktiviert sein, da der Textblock sonst verkleinert wird. Linker Rand muss 30\,mm,
	rechter Rand muss 20\,mm betragen (nachmessen).
	\item Selbstständigkeitserklärung unterschreiben!
	\item Es ist eine zusätzliche eigenständige Selbstständigkeitserklärung auszudrucken und zu unterschreiben.
\end{itemize}



\chapter{Programme zur Erstellung von Grafiken}
\label{cha:Anhang-Grafiken}

\section{Vektorgrafiken}

Vektorgrafiken bestehen aus geometrischen Formen, deren Beschreibung unabhängig
von der Auflösung ist. Sie sind \zB\ für Diagramme und mathematische Plots
sinnvoll und lassen sich aus vielen Programmen direkt als EPS oder PDF
speichern. Es ist besonders darauf zu achten, dass die Schriften und
Strichstärken zum Rest des Dokuments passen.

\renewcommand{\labelitemi}{$\bullet$}
	
\begin{itemize}
\item \LaTeX\ selbst bietet mit den Paketen \texttt{TikZ} und \texttt{pgfplots} 
	zwei sehr mächtige Werkzeuge um direkt in \LaTeX\ Vektorgrafiken, wie zum Beispiel
	Blockschaltbilder oder Plots zu erzeugen. Schriftart und -größe sind automatisch identisch
	mit dem übrigen laufenden Text, so dass hier keine Anpassungen mehr vorgenommen werden müssen.
	\begin{itemize}
		\item \texttt{TikZ} eignet sich hervorragend für Blockschaltbilder
		\item Mit \texttt{pgfplots} können aus einer \texttt{.txt}-Datei mit den Variablenwerten
					Plots direkt in \LaTeX\ erzeugt werden, so dass sich ein einheitliches
					Gesamtbild ergibt. Die benötigte \texttt{.txt}-Datei lässt sich mit \Matlab\ einfach
					erzeugen. Achsenbeschriftungen, Legenden, \etc können mit \Matlab-ähnlichen
					Befehlen einfach hinzugefügt werden.
	\end{itemize}
	Die Dokumentationen zu \texttt{TikZ}~\cite{TikZ} und \texttt{pgfplots}~\cite{pgfplots} sind
	sehr ausführlich und mit vielen Beispielen leicht verständlich erklärt.
	
	
\item \Matlab\ kann über den Befehl \verb|print -deps name.eps| Grafiken direkt
  als EPS abspeichern. Allerdings wird die Grafik so skaliert, als würde sie auf
  einem Drucker ausgegeben. Deshalb müssen unbedingt die Schrift/Linien oder die
  Größe angepasst werden. Hierzu gibt es verschiedene Techniken. Der direkte
  PDF-Export aus MATLAB ist derzeit noch nicht zu empfehlen.
\item Zeichenprogramme können in der Regel ebenfalls EPS direkt exportieren.
  Der am IAT verwendete Adobe Illustrator erzeugt auch sehr gutes PDF, da sein
  eigenes Format fast identisch mit PDF ist.
\item Bei allen Programmen, die eine Druckfunktion besitzen, kann man die
  Ausgabe eines beliebigen PostScript-Druckers in eine Datei umlenken. Dazu
  stellt man den Treiber auf das EPS-Format um, so dass nur noch in den
  wenigsten Fällen eine Nachbearbeitung der PostScript-Ausgabe \zB\ mit GSview
  erforderlich ist.
\item Liegt eine Grafik als \emph{Windows Metafile} (\verb|*.wmf|) oder
  \emph{Enhanced Metafile} (\verb|*.emf|) vor, lässt sie sich mit dem Tool
  WMF2EPS konvertieren. Zahlreiche Windows-Programme bieten eine
  Export-Möglichkeit in diese beiden Formate, allerdings kann es aufgrund des
  einfachen Grafikmodells zu Verlusten kommen.
\item Arbeitet ein Progamm mit der Windows-Zwischenablage zusammen, kann über
  WMF2EPS eine EPS-Datei direkt aus der Zwischenablage erstellt werden. Für
  Vektorgrafiken sollte WMF2EPS aber generell nur im Notfall für verwendet
  werden.
\end{itemize}
Zur Verwendung in pdf\TeX\ lässt sich eine Grafik, die im EPS-Format vorliegt,
einfach in PDF konvertieren. Hierfür eignen sich der Acrobat Distiller oder
Ghostscript in Verbindung mit EPSTOPDF (das in allen gängigen TeX-Distributionen
enthalten ist).


\section{Pixelgrafiken}

Pixelgrafiken besitzen eine feste Anzahl von Bildpunkten, d.\,h.\ ihre
Auflösung hängt von der Größe der Darstellung ab. Man benutzt Pixelformate
\zB\ für Fotos, Screenshots oder eingescannte Grafiken. Hierbei ist die Wahl der
Auflösung besonders wichtig. Die Bilder sollen einerseits eine gute
Druckqualität ergeben, andererseits aber auch eine zügige Bildschirmdarstellung
und kleine Dateigröße ermöglichen. Beim Scannen ist außerdem zu beachten, dass
sich die Auflösung ändert, wenn die Grafiken nicht in Originalgröße eingebunden
werden.
\begin{itemize}
\item Fotos liegen in der Regel im JPEG-Format vor; eine Auflösung von
  $100$--$\valunit{150}{dpi}$ ist häufig bereits ausreichend. Während pdf\TeX\ eine
  JPEG-Grafik direkt einlesen kann, muss sie für \TeX/Dvips nach EPS konvertiert
  werden. Das Kommandozeilen-Tool JPEG2PS erledigt dies, ohne die
  JPEG-Kompression zu verlieren.
\item Sonstige Farb- oder Graustufen-Grafiken (insb.\ wenn sie \glqq harte\grqq\
  Farbübergänge besitzen) werden am besten zunächst im PNG-Format gespeichert.
  Die richtige Wahl der Farbtiefe hat großen Einfluss auf die spätere
  Dateigröße. Bzgl.\ der Auflösung gibt es hier keine allgemeine Regel; oft
  liegt sie bereits fest (\zB\ bei Screenshots). pdf\TeX\ kann PNG-Grafiken
  direkt verarbeiten, für \TeX/Dvips müssen sie wieder in EPS konvertiert
  werden. Dazu kopiert man die Grafik mit einer beliebigen Pixelgrafik-Software
  in die Zwischenablage und benutzt WMF2EPS.
\item Schwarzweiße Strichzeichnungen müssen in relativ hoher Auflösung
  ($\ge$300\,dpi) vorliegen, um beim Drucken eine ausreichende Qualität zu
  gewährleisten. In Fall von \TeX/Dvips speichert man sie am besten im
  TIFF-Format mit der sog.\ \emph{CCITT Group 4} Kompression und konvertiert sie
  über die Zwischenablage und WMF2EPS in eine EPS-Datei. Da pdf\TeX\ keine
  TIFF-Grafiken einlesen kann, ersetzt man sie hier durch PNG oder bettet die
  TIFF-Grafik in eine PDF-Datei ein (\zB\ mit Adobe Acrobat), um die bessere
  Kompression zu erhalten.
\end{itemize}



\chapter{Das \TeX-System}\label{cha:TexSystem}

Die Klasse \verb|tudreport| unterstützt die Workflows \LaTeX\ $\rightarrow$ DVI
$\rightarrow$ PostScript $\rightarrow$ PDF und \LaTeX\ $\rightarrow$ PDF. Dazu
sind mehrere Programme nötig, die zusammenfassend als \emph{\TeX-System}
bezeichnet werden. Hinzu kommen noch verschiedene Hilfsprogramme.


\section*{Bestandteile des \TeX-Systems}

Als Basis wird am IAT derzeit die \TeX-Distribution Mik\TeX\ 2.9 unter
Windows~7 eingesetzt. Darin sind neben dem
(pdf)\TeX-Interpreter und \LaTeX\ (samt einer Vielzahl von Zusatzpaketen) auch
Programme wie Yap (DVI-Previewer) und Dvips (zum Konvertieren von DVI in
PostScript) enthalten.

Zusätzlich benötigt man noch Ghostscript und GSview zum Ansehen von
PostScript-Dateien und zum Konvertieren von PostScript und EPS in PDF. Im
Hinblick auf die Konvertierung ist vor allem das aktuelle Ghostscript 8.$x$ zu
empfehlen. Um die fertige PDF-Datei ansehen zu können, muss schließlich noch ein PDF-Viewer (\zB
Acrobat Reader oder SumatraPDF) installiert sein. Alternativ kann auch das kommerzielle
Acrobat-Paket verwendet werden, das mit dem Acrobat Distiller eine sehr gute
Möglichkeit zum Konvertieren von PostScript/EPS in PDF enthält.

Um die Bedienung der einzelnen Teilprogramme zu erleichtern, kommen spezielle
\TeX-Shells zum Einsatz. Diese besitzen neben einem komfortablen Editor
(Rechtschreibprüfung, Syntax-Highlighting, \ldots) Bedienelemente zum Aufruf von
\LaTeX, Dvips usw. Am IAT ist auf den meisten Rechnern das Shareware-Programm
WinEdt installiert, aber auch das kostenlose \TeX{}nicCenter ist inzwischen sehr
gut. Eine weitere Möglichkeit ist der Editor Emacs mit der Erweiterung AUC\TeX.
Die letzte Komponente sind die beiden Grafik-Tools JPEG2PS und WMF2EPS aus
Abschnitt~\ref{sec:Latex-Bilder}.

\section*{Umgang mit dem \TeX-System}

Während der \TeX-Interpreter sein eigenes Format DVI liefert, das dann weiter in
PostScript und PDF konvertiert wird, ist pdf\TeX\ eine neue \TeX-Variante, die
direkt PDF produziert. Beide Wege funktionieren problemlos; aus Sicht des
Anwenders unterscheiden sie sich in erster Line beim Einbinden von Grafiken,
siehe \tabref{tab:Einbinden} und \secref{sec:Latex-Bilder}.
\begin{table}[tbp]
  \caption{Einbinden von Grafiken in \TeX-Dokumente.}
  \label{tab:Einbinden}\centering
  \begin{tabular}{lcc}\hline
    Format & \TeX/Dvips           & pdf\TeX  \\ \hline
    EPS    & (direkt)             & Acrobat, EPSTOPDF \\
    PDF    & Acrobat, Ghostscript & (direkt) \\
    JPEG   & JPEG2PS              & (direkt) \\
    PNG    & WMF2EPS              & (direkt) \\
    TIFF   & WMF2EPS              & über PDF \\ \hline
  \end{tabular}
\end{table}
\TeX/Dvips ist von Vorteil, wenn viele Grafiken als EPS vorliegen. Für pdf\TeX\
entscheidet man sich, wenn vorwiegend Pixel-Bilder und PDF-Grafiken (\zB\ aus
Adobe Illustrator) vorhanden sind.

Im Fall von \TeX/Dvips reicht es während der Erstellung des Dokuments in der
Regel aus, eine DVI-Datei zu erzeugen. Gelegentlich sollte man diese aber auch
in PostScript und weiter in PDF konvertieren und das Ergebnis überprüfen. Bei
pdf\TeX\ arbeitet man ohnehin nur mit der PDF-Datei.

Beim Schreiben ist es manchmal hilfreich, wenn man die Option \verb|draft| verwendet. Statt der Bilder werden dann nur Rahmen
der gleichen Größe angezeigt, und es sind alle Stellen markiert, an denen der
Text über den Rand hinausragt. Solange man an der Arbeit schreibt, empfiehlt es
sich, bei den Umgebungen \verb|figure| und \verb|table| keine Positionsangaben
zu verwenden (insb. nicht \verb|[h]| oder gar \verb|[h!]|). Erst \emph{ganz am
Ende} überprüft man die Platzierung und ändert sie bei Bedarf durch
Positionsangaben, durch Verschieben im Quelltext oder mit Hilfe des
\verb|flafter|-Pakets. In jedem guten \LaTeX-Buch ist außerdem beschrieben, wie
man den Mechanismus zur Positionierung für das gesamte Dokument anpassen kann.

Häufig tritt auch die Frage auf, wie oft das Dokument mit \LaTeX\ bearbeitet
werden muss, damit alle Bezüge und Referenzen stimmen. \LaTeX\ legt während der
Bearbeitung Dateien (\verb|*.aux|, \verb|*.toc|, \verb|*.lof|, usw.) an, in
denen Informationen über die Gliederung, Bilder, Tabellen usw.\ abgelegt sind.
Bei einem erneuten Aufruf von \LaTeX\ werden diese ausgelesen und die
Informationen an den entsprechenden Stellen eingesetzt. Daraus ergeben sich
folgende Regeln:
\begin{itemize}
\item \textbf{$1\times$} bearbeiten bei kleineren Änderungen.
\item \textbf{$2\times$} bearbeiten, falls sich die Referenzen verändern,
   d.\,h.\ wenn neue Gleichungen oder Bilder eingefügt werden oder wenn sich
  Seitenumbrüche verschieben.
\item \textbf{$3\times$} bearbeiten, wenn sich ein Verzeichnis (Inhalt, Bilder,
  \ldots) um eine Seite verlängert, weil dann die Seitenzahlen im Verzeichnis
  nicht mehr stimmen.
\end{itemize}

Da während des Schreibens häufig nur kleine Änderungen zwischen zwei Durchläufen
vorgenommen werden, genügt meist der einmalige Aufruf von \LaTeX. Spätestens bei
der nächsten Iteration stimmen die Bezüge wieder. Bei der Benutzung von
\textsc{Bib}\TeX\ sollte \LaTeX\ \emph{vorher} 1$\times$ und \emph{nachher}
2$\times$ aufgerufen werden, damit alle Zitate stimmen.

